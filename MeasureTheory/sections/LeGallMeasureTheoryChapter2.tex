\chapter{Integration of Measurable Functions - Chapter 2}

\begin{remark}
	There is a very beautiful parallel between the notion of the partial sums of a series and the definition of integral for measurable functions. To see this, let $ f: E \to [0,\infty) $ be any measurable function, where $ (E,\mathcal{A}, \mu) $ is a measurable space. We define the integral of $ f $ as 
	\[ \int f d\mu = \sup_{h\in\mathcal{E}^+, h\leq f} \int h d\mu. \]
	In other words, the integral of the non-negative function $ f $ is defined as the sup of the integral of the approximating simple functions that are less than $ f $. This definition makes a complete sense if we consider the domain of $ f $ to be $ \N $ and the measure $ \mu $ to be the counting measure. We need to highlight that $ f $ is non-negative function. Then the definition above is simply
	\[ \sum_{n=1}^{\infty}f(n) = \lim_{m\to\infty} \sum_{n=1}^m f(n), \]
	since $ \sum_{n=1}^{m} f(n) $ is same as $ \sum_{n=1}^{\infty} g_m(n) $ where $ g_m $ is a simple function such that has same values as $ f $ but has value equal to zero for $ n>m $), and we have $ g_m\leq f $. In short, $ g_m $ takes finitely many values and we have $ g_m\leq f $. 
	
	In conclusion, the definition of the integral for measurable functions is the same as the definition of the value of a series, which is the limit of its partial sums, and in the case that $ f $ is positive, it is the same as the sup of the partial sums. Each partial sum is in fact an integral (infinite series) of a simple function like $ g_m $ as discussed above
\end{remark}