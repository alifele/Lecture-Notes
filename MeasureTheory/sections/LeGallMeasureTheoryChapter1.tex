\chapter{Le Gall Measure Theory - Chapter 1}



\section{Measure Spaces}
In a nutshell, the idea of measure is to assign a real number to \emph{some} subsets of a given set. However, we require this number assignment to be consistent in the sense that the number assigned to the disjoint union of sets is the sum of the numbers assigned to each set. This notion is quire similar to the notion of the homomorphisms of groups which preserves the group action.

\begin{remark}
	Because of some technical issues that are related to the axiom of choice, sometimes we do not consider defining a measure for all subsets of a given set. However, a $\sigma\text{-algebra}$ has sufficient structure to define the notion of measure for them.
\end{remark}


\begin{remark}[Definition of Borel $\sigma\text{-algebra}$]
	Often, the space for which its subsets we want to define a measure, is a topological space. We have a special name reserved for the smallest sigma algebra containing all the open sets. Let $ (\Omega,\mathcal{T}) $ be a topological space. The smallest sigma algebra containing $ \mathcal{T} $ (or alternatively, the sigma algebra generated by all open sets $ \mathcal{T} $), is called a Borel $\sigma\text{-algebra}$.
\end{remark}



\section{Solved Problems}

\begin{problem}[Properties of measure, page 6]
	Let $ (E, \mathcal{A}, \mu) $ be a measure space. Prove the following properties of the measure function $ \mu $.
	\begin{enumerate}[(a)]
		\item If $ A\subset B $ then $ \mu(A) \leq \mu(B) $. If in addition $ \mu(A) < \infty $, then
		\[ \mu(B\backslash A) = \mu(B) - \mu(A). \]
		\item \textbf{Principle of inclusion-exclusion}: Let $ A.B \in \mathcal{A} $, then
		\[ \mu(A) + \mu(B) = \mu(A\cup B) + \mu (A\cap B). \]
		\item \textbf{Continuity from below}: Let $ \set{A_n} $ be a collection of sets that $ A_n \uparrow A $, meaning $ A_1\subset A_2 \subset\cdots $ and we have $ A = \bigcup_n A_n $. Then
		\[ \mu(A_n) \to \mu(A) \quad \text{as $ n\to\infty $}. \]
		In other words
		\[ \mu(\bigcup_n A_n) = \lim_n \mu(A_n). \]
		\item \textbf{Continuity from above}: Let $ \set{A_n} $ be a collection of sets that $ A_n \downarrow A $, meaning $ A_1 \supset A_2 \supset \cdots $ and we have $ A = \bigcap_n A_n $. Then
		\[ \mu(A_n) \to \mu(A) \quad \text{as $ n\to\infty $}. \]
		\item \textbf{Sub-additivity}: Let $ \set{A_n} $ be a collection of sets in $ \mathcal{A} $. Then
		\[ \mu(\bigcup_n A_n) \leq \sum_n \mu(A_n). \]
	\end{enumerate}
\end{problem}
\begin{solution}
	\begin{enumerate}[(a)]
		\item 
		First, observe that we can write
		\[ B = A \cup (B\cap A^c). \]
		Since the union above is disjoint, then by $ \sigma $-additivity we have
		\[ \mu(B) = \mu(A) + \mu(B\cal A^c). \]
		Since measure is always positive, then it follows that
		\[ \mu(B) \geq \mu(A). \]
		In particular, observing that $ B\cap A^c = B\backslash A $ and using the fact that $ \mu(A)<\infty $, then
		\[ \mu(B\backslash A) = \mu(B) - \mu(A). \]
		\item We can write
		\[ A \cup B = A \cup (B \backslash (A\cap B)). \]
		Since the union above is disjoint, we have
		\[ \mu(A\cup B) = \mu(A) + \mu(B\backslash (A\cap B)) = \mu(A) + \mu(B) - \mu(A\cap B). \]
		\item Let $ B_1 = A_1 $ and define $ B_2 = A_2\backslash A_1, B_3 = A_3 \backslash A_2 $, and etc. Notice that 
		\[ A = \dot\bigcup_n B_n. \]
		By sigma additivity of $ \mu $ we have
		\[ \mu(A) = \sum_n \mu(B_n) \]
		Observe that $ \mu(A_N) = \sum_{n=1}^N \mu(B_n) $. Thus
		\[ \mu(A) = \lim_{n\to\infty} \mu(A_n). \]
		\item This follows easily by using the fact that $ A_n^c \uparrow A^c $ and $ A^c = \bigcup A_n^c $. Thus by continuity from below
		\[ \mu(A^c) = \lim_{n\to\infty}\mu(A^c_n). \]
		We can write
		\[ 1 - \mu(A) = \lim_{n\to\infty}(1 - \mu(A_n)) = 1 - \lim_{n\to\infty}\mu(A_n), \]
		where we have used the fact that $ \lim_{n\to\infty}\mu(A_n) $ exists. Because by assumption $ A_1 \supset A_2 \supset A_3 \cdots $ and by the monotonicity of the measure $ \mu(A_n) $ is a positive decreasing sequence, thus bounded from below. So it converges. 
		
		\item To see this define
		\[ B_n = A_n \backslash (\bigcup_{k=1}^{n-1} A_k). \]
		Notice that $ \bigcup_n A_n = \bigcup_n B_n $ and $ B_n \subseteq A_n $. Also note that $ B_n $s are disjoint. So
		\[ \mu(\bigcup_n A_n) = \mu(\bigcup_n B_n) = \sum_n \mu(B_n) \leq \sum_n \mu_(A_n). \]
	\end{enumerate}
\end{solution}


\begin{remark}
	Probability style proof of item (2) above: Consider
	\[ 1 - \prod_{i}(1 - \mathds{1}_{A_i}) = \sum_i \mathds{1}_{A_i} - \sum_{i<j}\mathds{1}_{A_i}\mathds{1}_{A_j} + \sum_{i<j<k}\mathds{1}_{A_i}\mathds{1}_{A_j}\mathds{1}_{A_k}+\cdots  \]
	Noting the fact that $ 1 - \mathds{1}_{A} = \mathds{1}_{A^c} $, $ \mathds{1}_{A_i}\mathds{1}_{A_j} = \mathds{1}_{A_i\cap A_j} $, $ \E{\mathds{1}_{A}} = \prob(A) $, and applying the expected value to both sides we will get
	\[ \E{1 - \prod_i (1-\mathds{1}_{A_i})} = \sum_i \mu(A_i) - \sum_{i<j}\mu(A_i\cap A_j) + \sum_{i<j<k}\mu(A_i\cap A_j \cap A_k) - \cdots \pm \mu(A_1\cap\cdots\cap A_n). \]
	Notice that the left hand side is
	\[ \E{1 - \prod_i (1-\mathds{1}_{A_i})} = \E{1 - \prod_i \mathds{1}_{A_i^c}} = \E{1 - \mathds{1}_{\cap_i A_i^c}} = \E{\mathds{1}_{\cup A_i}} = \mu(A_1\cup\cdots\cup A_n). \]
\end{remark}


\begin{proposition}
	Let $ \set{A_n} $ be a collection of measurable sets. We have
	\[ \mu(\liminf_n A_n) \leq \liminf_n \mu(A_n). \]
	Also
	\[ \mu(\limsup_n A_n) \geq \limsup_n \mu(A_n). \]
	Using the fact that $ \limsup_n \mu(A_n) \geq \liminf_n \mu(A_n) $ we can write
	\[ \mu(\liminf_n A_n) \leq \liminf_n \mu(A_n) \leq \limsup_n \mu(A_n) \leq \mu(\limsup_n A_n). \]
\end{proposition}
\begin{proof}
	First, observe that
	\[ \bigcap_{k\geq n} A_n \subseteq A_m \qquad \forall m\geq n. \]
	By monotonicity
	\[ \mu(\bigcap_{k\geq n} A_n) \leq \mu (A_m) \qquad \forall m \geq n. \]
	So it follows that 
	\[ \mu(\bigcap_{k\geq n} A_n) \leq \inf_{k\geq n}\mu(A_k).  \]
	Using the fact that $ \bigcap_{k\geq n} A_n \uparrow \liminf_n A_n $, and thus 
	\[\lim_{n\to\infty} \mu(\bigcap_{k\geq n} A_k) = \mu(\liminf_n A_n) \]
	we can take limit of both sides
	\[ \lim_{n\to\infty} \mu(\bigcap_{k\geq n}A_n) \leq \lim_{n\to\infty }\inf_{k\geq n} \mu(A_k). \]
	Using the fact that. Then we will have
	\[ \mu(\liminf_n A_n) \leq \liminf_n \mu(A_n). \]
\end{proof}


