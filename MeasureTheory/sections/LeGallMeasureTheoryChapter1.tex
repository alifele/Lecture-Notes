\chapter{Le Gall Measure Theory - Chapter 1}



\section{Measure Spaces}
In a nutshell, the idea of measure is to assign a real number to \emph{some} subsets of a given set. However, we require this number assignment to be consistent in the sense that the number assigned to the disjoint union of sets is the sum of the numbers assigned to each set. This notion is quire similar to the notion of the homomorphisms of groups which preserves the group action.

\begin{remark}
	Because of some technical issues that are related to the axiom of choice, sometimes we do not consider defining a measure for all subsets of a given set. However, a $\sigma\text{-algebra}$ has sufficient structure to define the notion of measure for them.
\end{remark}


\begin{remark}[Definition of Borel $\sigma\text{-algebra}$]
	Often, the space for which its subsets we want to define a measure, is a topological space. We have a special name reserved for the smallest sigma algebra containing all the open sets. Let $ (\Omega,\mathcal{T}) $ be a topological space. The smallest sigma algebra containing $ \mathcal{T} $ (or alternatively, the sigma algebra generated by all open sets $ \mathcal{T} $), is called a Borel $\sigma\text{-algebra}$.
\end{remark}




\begin{proposition}
	Let $ \set{A_n} $ be a collection of measurable sets. We have
	\[ \mu(\liminf_n A_n) \leq \liminf_n \mu(A_n). \]
	Also
	\[ \mu(\limsup_n A_n) \geq \limsup_n \mu(A_n). \]
	Using the fact that $ \limsup_n \mu(A_n) \geq \liminf_n \mu(A_n) $ we can write
	\[ \mu(\liminf_n A_n) \leq \liminf_n \mu(A_n) \leq \limsup_n \mu(A_n) \leq \mu(\limsup_n A_n). \]
\end{proposition}
\begin{proof}
	First, observe that
	\[ \bigcap_{k\geq n} A_n \subseteq A_m \qquad \forall m\geq n. \]
	By monotonicity
	\[ \mu(\bigcap_{k\geq n} A_n) \leq \mu (A_m) \qquad \forall m \geq n. \]
	So it follows that 
	\[ \mu(\bigcap_{k\geq n} A_n) \leq \inf_{k\geq n}\mu(A_k).  \]
	Using the fact that $ \bigcap_{k\geq n} A_n \uparrow \liminf_n A_n $, and thus 
	\[\lim_{n\to\infty} \mu(\bigcap_{k\geq n} A_k) = \mu(\liminf_n A_n) \]
	we can take limit of both sides
	\[ \lim_{n\to\infty} \mu(\bigcap_{k\geq n}A_n) \leq \lim_{n\to\infty }\inf_{k\geq n} \mu(A_k). \]
	Using the fact that. Then we will have
	\[ \mu(\liminf_n A_n) \leq \liminf_n \mu(A_n). \]
\end{proof}


\subsection{Measurable Functions}

The following proposition comes very handy when checking that some function between to measure spaces is a measurable function. In a nutshell, to check to see a function from $ (\Omega, \mathcal{F}) $ to $ (\Gamma, \mathcal{E}) $ is measurable, we do not need to check to see if the pre-image of all sets in $ \mathcal{E} $ is measurable. We only need to check this for a collection of sets that the $\sigma\text{-algebra}$ that they generate is $ \mathcal{E} $.

\begin{proposition}
	Let $ (\Omega, \mathcal{F}) $ and $ (\Gamma, \mathcal{E}) $ be two measure spaces and $ F: \Omega \to \Gamma $ a function. Then $ F $ is a measurable function if there exists some subclass $ C \subset \mathcal{E} $ such that $ \mathcal{E} = \sigma(C) $ and $ \inv{F}(B) \in \mathcal{F} $ for all $ B \in  C $.
\end{proposition}
\begin{proof}
	To see this, let 
	\[ \mathcal{G} = \set{B \in \mathcal{E}: \inv{F}(B) \in \mathcal{F}}. \]
	By assumption $ C \subset \mathcal{G} $. Since $ \inv{F}(A^c) = (\inv{F}(A))^c $, $ \inv{F}(\bigcup_n A_n) = \bigcup_n \inv{F}(A_n) $, and $ \inv{F}(\bigcap_n A_n) = \bigcap_n \inv{F}(A_n) $, it is easy to check that $ \mathcal{G} $ is a $\sigma\text{-algebra}$.  So $ C \subseteq \mathcal{G} $. Because $ \mathcal{E} = \sigma(C) $, then we conclude that $ \mathcal{G} = \mathcal{E} $ and this completes the proof.
\end{proof}


One of the very important applications of the proposition above is the following Lemma.
\begin{lemma}
	Let $ (E,\mathcal{A}) $, $ (F_1,\mathcal{B}_1) $ and $ (F_2,\mathcal{B}_2) $ be measurable spaces and equip the product space $ F_1\times F_2 $ with $ \mathcal{B}_1 \otimes \mathcal{B}_2 $ (that is defined to be $ \sigma(B_1\times B_2) $). Let $ f_1: E\to F_1 $ and $ f_2: E \to F_2 $ be measurable functions and define $ f: E \to F_1\times F_2 $ where $ f = (f_1,f_2) $. Then $ f $ is measurable if and only if $ f_1 $ and $ f_2 $ are measurable.
\end{lemma}
\begin{proof}
	For the $ \boxed{\Longrightarrow} $ direction, we know that $ f $ is measurable. So $ \inv{f}(B) \in \mathcal{A} $ for $ B \in \mathcal{B}_1\otimes \mathcal{B}_2 $. Let $ B_1\in\mathcal{B}_1 $ and $ B_2 \in \mathcal{B}_2 $, and define $ B = B_1\times B_2 $. Then since $ \inv{f}(B_1\times B_2) = \inv{f_1}(B_1) \cap \inv{f_2}(B_2) $, and $ \inv{f}(B_1\times B_2) \in \mathcal{A} $, we conclude that $ \inv{f}_1(B_1)\in \mathcal{A} $ and $ \inv{f}_2(B_2) \in \mathcal{A} $. Thus $ f_1 $ and $ f_2 $ are both measurable.
	
	For the $ \boxed{\Longleftarrow} $ direction, we know that $ \inv{f}_1 $ and $ \inv{f}_2 $ are measurable. So for any $ B_1 \in \mathcal{B}_1 $ and $ B_2\in \mathcal{B}_2 $ we have $ \inv{f}(B_1\times B_2) = \inv{f}_1(B_1) \cap \inv{f}_2(B_2) \in \mathcal{A} $ because $ \inv{f}_1(B_1) \in \mathcal{A} $ and $ \inv{f}_2(B_2)\in \mathcal{A} $. By proposition above, we can conclude that $ f $ is measurable on $ \sigma(\mathcal{B}_1\times \mathcal{B}_2) $.n 
\end{proof}



The following two simple properties of $ \inf $ and $ \sup $ are very important in establishing very interesting results.
\begin{lemma}[Simple properties of $ \inf $ and $ \sup $]
	Let $ \set{a_n} $ be a sequence of real numbers, and $ c \in \R $. Then
	\[ \sup_n a_n \leq c \implies \forall n \in \N\ a_n \leq c, \]
	and
	\[ \inf_n a_n \leq c \implies \exists n \in \N \st a_n \leq c. \]
\end{lemma}
\begin{proof}
	Proof follows immediately from the definition.
\end{proof}
The following proposition makes is very important and demonstrates the beautiful interplay between $ \limsup $ (and $ \liminf $) of numbers and sets.

\begin{proposition}
	Let $ (E,\mathcal{A}) $ be a measurable space, and let $ f: E \to \R $. Let $ a \in \R $. Then
	\[ \set{\limsup f_n \leq a} = \limsup \set{f_n \leq a}, \]
	and 
	\[ \set{\liminf f_n \leq a} = \liminf \set{f_n \leq a}. \]
	Note the remark below for the notation convention used above.
\end{proposition}
\begin{proof}
	Let $ x \in \set{\limsup f_n \leq a} $. So $ \limsup f_n(x) \leq a $. From definition $ \limsup f_n(x) = \inf_{n}\sup_{k\leq n} f_k(x) $. Having $ \inf_n \sup_{k\leq n} f_n(x) \leq a $ implies that $ \exists n \in N $ we have $ \sup_{k\geq n} f_n(x) \leq a $. The second statement in turn implies that $ \forall k \geq n  $ we have $ \sup_{k\geq n} f_n(x) \leq a $. In summary $ \limsup f_n(x)\leq a $ implies $ \exists N \in \N $ such that $ \forall n\geq N $ we have $ f_n(x) \leq a $. This is precisely
	\[ \set{\limsup_n f_n} = \bigcup_n \bigcap_{k\geq n} \set{f_n \leq a}. \] 
	For the second statement, let $ x \in \set{\liminf f_n \leq a} $. Then $ \liminf_n f_n(x)\leq a $. By definition $ \sup_{n}\inf_{k\geq n} f_n(x) \leq a $. So $ \forall n \in \N $ we have $ \inf_{k\geq n} f_n(x)\leq a $, which further implies that $ \exists k \geq n $ we have $ f_n(x) \leq a $. This is precisely
	\[ \set{\liminf f_n \leq a} = \bigcap_n \bigcup_{k\geq n} \set{f_n \leq a}.  \] 
\end{proof}


\begin{remark}
	In the proposition above use the following notation which is frequently used in probability theory context.
	\[ \set{f \leq a} = \set{x \in E: f(x)\leq a}. \]
\end{remark}


The Proposition and Lemma above result in the following very useful Corollary. Note that the co-domain of the measurable functions below is $ \R $.

\begin{corollary}
	Let $ \set{f_n} $ be a sequence of measurable function from the measurable space $ (E,\mathcal{A}) $ to $ \R $. Then
	\[ \inf_n f_n,\quad \sup_n f_n,\quad \liminf_n f_n,\quad \limsup_n f_n, \]
	are all measurable functions.
\end{corollary}
\begin{proof}
	We need to show that $ \set{f\leq a} \in \mathcal{A} $ for all $ a\in \R $. From proposition above 
	\[ \set{\limsup f_n \leq a} = \limsup \set{f_n \leq a}, \]
	and 
	\[ \set{\liminf f_n \leq a} = \liminf \set{f_n \leq a}. \]
	So it follows that $ \limsup_n f_n $ and $ \liminf_n f_n $ are measurable functions. Furthermore, be the Lemma above
	\[ \set{\inf_n f_n \leq a} = \bigcup_n \set{f_n \leq a}, \]
	and
	\[ \set{\sup_n f_n \leq a} = \bigcap_n \set{f_n \leq a}. \]
	Then it follows immediately that $ \sup_n f_n $ and $ \inf_n f_n $ are all measurable functions.
\end{proof}

We can use the tools developed above to show that the set of all points where a fiven function 

\section{Solved Problems}

\begin{problem}[Properties of measure, page 6]
	Let $ (E, \mathcal{A}, \mu) $ be a measure space. Prove the following properties of the measure function $ \mu $.
	\begin{enumerate}[(a)]
		\item If $ A\subset B $ then $ \mu(A) \leq \mu(B) $. If in addition $ \mu(A) < \infty $, then
		\[ \mu(B\backslash A) = \mu(B) - \mu(A). \]
		\item \textbf{Principle of inclusion-exclusion}: Let $ A.B \in \mathcal{A} $, then
		\[ \mu(A) + \mu(B) = \mu(A\cup B) + \mu (A\cap B). \]
		\item \textbf{Continuity from below}: Let $ \set{A_n} $ be a collection of sets that $ A_n \uparrow A $, meaning $ A_1\subset A_2 \subset\cdots $ and we have $ A = \bigcup_n A_n $. Then
		\[ \mu(A_n) \to \mu(A) \quad \text{as $ n\to\infty $}. \]
		In other words
		\[ \mu(\bigcup_n A_n) = \lim_n \mu(A_n). \]
		\item \textbf{Continuity from above}: Let $ \set{A_n} $ be a collection of sets that $ A_n \downarrow A $, meaning $ A_1 \supset A_2 \supset \cdots $ and we have $ A = \bigcap_n A_n $. Then
		\[ \mu(A_n) \to \mu(A) \quad \text{as $ n\to\infty $}. \]
		\item \textbf{Sub-additivity}: Let $ \set{A_n} $ be a collection of sets in $ \mathcal{A} $. Then
		\[ \mu(\bigcup_n A_n) \leq \sum_n \mu(A_n). \]
	\end{enumerate}
\end{problem}
\begin{solution}
	\begin{enumerate}[(a)]
		\item 
		First, observe that we can write
		\[ B = A \cup (B\cap A^c). \]
		Since the union above is disjoint, then by $ \sigma $-additivity we have
		\[ \mu(B) = \mu(A) + \mu(B\cal A^c). \]
		Since measure is always positive, then it follows that
		\[ \mu(B) \geq \mu(A). \]
		In particular, observing that $ B\cap A^c = B\backslash A $ and using the fact that $ \mu(A)<\infty $, then
		\[ \mu(B\backslash A) = \mu(B) - \mu(A). \]
		\item We can write
		\[ A \cup B = A \cup (B \backslash (A\cap B)). \]
		Since the union above is disjoint, we have
		\[ \mu(A\cup B) = \mu(A) + \mu(B\backslash (A\cap B)) = \mu(A) + \mu(B) - \mu(A\cap B). \]
		\item Let $ B_1 = A_1 $ and define $ B_2 = A_2\backslash A_1, B_3 = A_3 \backslash A_2 $, and etc. Notice that 
		\[ A = \dot\bigcup_n B_n. \]
		By sigma additivity of $ \mu $ we have
		\[ \mu(A) = \sum_n \mu(B_n) \]
		Observe that $ \mu(A_N) = \sum_{n=1}^N \mu(B_n) $. Thus
		\[ \mu(A) = \lim_{n\to\infty} \mu(A_n). \]
		\item This follows easily by using the fact that $ A_n^c \uparrow A^c $ and $ A^c = \bigcup A_n^c $. Thus by continuity from below
		\[ \mu(A^c) = \lim_{n\to\infty}\mu(A^c_n). \]
		We can write
		\[ 1 - \mu(A) = \lim_{n\to\infty}(1 - \mu(A_n)) = 1 - \lim_{n\to\infty}\mu(A_n), \]
		where we have used the fact that $ \lim_{n\to\infty}\mu(A_n) $ exists. Because by assumption $ A_1 \supset A_2 \supset A_3 \cdots $ and by the monotonicity of the measure $ \mu(A_n) $ is a positive decreasing sequence, thus bounded from below. So it converges. 
		
		\item To see this define
		\[ B_n = A_n \backslash (\bigcup_{k=1}^{n-1} A_k). \]
		Notice that $ \bigcup_n A_n = \bigcup_n B_n $ and $ B_n \subseteq A_n $. Also note that $ B_n $s are disjoint. So
		\[ \mu(\bigcup_n A_n) = \mu(\bigcup_n B_n) = \sum_n \mu(B_n) \leq \sum_n \mu_(A_n). \]
	\end{enumerate}
\end{solution}


\begin{remark}
	Probability style proof of item (2) above: Consider
	\[ 1 - \prod_{i}(1 - \mathds{1}_{A_i}) = \sum_i \mathds{1}_{A_i} - \sum_{i<j}\mathds{1}_{A_i}\mathds{1}_{A_j} + \sum_{i<j<k}\mathds{1}_{A_i}\mathds{1}_{A_j}\mathds{1}_{A_k}+\cdots  \]
	Noting the fact that $ 1 - \mathds{1}_{A} = \mathds{1}_{A^c} $, $ \mathds{1}_{A_i}\mathds{1}_{A_j} = \mathds{1}_{A_i\cap A_j} $, $ \E{\mathds{1}_{A}} = \prob(A) $, and applying the expected value to both sides we will get
	\[ \E{1 - \prod_i (1-\mathds{1}_{A_i})} = \sum_i \mu(A_i) - \sum_{i<j}\mu(A_i\cap A_j) + \sum_{i<j<k}\mu(A_i\cap A_j \cap A_k) - \cdots \pm \mu(A_1\cap\cdots\cap A_n). \]
	Notice that the left hand side is
	\[ \E{1 - \prod_i (1-\mathds{1}_{A_i})} = \E{1 - \prod_i \mathds{1}_{A_i^c}} = \E{1 - \mathds{1}_{\cap_i A_i^c}} = \E{\mathds{1}_{\cup A_i}} = \mu(A_1\cup\cdots\cup A_n). \]
\end{remark}


\begin{problem}
	Check that the $\sigma\text{-algebra}$ $ \mathcal{B}(\R) $ is also generated by the class of all intervals $ (a,b) $ with $ a,b\in \R $ and $ a<b $. Or, by the class of all $ (-\infty, a), a\in \R $, or even by the intervals $ (-\infty, a), a\in \Q $ (one cal also replace open intervals by closed intervals).
\end{problem}
\begin{solution}
	Let 
	\[ \mathcal{I}_1 = \set{(a,b): a\in \R, a<b}. \]
	We want to prove that $ \sigma(\mathcal{I}_1) = \sigma(\mathcal{B}) $, where $ \sigma(\mathcal{B}) $ is the $\sigma\text{-algebra}$ generated by all open sets $ \mathcal{T} $. First, observe that the set of all open intervals, i.e. the elements of $ \mathcal{I}_1 $ is a base for the usual topology of $ \R $ which we denoted by $ \mathcal{T} $. So any $ U \in \mathcal{T} $ can be written as a countable union of elements in $ \mathcal{I}_1 $. In other words, 
	\[ U = \bigcup_n I_n \qquad \text{for some $I_n \in \mathcal{I}_1,\ n=1,2,3,\cdots$}. \]
	Since all $ I_n $ belong to $ \sigma(\mathcal{I}_1) $, so does $ U $, i.e. $ U \in \sigma(\mathcal{I}_1) $. So $ \sigma(\mathcal{I}_1) $ is a $\sigma\text{-algebra}$ that contains open sets. On the other hand $ \sigma(\mathcal{B}) $ is the smallest $\sigma\text{-algebra}$ that contains open sets. This implies $ \sigma(\mathcal{B})\ \subset \sigma(\mathcal{I}_1) $. Note that the reverse inclusion, i.e. $ \sigma(\mathcal{I}_1)\subset\sigma(\mathcal{B}) $ is immediate as all open intervals are open sets and thus by definition $ \sigma(\mathcal{B}) $ contains all open intervals, hence containing the smallest $\sigma\text{-algebra}$ that contains all open intervals.
	
	Let
	\[ \mathcal{I}_2 = \set{(-\infty, a),\ a\in \R}. \]
	Note that $ (-\infty, a) $ for some $ a\in \R $ is an open set. So $ \sigma(\mathcal{I}_2) \subset \sigma(\mathcal{B}) $. For the converse, we again use some reasoning similar to above. I.e. we will show that $ \sigma(\mathcal{I}_2) $ contains all open intervals $ (a,b), a,b\in \R, a<b $, hence containing $ \sigma(\mathcal{I}_1) $, which by (a), contain $ \sigma(\mathcal{B}) $. Let $ (a,b), a,b\in \R, a<b $ be any interval. Then we can write
	\[ (a,b) = (-\infty, b) \cap (\bigcap_n [a+1/n, +\infty)). \]
	Note that $ (-\infty,b) \in \sigma(\mathcal{I}_2) $ as well as $ [a+1/n, +\infty) = (-\infty, a+1/n)^c \in \sigma(\mathcal{I}_2) $. So $ (a,b) \in \sigma(\mathcal{I}_2) $. So $ \sigma(\mathcal{I}_2) $ contains all open intervals, thus contain $ \sigma(\mathcal{I}_1) $ and by (a) contain $ \sigma(\mathcal{B}) $.
	
	Let 
	\[ \mathcal{I}_3 = \set{(-\infty, q),\ q\in \Q}. \]
	Note that any interval $ (-\infty, a), a\in \R $ can be written as a
	\[ (-\infty, a) = \bigcup (-\infty, q_n) \]
	where $ q_n $ is a sequence such that $ q_n\neq q $ and $ q_n $ goes to $ q $ from left as $ n\to\infty $. So $ \sigma(\mathcal{I}_3) $ contains $ \sigma(\mathcal{I}_2) $, which by (b) contain $ \sigma(\mathcal{B}) $.
\end{solution}


\begin{remark}
	The following Lemma is the key idea used in the proof.
	\begin{lemma}
		Let $ A $ be a set and $ B_1, B_2 \subset 2^A $. Then $ \sigma(B_1) $ contain $ \sigma(B_2) $ if $ \sigma(B_1) $ contains $ B_2 $.
	\end{lemma}
	\begin{proof}
		By definition $ \sigma(B_2) $ is the smallest $\sigma\text{-algebra}$ that contains the collection $ B_2 $. Knowing $ \sigma(B_1) $ also contains the collection $ B_2 $ then it implies that $ \sigma(B_2) \subset \sigma(B_1) $.
	\end{proof}
\end{remark}

\begin{problem}
	For every integer $ n\geq 1 $, let $ \mathcal{F}_n $ be the $\sigma\text{-algebra}$ on $ \N $ defined by $ \mathcal{F}_n := \sigma(I_n) $ where $ I_n = \set{\set{1},\set{2},\cdots,\set{n}} $. Show that $ (\mathcal{F}_n)_{n\geq 1} $ in an increasing sequence of $\sigma\text{-algebra}$ of $ \N $, but $ \bigcup_n \mathcal{F}_n $ is not a $\sigma\text{-algebra}$.
\end{problem}
\begin{solution}
	The sequence $ (\mathcal{F}_n)_n $ is increasing because $ I_{n} \subset I_{n+1} $. This implies that $ \sigma(I_n) \subset \sigma(I_{n+1}) $. To see that $ \bigcup_n \mathcal{F}_n $ is not a $\sigma\text{-algebra}$, first observe that for all $ m\in \N $ we have 
	\[ \set{2,4,6,\cdots,2m} \in \bigcup_n \mathcal{F}_n. \]
	However, 
	\[ \bigcup_{m\in\N} \set{2,4,6,\cdots,2m} = E, \]
	where $ E $ is the set of all even numbers, but $ E \notin \bigcup_n \mathcal{F}_n $.
\end{solution}

\begin{problem}
	Let $ E $ be an arbitrary set, and let $ \mathcal{A} $ be a $\sigma\text{-algebra}$ on $ E $. Show that $ \mathcal{A} $ cannot be countably infinite. \emph{Hint: By contradiction, suppose that $ \mathcal{A} $ is countably infinite and introduce the atoms of $ \mathcal{A} $ defined for every $ x\in E $ by}
	\[ E_x := \bigcap_{A \in \mathcal{A}, x\in A} A. \]
	\emph{Then show that $ E_x = E_y $ defines an equivalence relation on $ E $, and thus the equivalence classes for this relation form a partition of $ E $ be measurable sets}.
\end{problem}
\begin{solution}
	\emph{Note: I wrote the remark below before thinking about a solution for this problem. That is why I thought I will have a different reasoning than the one described below}. To see the proof, first observe that $ E_x = E_y $ is indeed an equivalent relation. Because $ \forall x\in E $ we have $ E_x = E_x $. Furthermore if for some $ x,y \in E $ we have $ E_x = E_y $ then it follows that $ E_y = E_x $. Finally if for some $ x,y,z \in E $ we have $ E_x = E_y $ and $ E_y = E_z $, then it immediately follows that $ E_x = E_z $. Let $ \set{P_n} $ be the collection of all partitions of the space $ E $. These are precisely the ``ATOMS'' discussed in the remark below. Each element of $ \mathcal{A} $ is a union of some atoms. Hence there are $ 2^\abs{\mathcal{N}} $ combinations. So $ \mathcal{A} $ has the size of continuum $ \mathfrak{c} $.
\end{solution}


\begin{remark}
	In Problem 2.7.9 Rosenthal we proved that if a $\sigma\text{-algebra}$ is finite, then its size should be of the form $ 2^m $ for some $ m\in \N $. For prove we first consider the set of all ATOMS i.e. the collection of sets that the $\sigma\text{-algebra}$ do not contain any non-empty subset of it. Then we observe that any element in the $\sigma\text{-algebra}$ can be written as the union of elements of the set of all atoms, hence $ 2^m $ combinations (where $ m $ is the number of atoms). There is a very beautiful connection between this and the problem above. In the problem above we prove that a $\sigma\text{-algebra}$ can not be countably infinite. It is clear from the fact discussed above. As long as the number of atoms is finite (like $ m $), then the cardinality of the $\sigma\text{-algebra}$ is $ 2^m $. When the number of atoms become countable, then the cardinality of $\sigma\text{-algebra}$ becomes $ 2^\N $ which is the size of continuum $ \mathfrak{c} $. In other words, there are no sets of cardinality $ \log_2(\abs{\N}) $.
\end{remark}


\begin{problem}
	Let $ (E,\mathcal{A},\mu) $ be a measure space, with $ \mu(E) > 0 $, and let  $ f: E \to \R $ be a measurable function. Show that, for every $ \epsilon>0 $, there exists a measurable set $ A \in \mathcal{A} $ such that $ \mu(A) > 0 $ and $ \abs{f(x) - f(y) < \epsilon} $ for every $ x,y \in A $.
\end{problem}
\begin{solution}
	First, we partition $ \R $ as follows. Let $ \epsilon>0 $ be given. Define
	\[ I_k = [\epsilon k, \epsilon(k+1)). \]
	Observe that the sequence $ \set{I_k}_{k\in \Z} $ is a partitioning of $ \R $. Define
	\[ E_k = \inv{f}(I_k). \]
	We know that $ E = \bigcup_k E_k $. From the sub-additivity of measure we have 
	\[ 0 < \mu(E) \leq \sum_k \mu(E_k). \]
	This implies that $ \mu(E_k) > 0 $ for some $ k \in \Z $. Observe that by construction $ \abs{f(x) - f(y)} < \epsilon $ for all $ x,y \in E_k $. This completes the proof.
\end{solution}




