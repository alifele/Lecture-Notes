\chapter{Second Order Linear Differential Equations}


We start with the definition of second order linear differential equation.

\begin{definition}[General form of second order linear differential equation]
	The second order differential equation for function $ y:(\alpha,\beta)\to\R $ is of the form
	 \[y'' + p(x)y' + q(x)y = f,\] 
	for some known functions $ p, q, f $. If we consider the right-hand side be identically zero function, 
	\[ y'' + p(x)y' + q(x)y = 0, \]
	is called a homogeneous differential equation.
\end{definition}

If we impose additional initiation conditions for the differential equation, then we will have an initial value problem.

\begin{definition}[Initial value problem]
	The differential equation
	\[y'' + p(x)y' + q(x)y = f, \qquad x \in (\alpha, \beta)\] 
	where $ p,q,f $ are known functions of $ x $, along with the initial conditions
	\[ y(t_0) = y_0, \qquad y'(t_0) = y'_0, \]
	where $ t_0 \in (\alpha,\beta) $, us called an initial value problem.
\end{definition}
\newpage

\begin{theorem}[Wronskian Condition]
	Let $ y_1 $ and $ y_2 $ be two solutions of the homogeneous differential equation 
	\[ y'' + py' + qy = 0. \]
	The 
	\[ y = c_1 y_1 + c_2 y_2, \]
	is the general solution of the differential equation (i.e. every solution can be written in this form) if we have
	\[ (y_1 y_2' - y_1'y_2)(t) \neq 0 \qquad  \forall t\in (\alpha, \beta).  \]
\end{theorem}

\begin{proof}
	Let $ y_1 $ and $ y_2 $ be two solutions of the differential equation and let $ \phi = c_1y_1 + c_2y_2 $ be the general solution. Let $ y $ be any solution. Then at some time $ t_0 \in (\alpha, \beta) $, it has the values $ y(t_0) = y_0 $ and $ y'(0) = y_0' $. Then for the general solution we need to have
	\[ \phi(t_0) = c_1y_1(t_0) + c_2y_2(t_0) = y_0, \qquad \phi'(t_0) = c_1y'_1(t_0)+ c_2y'_2(t_0) = y'_0  \]
	which forms the following system of equations
	\[ \matt{y_1(t_0)}{y_2(t_0)}{y_1'(t_0)}{y_2'(t_0)}\vectt{c_1}{c_2} = \vectt{y_0}{y'_0}. \]
	The only way that we can find the coefficients $ c_1 $ and $ c_2 $ (that can depend on time $ t_0 $) for every $ t_0 \in (\alpha, \beta) $ is when the matrix has non-zero determinant. This translates to
	\[ y_1(t_0)y_2'(t_0) - y_1'(t_0)y_2(t_0) \neq 0 \qquad \forall t_0 \in (\alpha,\beta) \]
\end{proof}

The theorem above is basically saying that if we can find two solutions for the linear second order ODE that has non-zero Wronskian, then every other solution of the differential equation will be a linear combination of these two solutions. This, in some sense, is the same as the linear independence.

\begin{definition}[Wronskian of two functions]
	The Wronskian of two function $ y_1 $ and $ y_2 $ defined as a function 
	\[ W[y_1,y_2] = y_1y_2' - y'_1y_2. \]
\end{definition}

The following Lemma shows the relation between the original linear second order homogeneous differential equation and the Wronskian.

\begin{lemma}
	Let $ y_1 $ and $ y_2 $ be two solutions of 
	\[ y'' + py' + qy = 0. \]
	Then their Wronskian $ W(t) := W[y_1,y_2](t) $ satisfies the following ODE.
	\[ W' + p(t)W = 0. \]
\end{lemma}
\begin{proof}
	The Wronskian is defined to be
	\[ W = W[y_1,y_2]= y_1y_2' - y_1'y_2. \]
	Then calculating the first derivative will yield
	\[ W' = y_1y_2'' - y_1''y_2. \]
	Since $ y_1 $ and $ y_2 $ satisfies the homogeneous ODE, then $ y_1'' = -py_1' - y_1 $ and $ y_2'' = -py_2' - y_2 $. Substituting in the equation above, we will get
	\[ W' = -p(y_1y_2' - y_1'y_2) = -pW.  \]
	Thus we have shown that the Wronskian of two solutions satisfies $ W' + pW = 0 $.
\end{proof}

Now, based on the Lemma above, we can state and prove the following important proposition.
\begin{proposition}[The Wronskian of two solutions is either always zero or non-zero]
	Let $ y_1 $ and $ y_2 $ be two solutions of 
	\[ y'' + py' + qy = 0, \qquad t\in (\alpha,\beta)\]
	Then the Wronskian of solutions is either identically zero in the interval $ (\alpha,\beta) $ or non-zero for all $ t \in (\alpha, \beta) $.
\end{proposition}
\begin{proof}
	From the Lemma above, we know that the Wronskian $ W(t) = W[y_1,y_2](t) $ satisfies 
	\[ W' + p W = 0, \qquad t\in (\alpha,\beta). \]
	Let $ t_0 \in (\alpha,\beta) $. First, observe that the zero function, i.e. $ O(t) \equiv 0 $ for all $ t\in(\alpha,\beta) $ is also a solution to the differential equation. If $ W(t_0)= 0 $, then by the existence-uniqueness theorem for the ODE that $ W $ satisfies, we conclude that $ W $ should also be identically zero in $ (\alpha,\beta) $. Let $ W(t) $ be non-zero for all $ t\in(\alpha,\beta) $. Then we can write
	\begin{align*}
		&W' + pW = 0. \\
		&W'/W = -p.\\
		&\frac{d}{dt} (\ln(W)) = -p.\\
		&\int_{t_0}^{t} (\ln(W(\tau)))'\ d\tau = -\int_{t_0}^{t} p(\tau)\ d\tau.\\
		& \ln(W(t)) - \ln(W(t_0)) = - \int_{t_0}^{t}p(\tau)\ d\tau. \\
		& W(t) = W(t_0)\exp(-\int_{t_0}^{t}p(\tau)\ d\tau).
	\end{align*}
	Observe that $ \exp(-\int_{t_0}^{t}p(\tau)\ d\tau) $ is never zero. And since $ W(t_0) $ is also non-zero $ \forall t_0 \in (\alpha,\beta) $, then we can infer that $ W(t) $ is either identically zero in $ (\alpha,\beta) $ or always non-zero.
\end{proof}


As we will see in the following theorem, when the Wronskian of two solutions is zero, then we can infer some useful information about those solutions.

\begin{proposition}
	Let $ y_1 $ and $ y_2 $ be two solutions of the following homogeneous linear second order ODE.
	\[ y'' + py' + qy = 0, \qquad t \in (\alpha,\beta). \]
	If $ W[y_1,y_2](t_0) = 0 $ for some $ t_0 \in (\alpha,\beta) $. Then one of the solutions is a constant multiple of the other. 
\end{proposition}
