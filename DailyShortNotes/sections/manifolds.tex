\chapter{October}

\section{Manifolds}

For a quick review on the manifolds (on 8 October), I am reading through the Lee's text book.

\begin{proposition}[Some properties of the Hausdorff spaces]
	Let $ (X,\mathcal{T}) $ be a Hausdorff topological spaces. I.e. for any distinct points $ x,y \in X $ there exists $ U,V \in T $ such that $ x \in U, y \in V $ and $ U\cap V = \emptyset $. Then we have:
	\begin{enumerate}[(a)]
		\item Any convergent sequence in $ X $ has a unique limit point.
		\item $ (X,\mathcal{T}) $ is Hausdorff if and only if the diagonal $ \Delta = \set{(x,x): x \in X} $ is closed under the product topology.
		\item Every compact set in $ X $ is closed.
	\end{enumerate}
	
\end{proposition}
\begin{proof}
	\begin{enumerate}[(a)]
		\item Assume otherwise. I.e. the sequence $ \set{x_n} $ converges to two different distinct points $ p,q \in X $. Since $ X $ is Hausdorff, then there exists disjoint $ U,V \in \mathcal{T} $ such that $ U\cap V = \emptyset $ and $ x \in U, y\in V $. Since $ x_n \to p $ and $ x_n \to q $ as $ n\to \infty $ then there exists $ N_1,N_2 \in \N $ such that $ \forall n > N_1 $ we have $ x_n \in U $ and $ \forall n > N_2 $ we have $ x_n \in V $. This is a contradiction as we assume that $ U\cap V = \emptyset $.
		
		\item Forward direction $ \boxed{\implies} $ : We assume that $ X $ is Hausdorff. Let $ (x,y) \in (X\times X) \backslash \Delta $. Since $ X $  is Hausdorff, then there exists $ U,V \in \mathcal{T} $ such that $ x \in U $ and $ y \in V $ and $ U\cap V = \emptyset $. Thus the set $ U\times V $ is also open in $ X\times X $  with the subspace topology and does not contain $ (x,x) $ nor $ (y,y) $. Thus for any point that is not on the diagonal we can find a open set in $ X\times X $ that is disjoint with the diagonal. This proves that $ \Delta^c $ is open thus $ \Delta  $ is closed.
		
		\noindent Backward direction $ \boxed{\Longleftarrow} $ : We assume that $ \Delta  $ is closed in $ X\times X $. Then for any $ (x,y) \in (X\times X) \backslash \Delta  $ we can find and open set $ M $ such that $ M\cap \Delta = \emptyset $. We can find an open set $ U\times V \in M $ such that $ x \in U $ and $ y \in V $. Since $ M \cap \Delta = \emptyset $ then $ U \cap V = \emptyset $. Since the point $ (x,y) $ was arbitrary, then $ X $ is Hausdorff.
		
		\item Let $ K $ be a compact set in $ X $ and let $ x \in K^c $. Since the space is Hausdorff then for any $ q \in K $ we can find $ U_q, V_q $ such that $ x \in U_q $ and $ q \in V_q $. The collection $ \set{V_q}_{q\in K} $ covers $ K $. Since $ K $ is compact, then we have a finite sub-cover, call $ \set{V_{q_1},\cdots,V_{q_n}} $. Let $ U = U_{q_1} \cap \cdots\cap U_{q_n} $. $ U $ is open (finite intersection of open sets) and is disjoint with all of the sets in the finite sub-cover (i.e. $ U \cap V_{p_i} = \emptyset $ for $ i=1,\cdots,n $). Thus $ U \cap K^c = \emptyset $. This implies that $K$ is closed.
	\end{enumerate}
	
\end{proof}


The history behind the following proposition is funny! When I first had analysis course, then had this view that the uniqueness of the limit of a convergent sequence comes from the triangle inequality. Then today, in proposition above, I saw that it comes form the space being Hausdorff. The following proposition and its proof makes my intuition about the limits of the convergent sequences and triangle inequality more clear. 

\begin{proposition}
	Every metric space $ (X,d) $ is a Hausdorff space.
\end{proposition}
\begin{proof}
	Let $ x,y \in X $ distinct. Then $ d(x,y) > 0  $. let $ d = d(x,y)/2 $ and define the open balls
	\[ U = B_{d/2}(x), \qquad V = B_{d/2}(y). \]
	We claim $ U\cap V = \emptyset $. To see the proof, assume otherwise. Then there exists $ z \in U \cap V $. By the triangle inequality we have
	\[ d = d(x,y) \leq d(x,z) + d(z,y) < d/2 + d/2 = d, \]
	which is a contradiction.
\end{proof}

\begin{proposition}[Properties of first countable space]
	Let $ (X,\mathcal{T}) $ be first countable. I.e. for all $ x \in X $ there is a neighborhood basis at $ x $. Let $ A \subset X $ be any subset, and $ x \in X $.
	\begin{enumerate}[(a)]
		\item $ x \in A^\circ $ if and only if every sequence in $ X $ converging to $ x $ is eventually in $ A $.
		\item $ x \in \closure{A} $ if and only if $ x $ is a limit of a sequence of points in $ A $.
		\item Every metric space is first countable.
	\end{enumerate}
\end{proposition}
\begin{proof}
	\begin{enumerate}[(a)]
		\item First, we prove the forward direction $ \boxed{\Longrightarrow} $. We prove this by contrapositive. Assume that there exists a sequence $ \set{a_n} $ in $ X $ converging to $ x $ such that for all $ N\in \N $ we can find $ n > N $ such that $ a_n \notin  A $. Since $ a_n \to x $ as $ n\to\infty $ then for all $ U \in \mathcal{N}(x) $ we have $ a_n \in U $ for all $ n $ large enough. However, we just observed that for $ n $ large enough we can find some $ a_n \notin A $. Thus for all neighborhood $ U $ of $ x $ there exists some $ a_n $ that $ a_n \notin A $, i.e. $ U \cap A^c \neq \emptyset $ for all $ U \in \mathcal{N}(x) $. Thus $ x \notin A^\circ $.
		
		\noindent For the converse direction $ \boxed{\Longleftarrow} $, again we use the prove by contrapositive. Let $ x \notin A^\circ $. Then for all neighborhoods $ U \in \mathcal{N}(x) $ we have $ A^c \cap U \neq \emptyset $. Since the space is first-countable, in particular for all $ \set{V_i}_{i\in\N}  \subset \mathcal{N}(x) $ we have $ A^c \cap V_i \neq \emptyset $. For each such $ V_i $ we choose $ a_i \in A^c\cap V_i $. Then the sequence $ \set{a_n} $ is converging to $ x $ that is not eventually in $ A $.
		
		\item First, we prove the forward direction $ \boxed{\Longrightarrow} $. Let $ x \in \closure{A} $. Then for all $ U \in \mathcal{N}(x) $ we have $ U \cap A \neq \emptyset $. In particular for the local base $ \set{V_i}_{i\in\N} =  B \subset \mathcal{N}(x) $ we have $ V_i \cap A \neq \emptyset $ for all $ i \in \N $. Let $ a_n \in V_n \cap A $. Then by definition $ a_n $ converges to $ x $ and this completes the proof.
		
		\noindent Then we prove the backward direction $ \boxed{\Longleftarrow} $. Let $ \set{a_n} $ be a sequence of points in $ A $ such that $ a_n \to x $ as $ n \to \infty $. Then by definition $ \forall U \in \mathcal{N}(x) $ we have $ a_n \in U $ for all $ n $ large enough. This implies that $ U \cap A \neq \emptyset $ for all $ U \in \mathcal{N}(x) $. This implies that $ x \in \closure{A} $.
		
		\item We provide a constructive prove. Let $ (X,d) $ be a metric space with $ x \in X $ be some point. Then a local countable basis for $ x $ is given by the collection of all open balls with rational radius and centered at $ x $.
	\end{enumerate}
\end{proof}


\begin{proposition}[Properties of second-countable spaces]
	\begin{enumerate}[(a)]
		\item Every second-countable topological space is separable (i.e. admits a countable dense subset).
		\item Every second-countable space is first-countable.
	\end{enumerate}
\end{proposition}
\begin{proof}
	\begin{enumerate}[(a)]
		\item We demonstrate a constructive proof. Since the space $ X $ is second-countable, then there exists a countable basis for the topology, call it $ \set{U_n}_{n\in\N} $. From each $ U_n $ get $ x_n $ (assuming that $ U_n $ is non-empty, otherwise discard it) and call the collection of all such points $ D $. Let $ V $ be any open set. Then there exists $ U_n \subseteq V $ and due to construction $ x_n \in V $ thus $ D \cap V \neq 0 $. Since $ V $ was arbitrary, then $ D $ is dense.
		
		\item For a point $ x \in X $ let $ B(x) $ (the local base) be the set of all open sets in the countable topology base such that contains the point $ x $. 
	\end{enumerate}
\end{proof}


\begin{remark}
	Note that all metric spaces are first-countable, however, only separable metric spaces are second-countable.
\end{remark}

Reviewing the nice properties of the second-countable spaces and the Hausdorff spaces reveals the rationale behind our choice that why a topological manifold should be second-countable and Hausdorff. In fact, we want the space be well-behaved when with respect to the sequences (i.e. unique limit of converging sequences, some useful sequencial characterizations, etc).


\begin{problem}[Exercise in Lee]
	Show that equivalent definitions of topological manifolds are obtained if instead of allowing $ U $ to be homeomorphic to any open subset of $ \R^n $, we require it to be homeomorphic to an open ball in $ \R^n $ or to $ \R^n $ itself.
\end{problem}
\begin{proof}
	We will have different parts for this proof.
	\begin{itemize}
		\item We want to show that the classic definition is equivalent to $ U $ being homeomorphic to an open ball in $ \R^n $. For the forward direction ($ \boxed{\Longrightarrow} $), we assume that $ U $ is homeomorphic to an open subset of $ \R^n $, call it $ A $. Let $ B \subset A $ be an open ball. Then under the homeomorphism map the pre-image of $ B $ is also open and is contained in $ U $. Thus there is an open neighborhood of the point on manifold that is homeomorphic to an open ball in $ \R^n $. The converse direction ($ \boxed{\Longleftarrow} $) follows immediately since an open ball in $ \R^n $ is itself open subset of $ \R^n $.
		
		\item We want to show that the classic definition is equivalent to $ U $ being homeomorphic to $ \R^n $. Using our result above, and noting the fact that an open ball in $ \R^n $ is homeomorphic to the whole space, then the proof is immediate. Note the remark below.
	\end{itemize}
\end{proof}
\begin{remark}
	Given an open ball in $ \R^n $ we can construct the homeomorphism explicitly using the tangent function. A $ \R^1 $ example is illustrative. Let $ (a,b) $ be an open ball in $ \R $. This is homeomorphic to sphere $ S^1 $ with the north pole removed (easy to give the explicit homeomorphism by relating the distance from the center of the interval to the angle of the corresponding point on the unit sphere). Then we can use the stereographic projection to project the unit sphere without the pole to the whole real line (using the tangent map).
\end{remark}










