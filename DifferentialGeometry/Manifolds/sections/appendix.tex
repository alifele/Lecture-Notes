\chapter{Appendix}


\section{Change of Basis}
Let $ V $ be a vector space of dimension $ n $, and $ \mathbb{B}_1 = \set{e_i}_{i=1}^{n} $ be a basis. Then for $ \alpha \in V $ we can write it uniquely as 
\[ \alpha  = \sum_i \alpha^i e_i. \]
Once we know a particular basis, we can do the following association
\[ e_i \longleftrightarrow 
\begin{bmatrix}
	0 \\
	\vdots \\
	\underbrace{1}_{i-\text{th element}} \\
	\vdots \\
	0
\end{bmatrix} 
\qquad 
\alpha \longleftrightarrow
\vec{a} = 
\begin{bmatrix}
	\alpha^1 \\
	\alpha^2 \\
	\vdots \\
	\alpha^{n-1} \\
	\alpha^n
\end{bmatrix} 
\]
And we define the transpose operator to be $ \vec{a}^T = (\alpha^1\ \cdots\ \alpha^n) $. Then we can write the unique decomposition of $ \alpha $ to the basis vectors as 
\[ \alpha = (\alpha_1\ \cdots\ \alpha_n) 
\begin{bmatrix}
	e^1\\
	\vdots\\
	e^n
\end{bmatrix}
 \]
where these symbols act with the usual multiplication of vectors. Since these representations of elements of the vector space by column vectors depends on the particular basis that we choose, then then natural question is that how these symbols will change if we change the basis. Assume that the old basis is changed to the new $ \mathbb{B}_2 = \set{e'_i}_{i=1}^n $ where
\[ 
\begin{bmatrix}
	e'_1 \\
	e'_2 \\
	\vdots\\
	e'_n
\end{bmatrix}
= 
\begin{pmatrix}
	r_{11} & r_{12} & \cdots & r_{1N} \\
	r_{21} & r_{22} & \cdots & r_{2N} \\ 
	\vdots & \vdots & \ddots & \vdots \\
	r_{N1} & r_{N2} & \cdots & r_{NN}
\end{pmatrix}^T
\begin{bmatrix}
	e_1 \\
	e_2 \\
	\vdots \\
	e_n
\end{bmatrix} = 
R^T 
\begin{bmatrix}
	e_1 \\
	e_2 \\
	\vdots \\
	e_n
\end{bmatrix}
 \]
To find the coordinates of $ \alpha $ in the new basis we can write
\[ \alpha = \underbrace{(\alpha_1\ \cdots\ \alpha_n) \inv{(R^T)} }_{(\beta_1\ \cdots\ \beta_2)}
\begin{bmatrix}
	e_1 \\
	e_2 \\
	\vdots \\
	e_n
\end{bmatrix}
 \]
Thus the the relation between then new coordinate of $ \alpha $ and the old coordinate is 
\[ \vecttt{\alpha_1}{\vdots}{\alpha_n} = R \vecttt{\beta_1}{\vdots}{\beta_n} \]

To observer how does linear map representation in new coordinates change consider the equation 
\[ \vecttt{x_1}{\vdots}{x_n} = M \vecttt{y_1}{\vdots}{y_n}. \]
where the representation of column vectors and the matrix $ M $ is in the basis $ \mathbb{B}_1 $. If we change it to a new basis by the change of basis matrix $ R $, then for the new coordinates we will have
\[ R\vecttt{x_1'}{\vdots}{x_n'} = M R \vecttt{y_1'}{\vdots}{y_n'}. \] 
This implies
\[ \boxed{\vecttt{x_1'}{\vdots`}{x_n'} = \underbrace{\inv{R}M R }_{M'}\vecttt{y_1'}{\vdots}{y_n'}}. \]

\begin{summary}[Change of basis]
	In a nutshell, Let $ V $ is a vector space of dimension $ N $. Then after specifying a basis $ \mathbb{B}_1 = \set{e_i} $, we can represent the elements of vector space as column vectors. If we want to change the basis to a new basis $ \mathbb{B}_2 = \set{e'_i} $ with the change of basis matrix $ R $ (whose $ j-\text{th} $) column contains the coordinates of $ e'_i $ in terms of $ \mathbb{B}_1 $ basis vectors. Then the relation between new coordinates and the old coordinates is 
	\[ \underbrace{\vecttt{x_1}{\vdots}{v_n}_{\mathbb{B}_1} }_\text{represented according to $ \mathbb{B}_1 $}= R \overbrace{\vecttt{x'_1}{\vdots}{x'_n}_{\mathbb{B}_2}}^{\text{represented according to $ \mathbb{B}_2 $}}. \]
	Also if 
	\[ \vecttt{y_1}{\vdots}{y_n}_{\mathbb{B}_1} = M \vecttt{x_1}{\vdots}{x_n}_{\mathbb{B}_2} \]
	then 
	\[ \vecttt{y'_1}{\vdots}{y'_n}_{\mathbb{B}_2} = \inv{R}M R \vecttt{x'_1}{\vdots}{x'_n}. \]
\end{summary}

