
\section{Solved Problems}
\begin{problem}[The differential of a map]
	\label{prob:DifferentialOfAMapIsADerivation}
	Check that $ F_*(X_p) $ is a derivation at $ F(p) $ and that $ F_*: T_p N \to T_{F(p)}M  $ is a linear map.
\end{problem}
\begin{solution}
	We start with showing that $ F_*(X_p) $ is a derivation at $ F(p) $. By definition we have
	\begin{align*}
		[(F_*(X_p))(f\cdot g)](F(p)) &=  [ X_p((f\cdot g)\circ F) ](p) = [ X_p((f\circ F) \cdot (g\circ F)) ](p)\\
		&= [X_p(f\circ F)](p) \cdot g(F(p)) + f(F(p))\cdot [X_p(g\circ F)](p)\\
		&= [(F_*(X_p))f](F(p)) \cdot g(F(p)) + f(F(p))\cdot [(F_*(X_p))g].
	\end{align*}
	This shows that $ F_*(p) $ is indeed a derivation at $ F(p) $. Furthermore, to prove the linearity, let $ X_p, Y_p \in T_p(N) $. Then
	\[ (F_*(X_p + Y_p))f = (X_p + Y_p)(f\circ F) = X_p(f\circ F) + Y_p(f\circ F) = (F_*(X_p)) f + (F_*(Y_p)) f. \]
\end{solution}


\begin{problem}[Velocity vector versus the calculus derivative (from L. Tu)]
	\label{prob:VelocutyVecotrAndCalculusNotation}
	Let $ c:(a,b) \to \R $ be a curve with target space $ \R $. Verify that $ c'(t) = \dot{c}(t) d/dx|_{c(t)} $.
\end{problem}
\begin{solution}
	Let $ f: \R \to \R $. From definition 
	\[ c'(t_0) f = c_*(\frac{d}{dt}\big|_{t_0})f = \frac{d}{dt}\big|_{t_0}(f\circ c) = \dot{c}(t_0) \frac{df}{dx}, \]
	where we have used the chain rule. Thus
	\[ c'(t_0) = \dot{c}(t) \frac{d}{dx}\big|_{c(t_0)}. \]
\end{solution}

\begin{problem}[Differential of left multiplication (from L. Tu)]
	If $ g $ is a matrix in the general linear group $ GL(n,\R) $, let $ \ell_g: GL(n,\R) \to GL(n,\R) $ be left 
	multiplication by $ g $. Thus $ \ell_g(B) = gB $ for any $ B \in GL(n,\R) $. Since $ GL(n,\R) $ is an open subset of the vector space $ \R^{n\times n} $, the tangent space $ T_g(GL(n,\R)) $ can be identified with $ \R^{n\times n} $. Show that with this identification the differential $ (\ell_g)_{*,I}: T_I(GL(n,\R)) \to T_g(GL(n,\R)) $ is also left multiplication by $ g $. 
\end{problem}
\begin{solution}
	We will use the idea of \autoref{prop:curvesToComputeDifferential} to compute the differential using a curve. Let $ c: (-\epsilon,\epsilon) \to GL(n,\R) $ be a curve such that $ c(0) = I $ and $ c'(0) = X_p $. Then 
	\[ (\ell_g)_{*,I} X_p = (\ell_g\circ c)'(0) = (gc)'(0) = g c'(0) = gX_p. \]
\end{solution}

\begin{problem}[Differential of a map (from L. Tu.)]
	Let $ F: \R^2 \to \R^3 $ be the map
	\[ (u,v,w) = F(x,y) = (x,y,xy). \]
	Let $ p = (x,y) \in \R^2 $. Compute $ F_*(\partial/\partial x|_p) $ as a linear combination of $ \partial/\partial u, \partial/\partial v $, and $ \partial/\partial w $ at $ F(p) $
\end{problem}
\begin{solution}
	We know that $ F_*(\partial/\partial x|_{p} ) \in T_{F(p)}\R^3 $. Letting $ (\R^3,\mathbbm{1}_{\R^3} = (u,v,w)) $ be an atlas for $ \R^3 $, we will have
	\[ F_*(\partial/\partial x|_{p} ) = a \partial/\partial u|_{F(p)} + b\partial/\partial v|_{F(p)} + c\partial/\partial w|_{F(p)}.    \]
	Applying $ u,v,w $ to both sides we will get
	\[ a = F_*(\partial/\partial x|_{p} ) u,\qquad b = F_*(\partial/\partial x|_{p} ) v, \qquad c = F_*(\partial/\partial x|_{p} ) w. \]
	Thus 
	\[ a = 1,\qquad b= 0, \qquad c = 1. \]
	
\end{solution}

\begin{problem}[Differential of a Linear Map]
	Let $ L:\R^n \to \R^m $ be a linear map. For any $ p \in \R^n $, there is a canonical identification of $ T_p(\R^n) $ to $ \R^n $ give by
	\[ \sum a^i \frac{\partial}{\partial  x^i}\big|_{p} \mapsto a = \< a^1,\cdots,a^n \>. \]
	Show that the differential $ L_{*,p}:T_p(\R^n) \to T_{L(p)}\R^m $ is the map $ L:\R^n \to \R^m $ itself with the identification of the tangent spaces as above.
\end{problem}
\begin{solution}
	{\color{red} \noindent TODO: Answer to be added.}
\end{solution}

\begin{problem}[Differential of a Map (from W. Tu)]
	Fix a real number $ \alpha $ and define $ F: \R^2 \to \R^2 $ by 
	\[ 
	\vectt{u}{v} = (u,v) = F(x,y) = \matt{\cos\alpha}{-\sin\alpha}{\sin\alpha}{\cos\alpha}\vectt{x}{y}.
	 \]
	 Let $ X = -y\partial/\partial x + x\partial/\partial y $ be a vector field on $ \R^2 $. If $ p = (x,y) \in \R^2 $ and $ F_*(X_p) = (a\partial/\partial u + b\partial/\partial v)|_{F(p)}  $, find $ a $ and $ b $ in terms of $ x,y $, and $ \alpha $.
\end{problem}
\begin{solution}
	We start with 
	\[ F_*(X_p) = (a\partial/\partial u + b\partial/\partial v)|_{F(p)}.  \]
	Applying $ u $ and $ v $ to both sides we will get
	\[ a = F_*(X_p) u, \qquad b = F_*(X_p) v. \]
	To compute $ a $ we will have
	\[ a = F_*(X_p)u = X_p (u\circ F) = (-y\frac{\partial}{\partial  x}\big|_{p} + x \frac{\partial}{\partial  x}\big|_{p})(x\cos\alpha - y \sin\alpha) = -y\cos\alpha -x\sin\alpha. \]
	Similarly for $ b $ we can write
	\[ b = F_*(X_p)v = X_p (v\circ F) = (-y\frac{\partial}{\partial  x}\big|_{p} + x \frac{\partial}{\partial  x}\big|_{p})(x\sin\alpha + y \cos\alpha) = -y\sin\alpha +x\cos\alpha. \]
\end{solution}

\begin{problem}[Transition matrix for coordinate vectors (from L. Tu)]
	Let $ x,y $ be the standard coordinates on $ \R^2 $, and let $ U $ be the open set
	\[ U = \R^2 - \set{(x,0)| x\geq 0}.\]
	On $ U $ the polar coordinates $ r,\theta $ are uniquely defined by
	\[ x = r\cos\theta, \qquad y = r\sin\theta \]
	where $ r>0 $ and $ 0<\theta<2\pi $.
	Find $ \partial/\partial r $ and $ \partial/\partial \theta $ in terms of $ \partial/\partial x $ and $ \partial/\partial y $.
\end{problem}
\begin{solution}
	We need to find the change of basis matrix. We can either use the idea of \autoref{summary:BasesFor TangentSpace} or do the computations manually. We will do the latter. We expand $ \partial/\partial r $
	\[ \partial/\partial r = a \partial/\partial x + b \partial/\partial y. \]
	Applying both sides to $ x $ and $ y $ respectively we will get
	\[ a = \partial x/\partial r = \cos\theta,  \qquad b = \partial y/\partial r = \sin\theta. \]
	Similarly for $ \partial/\partial \theta $ we can write
	\[ \partial/\partial \theta = c \partial/\partial x + d \partial/\partial y. \]
	Applying both sides to $ x $ and $ y $ respectively we will get
	\[ c = \partial x/\partial \theta = -r\sin\theta, \qquad \partial y/\partial \theta =r\cos\theta. \]
\end{solution}

\begin{problem}[Velocity vector (from L. Tu)]
	Let $ p = (x,y) $ be a point in $ \R^2 $. Then 
	\[ c_p(t) = \matt{\cos 2t}{-\sin 2t}{\sin 2t}{\cos 2t}\vectt{x}{y}, \qquad t \in \R,\]
	is a curve with initial point $ p \in \R^2 $. Compute the velocity vector $ c_p'(0) $.
\end{problem}
\begin{solution}
	We can either use the result of the \autoref{prop:VelocityVectorInLocalCorodinate} or we can do a manual calculation, which we will do the latter. Since $ c_p'(t) \in T_p(\R^2) $, then we can expand it in the bases induced by the local coordinates. 
	\[ c_p'(t) = c_{*,p}(\frac{\partial}{\partial  t }\big|_{0})  = a \frac{\partial}{\partial  x}\big|_{p} + b \frac{\partial}{\partial  y }\big|_{p}.\]
	Now applying $ x $ to both sides we will get
	\[ a = c_{*,p}(\frac{\partial}{\partial  t }\big|_{0})x = \frac{\partial}{\partial  t }\big|_{0} (x\circ c_p(t)) = -2 x\sin 2t  - 2y\cos 2t.  \]
	Similarly, by applying $ y $ to both sides we will get
	\[ b = c_{*,p} (\frac{\partial}{\partial  t }\big|_{0}) y = \frac{\partial}{\partial  t }\big|_{0} (y\circ c_p(t)) = 2x\cos 2t - 2y\sin 2t.\]
\end{solution}

\begin{problem}[Tangent Space to a Product]
	If $ M $ and $ N $ are manifolds, let $ \pi_1: M\times N \to M $ and $ \pi_2: M\times N \to N $ be the two projections. Prove that for $ (p,q) \in M\times N $,
	\[ (\pi_{1*},\pi_{2*}) : T_{(p,q)}(M\times N) \to T_p M \times T_q N \]
	is an isomorphism.
\end{problem}
\begin{solution}
	Consider the map
	\[ (\pi_1,\pi_2) : M\times N \to M\times N. \]
	This is indeed a diffeomorphism, and in fact $ (\pi_1,\pi_2) = \mathbbm{1}_{M\times N} $. This implies that $ (\pi_{1*},\pi_{2*}) $ is an isomorphism of vector spaces.
\end{solution}
\begin{carefull}
	I am not sure if my question above is a correct one. However, I have provided a more direct proof below.
\end{carefull}
\textbf{Second solution:} We will show this by explicitly constructing a basis for $ T_{(p,q)}(M\times N) $ as well as $ T_pM \times T_qN $ and we will show that the map $ (\pi_{1*},\pi_{2*}) $ caries the bases to bases, hence is an isomorphism of vector spaces. 

Let $ (U,\phi=(x^1,\cdots,x^n)) $ be a coordinate chart containing $ p \in M$ and $ (V,\psi=(y^1,\cdots,y^n)) $ be a coordinate chart containing $ q \in N $. Then a coordinate chart for $ M\times N $ that contains $ (p,q) $ would be $ (U\times V, (\phi\circ\pi_1, \psi\circ\pi_2) = (\bar{x}^1,\cdots,\bar{x}^n,\bar{y}^1,\cdots,\bar{y}^m)) $. Observe that 
\[ \pi_{1*}(\frac{\partial}{\partial  \bar{x}^i}\big|_{(p,q)}) = a^1_i \frac{\partial}{\partial  x^1}\big|_{p} + \cdots + a^n_i \frac{\partial}{\partial  x^n}\big|_{p}.  \]
By applying $ x^j $ to both sides we get
\[ a^j_i = \pi_{1*}(\frac{\partial}{\partial  \bar{x}^i}\big|_{(p,q)}) x^j = \frac{\partial}{\partial  \bar{x}^i}\big|_{(p,q)}(x^i\circ \pi_1) = \delta_{ij}. \] 
Thus we can write
\[  \pi_{1*}(\frac{\partial}{\partial  \bar{x}^i}\big|_{(p,q)}) = \frac{\partial}{\partial  x^i}\big|_{p}, \]
and by a similar calculation for $ \pi_{2*} $ we get
\[  \pi_{1*}(\frac{\partial}{\partial  \bar{y}^i}\big|_{(p,q)}) = \frac{\partial}{\partial  y^i}\big|_{p}. \]
On the other hand, observe that a basis for $ T_{(p,q)}(M\times N) $ is 
\[ \set{\frac{\partial}{\partial  \bar{x}^1}\big|_{(p,q)},\cdots,\frac{\partial}{\partial  \bar{x}^n}\big|_{(p,q)},\frac{\partial}{\partial  \bar{y}^1}\big|_{(p,q)},\cdots,\frac{\partial}{\partial  \bar{y}^m}\big|_{(p,q)}} \]
and a basis for $ T_p(M)\times T_q(N) $ is 
\[ \set{(\frac{\partial}{\partial  x^1}\big|_{p},0),\cdots,(0,\frac{\partial}{\partial  x^n}\big|_{p},0),\frac{\partial}{\partial  y^1}\big|_{q}, (0,\frac{\partial}{\partial  y^m}\big|_{q})} \]
We can now see that the map $ (\pi_{1*},\pi_{2*}) $ maps these two bases to each other, thus it is an isomorphism of vector spaces.

\begin{problem}[Differentials of multiplication and inverse (from L. Tu)]
	Let $ G $ be a Lie group with multiplication map $ \mu:G\times G \to G $, inverse map $ \iota: G \to G $, and identity element $ e $. 
	\begin{enumerate}[(a)]
		\item Show that the differential at the identity of the multiplication map $ \mu $ is addition:
		\begin{align*}
			\mu_{*,(e,e)}: T_eG \times T_eG \to T_e G, \qquad \mu_{*,(e,e)}(X_e,Y_e) = X_e + Y_e
		\end{align*}
		\item Show that the differential at the identity of $ \iota $ is the negative
		\[ \iota_{*,e}: T_eG \to T_eG, \qquad \iota_{*,e}(X_e) = -X_e. \]
	\end{enumerate}
\end{problem}
\begin{solution}
	$ \ $\\
	\begin{enumerate}[(a)]
		\item We will use the idea of \autoref{prop:curvesToComputeDifferential} to compute the differential. Let $ c:(-\epsilon,\epsilon) \to G\times G $ for some $ \epsilon>0 $, where $ c(t) = (g_1(t),g(t)) $ such that we have $ x(0) = (e,e) $ and $ c'(0)=(X_e,Y_e)$. Then we can write
		\[ \mu_{*,(e,e)} (X_e,Y_e) = (\mu \circ c)'(0) = (g_1g_2)'(0) = g_1'(0)g_2(0) + g_1(0)g_2'(0) = X_e + Y_e, \]
		where we have used the fact that $ (\cdot)' $ is a derivation.
		\item We again start with the idea of \autoref{prop:curvesToComputeDifferential}. Let $ c:(-\epsilon,\epsilon) \to G $ for some $ \epsilon>0 $ such that $ c(0) = e $ and $ c'(0) = X_e $. Thus we can write
		\[ \iota_{*,e} X_e = (\iota \circ c)'(0). \]
		Now we aim at finding $ (\iota \circ c)'(0) $ by using the identity
		\[ \mu(c(t), (\iota \circ c)(t)) = e, \quad \text{for $ t\in (-\epsilon,\epsilon) $}. \]
		Since the function above (which is from $ \R $ to $ G $) is constant, thus its derivation will be zero. I.e.
		\[ (c \cdot (\iota\circ c))(0) = c'(0)\underbrace{\iota(c(0))}_{e} + \underbrace{c(0)}_e(\iota\circ c)'(0) = 0 \]
		where we have used the fact that $ \iota(e) = e $ and $ (\cdot)' $ is a derivation. This implies that 
		\[ (\iota \circ c)'(0) = -X_p. \]
	\end{enumerate}
\end{solution}

\begin{problem}[Transforming vectors to coordinate vectors]
	Let $ X_1,\cdots,X_n $ be $ n $ vector fields on an open subset $ U $ of a manifold of dimension $ n $. Suppose that at $ p \in U $, the vectors $ (X_1)_p,\cdots,(X_n)_p $ are linearly independent. Show that there is a chart $ (V,x^1,\cdots,x^n) $ about $ p $ such that $ (X_i)_p = (\partial/\partial x^i)_p $ for $ i = 1,\cdots,n $.
\end{problem}
\begin{solution}
	Let $ (V,y^1,\cdots,y^n) $ be a coordinate chart containing $ p $. We expand the tangent vectors $ (X_1)_p,\cdots,(X_n)_p $ in terms of the bases $ \partial/\partial y^1|_{p},\cdots,\partial/\partial y^n|_{p} $ and arrange the coordinates as the columns of a matrix $ P $. Symbolically we can write
	\[ \left[(X_1)_p\ (X_2)_p\ \cdots\ (X_n)_p\right] = \left[\partial/\partial y^1|_{p}\ \partial/\partial y^2|_{p}\ \cdots\ \partial/\partial y^n|_{p}\right] P.
	 \]
	Since $ (X_i)_p $ are linearly independent, so $ P $ is non singular and is invertible. Thus we can write
	\[ \left[ \partial/\partial y^1|_{p}\ \partial/\partial y^2|_{p} \cdots\ \partial/\partial y^n|_{p} \right] = \left[ (X_1)_p\ (X_2)_p\ \cdots\ (X_n)_p \right] \inv{P}. \]
	Define
	\[ x^j = \sum_i\ [\inv{P}]_{ji}\ y^i. \]
	Since $ \inv{P} $ is non singular and $ (y^1,\cdots,y^n) $ is a coordinate maps, this implies that $ (x^1,\cdots,x^n) $ is also a coordinate map, thus $ (V,(x^1,\cdots,x^n)) $ is also a chart. From the change of bases rule for the differential in charts we know that
	\[ \partial/\partial y_i = \sum_j \frac{\partial x_j}{\partial y_i} \frac{\partial}{\partial x_j} \]
	where from the definition $ x^j $
	\[ \partial/\partial y^i = \sum_j \left[ \inv{P} \right]_{ji} \frac{\partial }{\partial x^j}. \]
	Symbolically we can write this as
	\[ [ \partial/\partial y^1\ \partial/\partial y^2\ \cdots\ \partial/\partial y^n ] = [\partial/\partial x^1\ \partial/\partial x^2\ \cdots\ \partial/\partial x^n] \inv{P}. \]
	By multiplying both sides as $ P $ we will get
	\[ [\partial/\partial x^1\ \partial/\partial x^2\ \cdots\ \partial/\partial x^n] =  [ \partial/\partial y^1\ \partial/\partial y^2\ \cdots\ \partial/\partial y^n ]P  \]
	Thus we can conclude that 
	\[ [\partial/\partial x^1\ \partial/\partial x^2\ \cdots\ \partial/\partial x^n] = \left[(X_1)_p\ (X_2)_p\ \cdots\ (X_n)_p\right]. \]
\end{solution}


\begin{problem}[Local Maxima (form L. Tu)]
	A real valued function $ f:M\to \R $ on a manifold is said to have a local maximum at $ p \in M $ if there is a neighborhood $ U $ of $ p $ such that $ f(p) \geq f(q) $ for all $ q \in U $.
	\begin{enumerate}[(a)]
		\item Prove that if a differentiable function $ f:I\to \R $ defined on an open interval $ I $ has a local maximum at $ p \in I $ then $ f'(p) = 0 $.
		\item Prove that a local maximum of a $ C^\infty $ function $ f:M\to \R $ is a critical point of $ f $.
	\end{enumerate}
\end{problem}
\begin{solution}
	$\ $
	\begin{enumerate}[(a)]
		\item Let $ f $ has a local maximum at $ p \in I $. Thus $ f(p) \leq f(x) $ for all $ x $ sufficiently close to $ p $ from left. Thus
		\[ f'(p) = \lim_{x \to p^-} \frac{f(x) - f(p)}{x-p} \leq 0. \]
		On the other hand for all $ x $ sufficiently close to $ p $ from left we have $ f(p) \geq f(x) $. This implies that 
		\[ f'(p) = \lim_{x \to p^+} \frac{f(x) - f(p)}{x-p} \geq 0. \]
		Combining the two inequalities above we will get
		\[ f'(p) = 0. \]
		\item Let $ p \in M $ be a local maximum of a function $ f $. Let $ X_p \in T_pM $ be any tangent vectors, and $ c $ be a smooth curve with velocity vector $ X_p $ that starts from $ p $. Then 
		\[ X_p f = (f\circ c)'(0). \]
		Note that $ f\circ c $ is a real function from $ \R\to \R $ that has a local maxima at $ t = 0 $, thus $ (f\circ c)'(0) = 0 $. Let $ (U,\phi = (x^1,\cdots,x^n)) $ be any coordinate chart. From the fact above we know that 
		\[ \frac{\partial}{\partial  x^i}\big|_{p} f = 0 \qquad \text{for $ i = 1,2,\cdots,n $}. \]
		So by \autoref{prop:CriticalPointOfARealMap} we conclude that $ p $ is a critical point.
	\end{enumerate}
	\begin{carefull}[The converses is not true]
		Note that the converse of the theorem above is not true. A critical point of a real-valued function is not necessarily an local maxima. It can be a so called saddle point. For instance, the function $ F:\R^2 \to \R$ given by $x^3 -6xy + y^2 $ has a saddle point at $ (0,0) $.
	\end{carefull}
\end{solution}


\begin{problem}[Maxima Integral Curve (from L. Tu)]
	Let $ X = x^2 d/dx $ be a vector field on the real number line $ \R $. Fine maximal integral curve of $ X $ starting at $ x =2 $.
\end{problem}
\begin{solution}
	Let $ c:(a,b) \to \R $ given as $ c(t) = x(t) $ such curve such that $ c(0) = 2 $. By \autoref{prop:curvesToComputeDifferential} we can write
	\[ c'(t) = \dot{c}(t) \frac{d}{dt} = \dot{x} \frac{d}{dx}. \]
	For $ c $ to an integral curve we need to have
	\[ c'(t) = X|_{c(t)} \qquad \text{for } t\in (a,b).  \]
	Thus we will have
	\[ \dot{x} = x^2, \qquad x(0) = 2. \]
	By solving this initial value problem with separation of variables method, we will get
	\[ x(t) = \frac{-2}{2t+1}. \]
	The maximal domain for this curve is $ t \in (-\infty, -1/2) $.
\end{solution}


\begin{problem}[Recovering vector field from a global flow (from L. Tu)]
	Let $ F: \R\times \R^2 \to \R^2 $ be a global flow on $ \R^2 $ give by
	\[ F\left( t, \vectt{x}{y} \right) = \matt{\cos t}{-\sin t}{\sin t}{\cos t} \vectt{x}{y}. \]
	Recover the vector field that generates this global flow.
\end{problem}
\begin{solution}
	Since $ F $ is a global flow, then for any given $ p = \vectt{x}{y} \in \R^2 $, $ F(t,p) $ will be an integral curve starting at $ p $. Thus its velocity vector at $ p $ will give the vector field at $ p $. The velocity vector of $ F $ at $ p $ is given by
	\[ X_p = \frac{\partial F }{\partial t}\big|_{0}(0,p) = \matt{-\sin t}{-\cos t}{\cos t}{-\sin t}\big|_{0} \vectt{x}{y} = -y\frac{\partial }{\partial x} + x \frac{\partial }{\partial y}. \]
	Thus the generating vector field is given as
	\[ X = -y \frac{\partial }{\partial x } + x \frac{\partial }{\partial y}. \]
\end{solution}

\begin{problem}[Equality of vector fields (from L. Tu)]
	Show that two $ C^\infty $ vector fields $ X $ and $ Y $ on a manifold $ M $ are equal if and only if for every $ C^\infty $ function $ f $ on $ M $ we have $ Xf = Yf $.
\end{problem}

\begin{solution}
	$ \ $
	\begin{itemize}
		\item[$\boxed{\Longrightarrow}$] This direction is obvious.
		\item[$\boxed{\Longleftarrow}$] For this direction we will use the proof by contrapositive. Let $ X, Y $ be two vector fields that do not match at some $ p \in M $, i.e. $ X_p \neq Y_p $. With reference to a local coordinate chart we can expand $ X_p, Y_p $ in terms of the local bases vectors. We will have
		\[ X_p = \sum_i a^i \frac{\partial}{\partial  x^i}\big|_{p}, \qquad Y_p = \sum_i b^i\frac{\partial}{\partial  x^i}\big|_{p}. \]
		Since $ X_p \neq Y_p $ then there exists at least one $ i $ such that $ a^i \neq b^i $. Define $ f:M \to \R $ to be $ f = x^i $. Then 
		\[ X_pf = a^i, \qquad Y_pf = b^i. \]
		Since $ a^i \neq b^i$, then we conclude that $ X_pf \neq Y_p $. This implies $ Xf \neq Yf $.
	\end{itemize}
\end{solution}

\begin{problem}[Vector field on an odd sphere (from L. Tu)]
	Let $ x^1,y^1,\cdots,x^n,y^n $ ve the standard coordinates on $ \R^{2n} $. The unit sphere $ S^{2n-1} $ in $ \R^{2n} $ is defined by the equation $ \sum_{i=1}^n (x^i)^2 + (y^i)^2  = 1$. Show that 
	\[ X = \sum_{i}^{n} \underbrace{-y^i}_{a^i} \frac{\partial}{\partial  x^i} + \underbrace{x^i}_{b^i} \frac{\partial }{\partial y^i} \]
	is a nowhere-vanishing smooth vector field on $ S^{2n-1} $. Since all spheres of the same dimension are diffeomorphic, this proves that on every odd-dimensional sphere there is a nowhere-vanishing smooth vector field. It is a classical theorem of differential and algebraic topology that on an even-dimensional sphere every continuous vector field must vanish somewhere.
\end{problem}

\begin{solution}
	For this, we first use the following result that is proved elsewhere in the problems.
	\begin{lemma}
		The unit sphere $ S^n $ in $ R^{n+1} $ is defined by the equation $ \sum_{i=1}^{n+1} (x^i)^2 = 1 $. For $ p = (p^1,\cdots,p^{n+1}) \in S^n $, a necessary and sufficient condition for 
		\[ X_p = \sum a^i \partial/\partial x^i|_{p} \in T_p(\R^{n+1})  \]
		to be tangent to $ S^n $ at $ p $ is $ \sum a^i p^i = 0 $.
	\end{lemma}
	Let $ p = (p^1,q^1,\cdots,p^n,q^n) $ be a point on the unit sphere $ S^{2n-1} $. Thus from definition we have
	\[ \sum_{i=1}^{n} (p^i)^2 + (q^i)^2 = 1. \]
	At this point the vector field $ X $ is
	\[ X_p = \sum_{i=1}^{n} - q^i \frac{\partial}{\partial x^i} + p^i \frac{\partial}{\partial y^i}. \]
	Thus it follows immediately that 
	\[ \sum_{i=1}^{n} a^i p^i + b^i q^i = \sum_{i=1}^{n} -q^i p^i + p^i q^i = 0.  \]
	This shows that $ X_p $ is a tangent vector to $ p \in S^{2n-1} $. On the other hand, for the vector field $ X $ to vanish, all of its coefficient functions must vanish, which occurs only at the origin, where it is not located on the unit sphere $ S^{2n-1} $.
	
	We summarize the result of this problem in the theorem below.
	\begin{theorem}[Hairy ball theorem]
		On an even-dimensional unit sphere, every continuous tangent vector field should vanish somewhere.
	\end{theorem}
\end{solution}

\begin{problem}[Maximal integral curve on a punctured line (from L. Tu)]
	Let $ M $ be $ \R - \set{0} $ and let $ X $ be the vector field $ d/dx $ on $ M $. Find the maximal integral curve of $ X $ starting at $ x = 1 $.
\end{problem}
\begin{solution}
	Let $ c:(a,b) \to \R $ be the integral curve of $ X = d/dx $ starting at $ p = 1 $. Since $ c'(t) = \dot{c}d/dx $, then we need to have
	\[ \dot{c} = 1,\quad c(0) = 1. \]
	Solving the initial value problem above we will get
	\[ c(t) = t + 1. \]
	The maximal domain for this curve is $ (0,\infty) $.
\end{solution}

\begin{problem}[Integral curves in the plane (from L. Tu)]
	Find the integral curves of the vector field
	\[ X_{(x,y)} = x\frac{\partial }{\partial x} - y\frac{\partial }{\partial y} = \vectt{x}{-y} \quad \text{on } \R^2. \]
\end{problem}
\begin{solution}
	Let $ c:(a,b) \to \R^2 $ given by $ c(t) = (x(t),y(t)) $ to be an integral curve. Then we need to have
	\[ \vectt{\dot{x}}{\dot{y}} = \vectt{x}{-y}. \]
	Thus the integral curves are given as
	\[ c(t) = (x(t),y(t)) = (Ae^{t},Be^{-t}), \]
	where $ A $ and $ B $ are determined by the starting point of the curve. 
\end{solution}

\begin{problem}[Hypersurface (from L. Tu)]
	Show that the solution set of the equation $ x^3 + y^3 + z^3 = 1 $ in $ \R^3 $ is a manifold of dimension 2. 
\end{problem}
\begin{solution}
	Define $ F: \R^3 \to \R $ as $ f(x,y,z) = x^3 + y^3 + z^3 $. The solution set of $ x^3 + y^3 + z^3 = 1 $ is the same as the level curve of $ f $ with level 1. Since $ f_* = \left[3x^2 \quad 3y^2 \quad 3z^2 \right] $ is not surjective only at $ (0,0,0) $, and this point is not on the level set, then $ f_* $ is surjective for all points of the level set, hence a regular level set. By \autoref{thm:regularLevelSetTheorem} the level set is a submanifold of dimension $ 3-1 =2 $, and by \autoref{prop:regularSubmanifoldIsAManifold} it is a manifold of dimension 2. 
\end{solution}

\begin{problem}[Regular values (from L. Tu)]
	Define $ f:\R^2 \to \R $ by
	\[ f(x,y) = x^3 - 6xy + y^2. \]
	Find all values $ c \in \R $ for which the level set $ \inv{f}(c) $ is a regular submanifold of $ \R^2 $.
\end{problem}
\begin{solution}
	A regular level set of a function is the pre-image of a regular value of that function. We need to find the critical points of the map. The differential of the map is given by
	\[ F_* = \begin{bmatrix}
		3x^2-6y & -6x + 2y
	\end{bmatrix} \]
	Thus $ F $ is not surjective at the points $ (0,0) $ and $ (6,18) $ in its domain. So for all $ c \in \R - \set{0,-108} $ the level curve $ \inv{f}(c) $ will be a regular level set. 
\end{solution}

\begin{problem}[Solution set of one equation (from L . Tu)]
	Let $ x,y,z,w $ be the standard coordinates on $ \R^4 $. Is the solution set of $ x^5 + y^5 + z^5 + w^5 = 1 $ in $ \R^4 $ a smooth manifold? Explain why or why not. (Assume that the subset is given the subspace topology).
\end{problem}
\begin{solution}
	newEnv
\end{solution}


\begin{problem}[Solution set of two polynomial equations (from L. Tu)]
	Decide whether the subset $ S $ of $ \R^3 $ defined by the two equations
	\[ x^3 + y^3 + z^3 = 1, \qquad x+y+z = 0, \]
	is a regular submanifold of $ \R^3 $.
\end{problem}
\begin{solution}
	Define $ F:\R^3 \to \R^2 $ as $ (f_1, f_2) = (x^3+y^3+z^3, x+y+z) $. The solution set of the two polynomial equations above is the same as the level set $ \inv{F}(0,1) $. The Jacobian of this map at $ p \in \R^3 $ is given by
	\[ F_* = \begin{bmatrix}
		3x^2 & 3y^2 & 3z^2 \\
		1 & 1 & 1
	\end{bmatrix} \]
	For $ F_* $ not to be surjective on the level set, then all of its $ 2\times 2 $ minors should vanish. I.e. in order for $ F_* $ not to be surjective we need to have
	\[ 3x^2 - 3y^2 = 0, \qquad 3x^2 - 3z^2 = 0, \qquad 3y^2 - 3z^2 = 0. \]
	The solution set of these equations is $ A = A_1 \cup A_2 \cup A_3 \cup A_4 \cup A_5 $ where
	\begin{align*}
		&A_1 = \set{(t,t,t): t\in \R-\set{0}},\quad &&A_2 = \set{(-t,t,t): t\in \R-\set{0}},\\
		&A_3 = \set{(t,-t,t): t\in \R-\set{0}},\quad &&A_4 = \set{(t,t,-t): t\in \R-\set{0}},\qquad A_5=\set{(0,0,0)}.
	\end{align*}
	Non of these solution sets satisfy the two polynomial equations (or equivalently are not on the level set of $ F $ with level $ (1,0) $). So the level set is a regular level set, hence a manifold of degree 3-2=1. 
\end{solution}


\begin{problem}[Maximal integral curve in the plane (from L. Tu)]
	Find the maximum integral curve $ c(t) $ starting at the point $ (a,b \in \R^2) $ of the vector field $ X_{(x,y)} = \partial/\partial x + x \partial/\partial y $ on $ \R^2 $.
\end{problem}
\begin{solution}
	Let $ c:(\alpha,\beta) \to \R^2 $ be an integral curve given as $ c(t) = (x(t),y(t)) $ of $ X $ starting at $ (a,b) $. Since
	\[ c'(t) = \dot{x}\frac{\partial }{\partial x} + \dot{y} \frac{\partial }{\partial y}. \]
	Thus we need to have
	\[ \vectt{\dot{x}}{\dot{y}} = \vectt{1}{x},\qquad \vectt{x(0)}{y(0)} = \vectt{a}{b}. \]
	Solving the ODE system above we will get
	\[ x(t) = t + A, \qquad y(t) = \frac12 t^2 + A t + B. \]
	Applying the initial conditions we will get
	\[ A = a,\qquad B = b. \]
	So the integral curve is given as
	\[ c(t) = (t+a, \frac12 t^2 + at + b). \]
	The maximal domain for this integral curve is $ (-\infty,\infty) $.
\end{solution}

\begin{problem}[Integral curve starting at a zero of a vector field (from L. Tu)]
	$ \ $
	\begin{enumerate}[(a)]
		\item Suppose the smooth vector field $ X $ on a manifold $ M $ vanishes at a point $ p \in M $. Show that the integral curve of $ X $ with initial point $ p $ is the constant curve $ c(t) \equiv p $.
		\item Show that if $ X $ is the zero vector field on a manifold $ M $, and $ c_t(p) $ is the maximal integral curve of $ X $ starting at $ p $, then the one parameter group of diffeomorphisms $ c: \R \to \operatorname{Diff}(M) $ is the constant map $ c(t) \equiv \mathbbm{1}_M $.
	\end{enumerate}
\end{problem}
\begin{solution}
	$ \ $
	\begin{enumerate}[(a)]
		\item 	Let $ c: (a,b) \to M $ be the integral curve of $ X $ starting at $ p $. Thus we have the following initial value problem
		\[ \dot{c}^i(t) = 0, \quad c^i(0) = p^i. \]
		This implies that the integral curve is the constant curve $ c(t) \equiv p $.
		\item For this we need to find the global flow generated by the vector field. Let the global flow be $ F: \R \times M \to M $. From part (a) above we can see that 
		\[ F(t,p) \equiv p \qquad \forall t \in \R,\ p \in M. \]
		Thus for a fixed $ p \in M $ the function $ F_t: M \to M $ is the identity map $ \mathbbm{1}_M $.
	\end{enumerate}
\end{solution}

\begin{problem}[Maxima integral curve (from L. Tu)]
	Let $ X $ be the vector field $ x d/dx $ on $ \R $. For each $ p \in \R $, find the maximal integral curve $ c(t) $ of $ X $ starting at $ p $.
\end{problem}
\begin{solution}
	Let $ c: (a,b) \to \R$ given as $ c(t) = x(t) $ be the maximal integral curve of interest. Then we will have the following initial value problem.
	\[ \dot{x} = x, \qquad x(0) = p.\]
	The solution of the initial value problem above is given as
	\[ x(t) = p e^t. \]
	The maximal domain of this curve is given as $ (-\infty,\infty) $.
\end{solution}

\begin{problem}[Maximal integral curve (from L. Tu)]
	Let $ X $ be the vector field $ x^2 d/dx $ on the real line $ \R $. For each $ p > 0 $ in $ \R $, find the maximal integral curve of $ X $ with initial point $ p $.
\end{problem}
\begin{solution}
	Let $ c: (a,b) \to \R$ given as $ c(t) = x(t) $ be the maximal integral curve of interest. Then we will have the following initial value problem.
	\[ \dot{x} = x^2, \qquad x(0) = p. \]
	The solution of the initial value problem above is given as 
	\[ x(t) = \frac{-1}{t - 1/p} = \frac{p}{1-pt}.  \]
	If $ p > 0 $, the maximal domain for this curve is $ (-\infty, 1/p) $. However, if $ p < 0 $, then the maximal domain for this curve is given as $ (1/p,\infty) $.
\end{solution}

\begin{problem}[Reparameterization of an integral curve]
	Suppose $ c: (a,b) \to M $ is an integral curve of the smooth vector field $ X $ on $ M $. Show that for any real number $ s $, the map
	\[ c_s: (a+s,b+s) \to M, \qquad c_s(t) = c(t-s), \] 
	is also an integral curve of $ X $.
\end{problem}
\begin{solution}
	Without loss of generality, let $ 0 \in (a,b) $. Let $ c(0) = p $, and $ x'(0) = X_p $. Then $ s \in (a+s, b+s) $. So for $ c_s(s) $ we have
	\[ c_s(s) = c(s-s) = c(0) = p. \]
	Furthermore, since from the chain rule we have $ c'_s(t) = c'(t-s) $ then 
	\[ c'_s(s) = c'(0) = X_p. \]
	Thus $ c_s(t) $ is an integral curve of $ X $ that passes from the point $ p $ at $ t = s $.
	\begin{observation}[Uniqueness of integral curve]
		As we discussed in \autoref{thm:existenceUniqTheoremODE}, the solution of an initial value problem is unique. But based on the observation above, a more accurate term to state the statement above is to say that the solution to an initial value problem is unique up to a reparameterization.
	\end{observation}
\end{solution}

\begin{problem}[Lie bracket of vector fields (from L. Tu)]
	If $ f $ and $ g $ are $ C^\infty $ functions and $ X $ and $ Y $ are $ C^\infty $ vector fields on a manifold $ M $, show that 
	\[ [fX, gY] = fg[X,Y] + f(Xg)Y - g(Yf)X. \]
\end{problem}
\begin{solution}
	To see this, first note that $ X,Y $ are both derivations. Thus expanding the Lie bracket we will get
	\[ [fX,gY] = (fX)(gY) - (gY)(fX) \]
	For the first term in RHS we have
	\[ (fX)(gY) = fX(gY) = f[X(g)Y + gXY] = fX(g)Y + fg XY, \]
	while for the second term we have
	\[ (gY)(fX) = gY(fX) = g[Y(f)X + fYX] = gY(f)X + gf YX. \]
	Combining these two terms while noting that $ fg = gf $
	\[ [fX,gY] = fg (XY - YX) + fX(g)Y - gY(f)X = fd[X,Y] + fX(g)Y - gY(f)X. \]
\end{solution}

\begin{problem}[Lie bracket of vector fields on $ \R^2 $ (from L. Tu)]
	Compute the Lie bracket 
	\[ [-y\partial/\partial x + x\partial/\partial y, \partial/\partial x] \]
	on $ \R^2 $.
\end{problem}
\begin{solution}
	Using the definition we can write
	\begin{align*}
		[-y\partial/\partial x + x\partial/\partial y, \partial/\partial x] &= (-y\partial/\partial x + x\partial/\partial y)( \partial/\partial x) -( \partial/\partial x)(-y\partial/\partial x + x\partial/\partial y) \\
		&= -y \partial^2/\partial x^2 + x \partial^2/\partial y\partial x + y\partial^2/\partial x^2 - x\partial^2/\partial x \partial y \\
		& = 0.
	\end{align*}
\end{solution}

\begin{problem}[Lie bracket in local coordinates]
	Consider two $ C^\infty $ vector field $ X, Y $ on $ \R^n $:
	\[ X = \sum a^i \frac{\partial}{\partial x^i},\qquad Y = \sum b^j \frac{\partial}{\partial x^j}, \]
	where $ a^i, b^j$ are $ C^\infty $ functions on $ \R^n $. Since $ [X,Y] $ is also a $ C^\infty $ vector field $ \R^n $,
	\[ [X,Y] = \sum c^k \frac{\partial }{\partial x^k} \]
	for some $ C^\infty $ function $ c^k $. Find the formula for $ c^k $ in terms of $ a^i $ and $ b^j $.
\end{problem}
\begin{solution}
	By the bilinearity of the Lie bracket, we can write
	\[ [X,Y] = \sum_i [a^i \frac{\partial}{\partial x^i}, \sum_j b^j \frac{\partial}{\partial x^j}] 
	= \sum_{i,j}[a^i \frac{\partial }{\partial x^i}, b^j\frac{\partial }{\partial x^j}]
	= \sum_{i,j}( a^i b^j [\frac{\partial }{\partial x^i},\frac{\partial }{\partial x^j}] + a^i \frac{\partial b^j}{\partial x^i}\frac{\partial }{\partial x^j} - b^j \frac{\partial a^i}{\partial x^j}\frac{\partial }{\partial x^i})
	 \]
	By using the fact that $ [\frac{\partial }{\partial x^i},\frac{\partial}{\partial x^j}] = 0 $ for all $ i,j $, and by re-arranging the similar terms we can get
	\[ [X,Y] = \sum_k 
	\det \matt{\sum_i a^i \frac{\partial }{\partial x^i}}{\sum_j b^j \frac{\partial}{\partial x^j}}{a^k}{b^k}\ \frac{\partial }{\partial x^k}.\]
	Thus we have
	\[ c^k = \det \matt{\sum_i a^i \frac{\partial }{\partial x^i}}{\sum_j b^j \frac{\partial}{\partial x^j}}{a^k}{b^k}. \]
\end{solution}