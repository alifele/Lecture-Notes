\chapter{Tangent Spaces}

\section{Tangent Spaces}

For several reasons, the notion of tangent space is more easily developed using algebraic view point rather than the geometric one. Then we can easily construct our geometric intuition from using the developed theory. We start with the notion of derivation.


\begin{definition}[Derivation at a point]
	Let $ M $ be a manifold and $ p $ a point of manifold. Let $ C_p^\infty(M) $ denote the germ of $ C^\infty $ functions at $ p $. Derivation at $ p $ is a linear map 
	\[ D: C_p^\infty(M) \to \R \]
	such that for $ f,g \in C_p^\infty(M) $ we have
	\[ D(fg) = D(f)g + fD(g). \]
\end{definition}
The definition above is a purely algebraic one.
\begin{remark}
	The set of all point derivations at $ p $ forms a vector space.
\end{remark}

\begin{definition}[Tangent space]
	Let $ M $ be a manifold and $ p \in M $. The tangent space at $ p $ denoted by $ T_p(M) $ is the vector space of all point derivations at $ p $.
\end{definition}
In the following proposition we will see a very familiar point derivation.
\begin{proposition}[Partial derivative $ \partial/\partial x^i |_p $ is a tangent vector]
	Let $ M $ be a manifold and $ p \in M $. Let $ (U,\phi) $ be a coordinate chart containing $ p $ and $ f:M \to \R $ a function defined at a neighborhood of $ p $. By definition we have
	\[ \frac{\partial }{\partial x^i}\big|_p f = \frac{\partial}{\partial r^i}\big|_{\phi(p)} (f\circ\inv{\phi}) \]
	where $ x^i = r^i \circ f $ with $ r^i $ as the standard coordinate function of $ \R^n $. The map $ \frac{\partial}{\partial x^i}\big|_{p} $ is a point derivation at $ p $, thus a tangent vector.
\end{proposition}
\begin{proof}
	First observe that $ \partial/\partial x_i|_p $ is indeed a linear map from $ C_p^\infty(M) $ to $ \R $. Let $ f,g \in C_p^\infty(U) $ and $ (U,x^1,\cdots,x^n) $ a coordinate chart. Then we have
	\begin{align*}
		\frac{\partial}{\partial x^i}\big|_p (fg) &=  \frac{\partial}{\partial  r^i}\big|_{\phi(p)}((fg)\circ \inv{\phi} ) = \frac{\partial}{\partial  r^i}\big|_{\phi(p)}(f\circ\inv{\phi} \cdot g\circ\inv{\phi}) \\
		&= \frac{\partial}{\partial  r^i}\big|_{\phi(p)}(f\circ\inv{\phi})\cdot g(p) + f(p)\cdot \frac{\partial}{\partial  r^i}\big|_{\phi(p)} (g\circ\inv{\phi})\\
		&=(\frac{\partial}{\partial  x^i}\big|_{p}f ) g + f(\frac{\partial}{\partial  x^i}\big|_{p} g).
	\end{align*}
\end{proof}

\begin{definition}[Differential of a smooth map]
	Let $ F:N\to M $ be a smooth map between manifolds. This map induces a \emph{linear} map  called the \emph{differential of $ F $}
	\[ F_*: T_p N \to T_{F(p)}M \]
	where for $ X_P \in T_p N $ we have
	\[ (F_*(X_p))f = X_p(f\circ F) \]
	where $ f $ is a germ at $ p $.
\end{definition}

\begin{proposition}
	Let $ F:N\to M $. It differential $ F_* $ is a derivation at $ F(p) $ for $ p \in N $. 
\end{proposition}
\begin{proof}
	See \autoref{prob:DifferentialOfAMapIsADerivation}
\end{proof}

\subsubsection{Differential of a map between Euclidean spaces}
To see what kind of an object is the differential of a map, we calculate it explicitly on Euclidean spaces. Let $ F:\R^n \to \R^m $ be a map and let $ x^1,\cdots,x^n $ be the standard coordinate functions of $ \R^n $ while $ y^1,\cdots,y^m $ be the standard coordinate functions of $ \R^m $. Let $ p \in \R^n $. since $ F_* $ is a linear map between two vector spaces, then we can write it as a matrix once we fix the basis for the spaces. The basis for $ T_p(\R^n) $ is the set $ \set{\partial/\partial x^i|_p}_{i} $, while for $ T_p(\R^m) $ is $ \set{\partial/\partial y^j|_p}_j $. To get the matrix representation of $ F_* $ we need to find its effect on the basis vectors
\[ F_*(\frac{\partial}{\partial  x^j}\big|_{p}) = \sum_{k} a^k_j\  \frac{\partial}{\partial  y^k}\big|_{F(p)}. \]
By applying it on $ y_i $ we will get
\[ a_j^i = F_*(\frac{\partial}{\partial  x^j}\big|_{p})y^i = \frac{\partial}{\partial  x^j}\big|_{p} (y^i \circ F) =  \frac{\partial}{\partial  x^j}\big|_{p}F^i. \]
This is precisely the Jacobian matrix of $ F $ evaluated at $ p $.

\subsubsection{The Chain Rule}
A nice thing behind this algebraic point of view is that we can now write the notion of chain rule in a much simpler form. The following propositions captures this ides.

\begin{proposition}[The chain rule]
	\label{prop:ChainRule}
	Let $ F: N \to M $ and $ G:M\to P $ be maps between manifolds, and $ p \in N $. Then the differential of $ F $ at $ p $ and $ G $ at $ F(p) $ are linear maps
	\[ T_pN \xrightarrow{F_{*,p}} T_pM \xrightarrow{G_{*,F(p)}} T_pP. \]
	Then we have
	\[ (G\circ F)_{*,p} = G_{*,F(p)} F_{*,p}. \]
\end{proposition}
\begin{proof}
	Let $ X_p \in T_p(N) $. From definition we have for some $ f: G \to \R $ we have
	\[ ((G\circ F)_{*,p}X_p)f = X_p(f\circ G\circ F) = X_p ((f\circ G)\circ F) = (F_{*,p}X_p)(f\circ G).\]
	Let $ Y_q = F_{*,p}X_p $ where $ q = F(p) $. Then 
	\[ Y_q (f\circ G) = (G_{*,q}Y_q) f.  \]
	This implies that 
	\[ (G\circ F)_{*,p} = G_{*,q} F_{*,p} \]
	and this completes the proof.
\end{proof}

\begin{example}
	Let $ F:\R \to \R^3 $ and $ G:\R^3 \to \R $. Then the differential of the composite function $ G\circ F $ is the matrix multiplication of the differential of each map. I.e.
	\[ (G \circ F)_* = G_* F_* = (G_x\ G_y\ G_z)\cdot 
	\begin{pmatrix}
		F_1' \\
		F_2' \\
		F_3'
	\end{pmatrix}
	= G_x F_1' + G_y F_2' + G_z F_3' 
	= \frac{\partial G}{\partial x} \frac{d F_1}{dt} 
	+\frac{\partial G}{\partial y} \frac{d F_2}{dt} 
	+\frac{\partial G}{\partial z} \frac{d F_3}{dt},
	 \]
	 which is the same chain rule in calculus.
\end{example}
Writing the chain rule in this form not only is visually appealing, but also we can deduce two important Corollaries from it as follows.

\begin{corollary}[Diffeomorphism of Manifolds vs. Isomorphism of Tangent Spaces]
	Let $ F:N\to M $ be a diffeomorphism of manifolds and $ p \in N $. Its differential $ F_* $ is an isomorphism of tangent spaces $ T_pN $ and $ T_{F(p)}M $.
\end{corollary}
\begin{proof}
	$ F $ being a diffeomorphism of manifolds, it implies that we have $ G: M \to N $ such that $ G\circ F = \mathbbm{1}_N $ and $ F\circ G = \mathbbm{1}_M $. Now using \autoref{prop:ChainRule} we can write
	\begin{align*}
		(G\circ F)_* = G_* \circ F_* = (\mathbbm{1}_N)_* = \mathbbm{1}_{T_pN}\\
		(F\circ G)_* = F_* \circ G_* = (\mathbbm{1}_M)_* = \mathbbm{1}_{T_{F(p)}M}.
	\end{align*}
	This shows that $ F_* $ and $ G_* $ are the right and left inverses of each other, hence an isomorphism of vector spaces.
\end{proof}
\begin{corollary}[Invariance of Dimension]
	Let $ U \subset \R^n $ and $ V \subset R^m $ be open subsets that are diffeomorphic. Then $ m = n $.
\end{corollary}
\begin{proof}
	Let $ F: U \to V $ be a diffeomorphism and $ p \in U $. Then $ F_* $ is an isomorphism of tangent spaces $ T_pU $ and $ T_{F(p)}V $, thus of the same dimension. Since in a Euclidean space the tangent space has the same dimension as the space itself (by definition and the way that we construct a basis for tangent space using the standard coordinate vectors) then this implies that $ n = m $.
\end{proof}

\begin{corollary}[Tangent Space Has the Same Dimension as Manifold]
	\label{cor:TangentSpaceIsoMorphismToRn}
	Let $ M $ be a manifold of dimension $ n $ and $ p \in M $. Then $ T_p(M) $ is isomorphic to $ \R^n $ and has dimension $ n $.
\end{corollary}
\begin{proof}
	Let $ (U,\phi) $ be a coordinate chart containing the point $ p $. Since $ M $ is assumed to be a smooth manifold, then the coordinate map $ \phi $ automatically upgrades to a diffeomorphism between $ U $ and $ \R^n $. Thus the tangent spaces $ T_p(U) $ and $ T_{\phi(p)}(\R^n) $ are isomorphic. This implies that $ T_p(U) $ is also of dimension $ n $.
\end{proof}

\section{Basis for the Tangent Space at a Point}
\begin{proposition}[Differential of a Coordinate Map]
	\label{prop:DifferentialOfACoordinateMap}
	Let $ M $ be a manifold, $ p \in M $, and $ (U,x^1\cdots,x^n) $ a coordinate chart containing $ p $. Then
	\[ \phi_* (\frac{\partial}{\partial  x^i}\big|_{p}) = \frac{\partial}{\partial  r^i}\big|_{\phi(p)} \]
\end{proposition}
\begin{proof}
	This follows immediately from the definition. Namely for a function $ f $ defined on $ U $ we have
	\begin{align*}
		(\phi_*(\frac{\partial}{\partial  x^i}\big|_{p}))f = \frac{\partial}{\partial  x^i}\big|_{p}(f\circ \phi) = \frac{\partial}{\partial  r^i}\big|_{\phi(p)}(f\circ\phi\circ\inv{\phi}) = \frac{\partial}{\partial  r^i}\big|_{\phi(p)}f.
	\end{align*}
	This implies 
	\[ \phi_*(\frac{\partial}{\partial  x^i}\big|_{p}) = \frac{\partial}{\partial  r^i}\big|_{\phi(p)}. \]
\end{proof}
\begin{remark}
	\label{remark:differentialOfCoordinateMapActingOnBases}
	Remember that in \autoref{def:DirectionalDerivativeOnManifold} we defined
	\[ \frac{\partial}{\partial  x^i}\big|_{p} f = \frac{\partial}{\partial  r^i}\big|_{\phi(p)}(f\circ\inv{\phi}).\tag{\eighthnote} \]
	where $ (U,\phi) $ is a coordinate chart containing $ p $ and $ x^i = r^i \circ \phi $. The proposition above justifies this definition. This proposition states that $ T_p(M) \simeq T_{\phi(p)}(R^n) $ with $ \phi_* $ as an isomorphism. This basically means that $ \partial/\partial x^i $ maps to $ \partial/\partial r^i $ with map $ \phi_* $, or equivalently,  $ \partial/\partial r^i $ maps to $ \partial/\partial x^i $ with the map $ \inv{\phi}_* $. I.e.
	\[ \frac{\partial}{\partial  x^i}\big|_{p} \mapsto \frac{\partial}{\partial  r^i}\big|_{\phi(p)} \quad\text{via } \phi_* \]
	and 
	\[ \frac{\partial}{\partial  r^i}\big|_{p} \mapsto \frac{\partial}{\partial  x^i}\big|_{\phi(p)} \quad\text{via } \inv{\phi}_* \]
	Thus starting with the LHS of $ (\eighthnote) $ we can write
	\[ \frac{\partial}{\partial  x^i}\big|_{p}f = (\inv{\phi}_*(\frac{\partial}{\partial  r^i}\big|_{\phi(p)}))f = \frac{\partial}{\partial  r^i}\big|_{\phi(p)}(f\circ \inv{\phi}).  \]
\end{remark}


\begin{proposition}[Basis for Tangent Space]
	\label{prop:BasisForTangetSpace}
	Let $ M $ be a manifold of dimension $ n $ and $ p \in M $. Then the vectors $ \partial/\partial x^1|_p,\cdots, \partial/\partial x^n|_p $ is a basis for the tangent space $ T_p(M) $.
\end{proposition}
\begin{proof}
	Let $ (U,\phi) $ be a coordinate chart containing $ p $. By \autoref{cor:TangentSpaceIsoMorphismToRn} we see that $ T_pM $ is isomorphic to $ \R^m $ with $ \phi_* $ as a possible isomorphism. Since an isomorphism of vector spaces carries basis to a basis, and by \autoref{prop:DifferentialOfACoordinateMap} $ \phi_* $ maps the standard basis $ \frac{\partial}{\partial  r^1}\big|_{\phi(p)},\cdots,\frac{\partial}{\partial  r^n}\big|_{\phi(p)} $ to $ \partial/\partial x^1|_p,\cdots, \partial/\partial x^n|_p $ respectively, then this forms a basis for $ T_p(M) $.
\end{proof}

\begin{remark}
	For a point $ p \in M $, once we fix a coordinate chart containing $ p $ we have a basis for $ T_p(M) $
\end{remark}


\begin{proposition}[Transition Matrix for Coordinate Vectors]
	Let $ M $ be a manifold and $ p \in M $. Let $ (U,\phi=(x^1,\cdots,x^n)) $ and $ (V,\psi=(y^1,\cdots,y^n)) $ be two coordinate charts containing $ p $. By \autoref{prop:BasisForTangetSpace} the following two collection of tangent vectors form two possible bases for $ T_pM $
	\[ \set{ \frac{\partial}{\partial  x^1}\big|_{p}, \cdots, \frac{\partial}{\partial  x^n}\big|_{p} }, \qquad \set{\frac{\partial}{\partial  y^1}\big|_{p},\cdots,\frac{\partial}{\partial  y^n}\big|_{p}}. \]
	The transition rule between these two basis is given by
	\[ \frac{\partial}{\partial  x^j} = \sum_{i} \frac{\partial y^i}{\partial x^j} \frac{\partial}{\partial y^i},  \]
	on $ U \cap V $.
\end{proposition}
\begin{proof}
	We can think of this as $ \psi $ is a map between manifolds, i.e. $ \psi: U\cap V \subset M \to \R^n $. Considering the chart $ (W,\mathbbm{1}_{\R^n}) $ for the target manifold of $ \psi $, what we are looking after is to see how does $ \psi_* $ maps the tangent vector $ \partial/\partial x^j $. For $ \psi_* $ it is a map from $ T_pN $ to $ T_{\psi(p)}\R^n $ where the basis for $ T_{\psi(p)}\R^n $ is the collection $ \set{\partial /\partial r^i}_i $. Thus we can write
	\[ \psi_*(\partial/\partial x^j|_p) = \sum_k a_j^k \partial/\partial r^k|_{\psi(p)}. \]
	Applying $ r^i $ to both sides will result in 
	\[ a_j^i  =  \phi_*(\partial/\partial x^j|_p) r^i = \partial/\partial x^j|_p (r^i\circ\phi) = \partial/\partial x^j|_p y^i. \]
	So far we have shown that 
	\[ \psi_*(\frac{\partial }{\partial x^j}) = \sum_i \frac{\partial y^i}{\partial x^j}\frac{\partial}{\partial r^i} \]
	Now by applying $ \inv{\psi_*} $ to both sides and using the observation in \autoref{remark:differentialOfCoordinateMapActingOnBases} we can write
	\[ \frac{\partial}{\partial x^j} = \sum_i \frac{\partial y^i}{\partial x^j} \frac{\partial}{\partial y^i} \]
	where we have used the fact that $ \inv{\psi}_* $ is linear.
\end{proof}
\begin{remark}
	The relation above can be written symbolically as matrix multiplication
		\[ 
	\begin{bmatrix}
		\partial/\partial x^1 \\
		\partial/\partial x^2 \\
		\vdots \\
		\partial/\partial x^n
	\end{bmatrix} = 
	\begin{pmatrix}
		\partial y^1/\partial x^1 & \partial y^1/\partial x^2 &  \cdots & \partial y^1/\partial x^n \\
		\partial y^2/\partial x^1 & \partial y^2/\partial x^2 &  \cdots & \partial y^2/\partial x^n \\
		\vdots & \vdots & \ddots & \vdots \\
		\partial y^n/\partial x^1 & \partial y^n/\partial x^2 &  \cdots & \partial y^n/\partial x^n
	\end{pmatrix}^T
	\begin{bmatrix}
		\partial/\partial y^1 \\
		\partial/\partial y^2 \\
		\vdots \\
		\partial/\partial y^n
	\end{bmatrix}
	\]
\end{remark}


\subsubsection{A Local Expression for the Differential}
We start with the following proposition.
\begin{proposition}[Local Expression for the Differential]
	Let $ F: N\to M $ be a map between manifolds, and $ p \in N $. Let $ (U,x^1\cdots,x^n) $ be a coordinate chart in $ N $ containing $ p $ and $ (V,y^1,\cdots,y^n) $ containing $ F(p) $ be a coordinate chart in $ M $. Relative to the bases $ \set{\partial /\partial x^j|_{p}} $ for $ T_pN $ and $ \set{\partial/\partial y^i|_{F(p)}} $ for $ T_pM $, the differential $ F_*: T_pN \to T_pM $ is represented by the matrix
	\[ [\partial F^i/\partial x^j(p)], \]
	where $ F^i = y^i \circ F $.
\end{proposition}
\begin{remark}
	We can state the inverse function theorem for manifolds \autoref{thm:InverseFunctionTheoremForManifolds} in a coordinate free description: A $ C^\infty $ function $ F:N\to M $ is locally invertible at point $ p\in N $ if and only if $ F_*: T_pN \to T_pM $ is an isomorphism of vector spaces.
\end{remark}


\section{Curves in a Manifold}
The notion of a curve in a manifold is a very important one as it will allow us to define the tangent space in an alternative way. We start by the definition of a curve in a manifold.

\begin{definition}[Curves in a Manifold]
	Let $ M $ be a manifolds and $ p \in M $. a \emph{curve} in manifold is a map
	\[ c: (a,b) \to M. \]
	Usually $ 0 \in (a,b) $ and we say $ c $ is a \emph{curve starting at $ p $} if $ c(0) = p $. The \emph{velocity vector} $ c'(t_0) $ of the curve at time $ t_0 \in (a,b) $ is defined to be
	\[ c'(t_0) = c_*(\frac{d}{dt}\big|_{t_0}) \in T_{c(t_0)}M. \]
\end{definition}

\begin{remark}
	Note that the velocity vector of the curve $ c $ at time $ t_0 $ denoted by $ c'(t_0) $ is a tangent vector at $ T_{c(p)}M $. This in particular can be a source of confusion when the target space of the curve is $ \R $. That is because in the calculus the notation $ c'(t) $ is the derivative of a real-valued function and is therefore a scalar. If it is necessary to distinguish between these two meanings of $ c'(t) $ when $ c $ maps into $ \R $, we will write $ \dot{c}(t) $ for the calculus derivative. For a clarification on this point visit \autoref{prob:VelocutyVecotrAndCalculusNotation}.
\end{remark}
\begin{remark}
	Alternative notations for $ c'(t_0) $ are
	\[ \frac{d}{dt}\big|_{t_0} c, \qquad \frac{dc}{dt}(t_0). \]
	However, I prefer the notation introduced in the definition box above as it has more emphasis on the fact that $ c'(t) $ is a vector in $ T_pM $.
\end{remark}

\begin{example}{Velocity vector of a curve is consistent with the results in calculus}
	\label{example:VelocityVectorWorkedExample}
	In this example we will demonstrate that the notion of the velocity vector of a curve is consistent with the similar notion in calculus. Define $ c:\R \to \R^2 $. 
	\[ c(t)  = (t^2, t^3). \]
	The velocity vector $ c'(t) $ is in tangent space $ T_{c(t)}\R^2 $. Let $ (\R^2, \mathbbm{1}_{\R^2}) $ be the atlas for manifold $ \R^2 $. Then we can write $ c'(t)$ in terms of the basis for tangent space.
	\[ c'(t) = a \frac{\partial}{\partial  x}\big|_{c(t)} + b \frac{\partial}{\partial  y}\big|_{c(t)}. \] 
	To determine the coefficients we apply the standard coordinate function $ x,y $ to both sides and we will get
	\[ c'(t)x = a, \qquad c'(t)y = b. \]
	From definition we have
	\[ c'(t) x = c_*(\frac{d}{dt}\big|_t) x = \frac{d}{dt}\big|_t (x\circ c) = \frac{d}{dt}\big|_t (t^2) = 2t.   \]
	Similarly we can get $ c'(t) y = 3t^2$. Thus
	\[ c'(t) = (2t) \frac{\partial}{\partial  x}\big|_{c(t)} + (3t^2) \frac{\partial}{\partial  y}\big|_{c(t)}. \]
\end{example}

The steps that we took in the example above motivates us for the following proposition.
\begin{proposition}[Velocity Vector in a Local Coordinate]
	\label{prop:VelocityVectorInLocalCorodinate}
	Let $ c:(a,b)\to M $ be a smooth curve, and let $ (U,x^1,\cdots,x^n) $ be a coordinate chart about $ c(t) $. Write $ c^i = x^i\circ c $ for the i-th component of $ c $ in the chart. Then $ c'(t) \in T_{c(t)}M $ is given by
	\[ c'(t) = \sum_i \dot{c}_i \frac{\partial}{\partial x^i}\big|_{c(t)}. \] 
	Thus relative to the bases $ \set{\partial /\partial x^i|_p} $ for $ T_{c(t)}M $ the velocity vector $ c'(t) $ can be written as
	\[ \begin{bmatrix}
		\dot{c}_1(t) \\
		\dot{c}_2(t) \\
		\vdots \\
		\dot{c}_n(t)
	\end{bmatrix} \]
\end{proposition}
\begin{proof}
	First, observe that by definition $ c'(t) \in T_{c(t)}M  $ thus can be written in the local coordinate  of $ T_{c(t)}M $ as
	\[ c'(t) = \sum_i a^i \frac{\partial}{\partial x^i}\big|_{c(t)}. \]
	Now we apply both sides to $ x_j $, and simply use the definition of the differential of a map and write
	\[x^j = c_*(\frac{d}{dt}\big|_{t}) x^j  =  \frac{d}{dt}\big|_{t}(x^j\circ c) = \dot{c}^j(t). \]
	This completes the proof.
\end{proof}

So far, we observed that every smooth curve $ c $ at $ p $ in a manifold $ M $ gives rise to a tangent vector in $ T_pM $. The following proposition states the converse of this, i.e. for every tangent vector in tangent space $ T_pM $ there exists a curve passing through $ p $.

\begin{proposition}[Existence of a Curve with a Given Initial Vector]
	For any point $ p $ in a manifold $ M $ and any tangent vector $ X_p \in T_pM $, there are $ \epsilon>0 $ and a smooth curve $ c:(-\epsilon,\epsilon) \to M $ such that $ c(0) = p $ and $ c'(0) = X_p $
\end{proposition}
\begin{proof}
	Let $ p \in M $ and $ X_p \in T_pM $.  Let $ (U,\phi = (x^1,\cdots,x^n)) $ be a coordinate chart centered at $ p $. The differential of $ \phi $ given by $ \phi_* $ maps $ X_p $ to $ Y_{\mathbf{0}} \in T_\mathbf{0}\R^n $. For instance, assume 
	\[ X_p = \sum_i a^i \frac{\partial}{\partial  x^i}\big|_{p}. \]
	Applying $ \phi_* $ to both sides and using \autoref{prop:DifferentialOfACoordinateMap} we will get
	\[ Y_\mathbf{0} =  \sum_i a^i \frac{\partial}{\partial  r^i}\big|_{\mathbf{0}}  \]
	where $ r^i $ is the standard coordinate function of $ \R^n $. Define the smooth curve $ \gamma: (-\epsilon, \epsilon) \to \R^n $ given by
	\[ \gamma(t) = (a^1t, \cdots, a^n t). \]
	For small enough $ \epsilon $ this curve is in $ \phi(U) $. We claim that the curve $ c:(-\epsilon,\epsilon)\to M $ given by
	\[ c = \inv{\phi} \circ \gamma \] 
	has the desired property. First, observe by definition we will have $ c(0) = p $
	\[ c(0) = \inv{\phi}(\gamma(0)) = p. \]
	For $ c'(0) $ we use the definition
	\[ c'(0) = c_*(d/dt|_{0}) = (\inv{\phi}\circ \gamma)_*(d/dt|_{0}) = (\inv{\phi}_*  \gamma_*)(d/dt|_{0})  \]
	It now remains to calculate the $ (\inv{\phi}_* \gamma_*) $. To calculate this, first observe that $ \gamma_* $ maps $ d/dt|_0 $ to $ Y_\mathbf{0} $. And also observe that $ \inv{\phi}_* $ maps $ X_p $ to $ Y_\mathbf{0} $. I.e.
	\[ Y_\mathbf{0} = \phi_* X_p, \qquad Y_\mathbf{0} = \gamma_* \frac{d}{dt}\big|_0. \]
	Then we will get
	\[ \phi_* X_p = \gamma_* \frac{d}{d  t}\big|_{0} \implies X_p = \inv{\phi}_* \gamma_* \frac{d}{dt}\big|_0. \]
	Thus we have
	\[ c'(0) =  \inv{\phi}_* \gamma_* \frac{d}{dt}\big|_0. = X_p\]
	and this completes the proof.
\end{proof}

Now using the notion of curves, we can calculate the differential of a map between manifolds via curves. The following proposition discusses this statement.
\begin{proposition}[Computing Differentials Using Curves]
	\label{prop:curvesToComputeDifferential}
	Let $ F: N  \to M$ be a smooth map between manifolds, and $ p \in N $ and $ X_p \in T_pN $. Let $ c $ be a curve on $ N $ such that $ c(0) = p $, and $ c'(0) = X_p $. Then
	\[ F_{*,p} X_p = (F\circ c)' (0). \]
\end{proposition}
\begin{proof}
	We simply follow the definition
	\[(F\circ c)'(0) = (F\circ c)_* (\frac{d}{dt}\big|_0) = F_* c_* (\frac{d}{dt}\big|_0) = F_* \left(  c_* (\frac{d}{dt}\big|_0) \right) = F_* X_p \]
	and this completes the proof.
\end{proof}


\section{Immersion and Submersion}
We start with the definitions of Immersion and Submersion.
\begin{definition}[Immersion and Submersion]
	Let $ F:N\to M $ be a smooth map, and let $ F_{*,p} $ denote its differential at a point $ p \in N $. Then 
	\begin{itemize}[noitemsep]
		\item $ F $ is an Immersion if and only if $ F_{*,p} $ is injective.
		\item $ F $ is a Submersion if and only if $ F_{*,p} $ is surjective.
	\end{itemize}
\end{definition}

Now we can define the notion of critical point and regular points of a mapping.
\begin{definition}[Critical and Regular points of a smooth map between manifolds]
	Let $ F:N \to M $ be a smooth map between manifolds. A point point $ p  \in N$ is
	\begin{itemize}[noitemsep]
		\item a \emph{critical point} if $ F_{*,p} $ fails to be surjective, and
		\item a \emph{regular point} if $ F_{*,p} $ is surjective.
	\end{itemize}
	A point $ q \in M $ is critical value of the map $ F $ if it is the image of a critical point.
\end{definition}

The following proposition gives a more intuitive characterization of the critical points of function defined from a manifold to the real number line.

\begin{proposition}[Critical points of a map to real number line]
	Let $ f: N \to \R $ be a smooth map from manifold $ N $ to the real number line. A point $ p \in N $ is a critical point of the map $ f $ in and only if all of its partial derivatives relative to some chart $ (U,\phi=(x^1,\cdots,x^n)) $ are zero at $ p $, i.e.
	\[ \frac{\partial }{\partial x^i}\big|_{p} f = 0 \qquad \text{for all } i = 1,\cdots, n. \]
\end{proposition}
\begin{proof}
	The differential of $ f $ in the local coordinates can be written as the matrix
	\[ f_{*,p} = [ \frac{\partial}{\partial  x^1}\big|_{p} f, \cdots, \frac{\partial}{\partial  x^n}\big|_{p} f ]. \]
	The point $ p $ is a critical point if and only if $ f_{*,p} $ is not surjective. Since the image of $ f_{*,p} $ is $ \R $, then its column space is either one dimensional or zero dimensional. Thus $ f_{*,p} $ is not surjective if and only if all of its entries are zero, i.e.
	\[ \frac{\partial }{\partial x^i}\big|_{p} f = 0 \qquad \text{for all } i = 1,\cdots, n. \]
\end{proof}


\newpage




\section{Summary}
\begin{summary}[Coordinate Map and Its Differential]
	Let $ M $ be a manifold and $ p \in M $. Let $ (U,\phi) $ be a coordinate chart containing $ p $. We saw in \autoref{propositio:coordinateMapsAreSmooth} that $ \phi $ is a diffeomorphism between $ U $ and $ \R^n $. In \autoref{cor:TangentSpaceIsoMorphismToRn} we also saw that $ T_p(U) $ is isomorphic to $ T_{\phi(p)}(\R^n) $. In a nutshell
	\begin{quote}
		Under a coordinate chart $ (U,\phi) $, $ U $ is diffeomorphic to $ \R^n $ and at a point $ p \in U $ the tangent space at $ p $, i.e. $ T_p(U) $, is isomorphic to the tangent space at $ \phi(p) $, i.e. $ T_\phi(p)_n $.
	\end{quote}
\end{summary}

\begin{summary}[Differential of a Map Between Euclidean Spaces]
	Let $ F: \R^n \to \R^m $. It differential at point $ p \in \R^n $ is its Jacobian matrix evaluated at point $ p $.
	\[ F_* = 
	\begin{pmatrix}
		\frac{\partial F^1}{\partial x^1} & \cdots & \frac{\partial F^1}{\partial x^n} \\
		\frac{\partial F^2}{\partial x^1} & \cdots & \frac{\partial F^2}{\partial x^n} \\
		\vdots & \ddots & \vdots \\
		\frac{\partial F^m}{\partial x^1} & \cdots & \frac{\partial F^m}{\partial x^n}
	\end{pmatrix}(p)
	 \]
\end{summary}

\begin{summary}{Differential of a coordinate map acting on a tangent vector}
	Let $ M $ be a manifold and $ (U,\phi = (x^1,\cdots,x^n)) $ a coordinate chart containing $ p \in M $. Then we have
	\[ \phi_*(\frac{\partial}{\partial  x^i}\big|_{p}) = \frac{\partial}{\partial  r^i}\big|_{\phi(p)} ,\]
	or equivalently
	\[ \inv{\phi_*}(\frac{\partial}{\partial  r^i}\big|_{\phi(p)}) = \frac{\partial}{\partial  x^i}\big|_{p}. \]
	To show these equation we use the definition of directional derivatives on a manifold (\autoref{def:DirectionalDerivativeOnManifold}). However, we can see that how these equations can reproduce the definition of the directional derivative on a manifold. Let $ f:M \to \R $
	\[ \frac{\partial}{\partial  x^i}\big|_{p} f = \inv{\phi_*}(\frac{\partial}{\partial  r^i}\big|_{\phi(p)})f = \frac{\partial}{\partial  r^i}\big|_{\phi(p)}(f\circ\inv{\phi}).  \]
	where we have used the definition of the differential of a map.
\end{summary}

\begin{summary}[Bases for Tangent Space at a Point on a Manifold]
	\label{summary:BasesFor TangentSpace}
	Let $ M $ be a manifold with $ p \in M $. $ T_pM $ is the the tangent space at point $ p $ that contains all of the point derivatives $ D: C_p^\infty \to \R$. For each coordinate chart containing $ p $ we can write a natural bases for $ T_pM $. For instance, let $ (U,\phi=(x^1,\cdots,x^n)) $ be a coordinate chart containing $ p $. Then the collection of tangent vectors
	\[ \set{ \frac{\partial}{\partial  x^1}\big|_{p}, \cdots, \frac{\partial}{\partial  x^n}\big|_{p} } \]
	is a basis for $ T_pM $. While choosing a different coordinate chart containing $ p $, e.x. $ (V,\psi=(y^1,\cdots,y^n)) $ we can write another natural basis for $ T_pM $
	\[ \set{\frac{\partial}{\partial  y^1}\big|_{p},\cdots,\frac{\partial}{\partial  y^n}\big|_{p}}. \]
	The change of basis matrix $ R $ maps these two basis to each other. This map is given by
	\[ 
	\begin{bmatrix}
		\partial/\partial x^1 \\
		\partial/\partial x^2 \\
		\vdots \\
		\partial/\partial x^n
	\end{bmatrix} = 
	\begin{pmatrix}
		\partial y^1/\partial x^1 & \partial y^1/\partial x^2 &  \cdots & \partial y^1/\partial x^n \\
		\partial y^2/\partial x^1 & \partial y^2/\partial x^2 &  \cdots & \partial y^2/\partial x^n \\
		\vdots & \vdots & \ddots & \vdots \\
		\partial y^n/\partial x^1 & \partial y^n/\partial x^2 &  \cdots & \partial y^n/\partial x^n
	\end{pmatrix}^T
	\begin{bmatrix}
		\partial/\partial y^1 \\
		\partial/\partial y^2 \\
		\vdots \\
		\partial/\partial y^n
	\end{bmatrix}
	 \]
\end{summary}

\begin{summary}[Velocity vector of a curve into $ \R^n $]K
	Let $ c:(a,b) \to \R^n $. Then is velocity vector at $ t_0 \in (a,b) $ is given by
	\[ 
	c'(t) = 
	\begin{bmatrix}
		\dot{c}_1(t_0) \\
		\dot{c}_2(t_0) \\
		\vdots \\
		\dot{c}_n(t_0)
	\end{bmatrix}
	 \]
	where $ c_1,\cdots,c_n $ are the components of the curve $ c $. See \autoref{example:VelocityVectorWorkedExample} for a worked example.
\end{summary}





\section{Solved Problems}
\begin{problem}[The differential of a map]
	\label{prob:DifferentialOfAMapIsADerivation}
	Check that $ F_*(X_p) $ is a derivation at $ F(p) $ and that $ F_*: T_p N \to T_{F(p)}M  $ is a linear map.
\end{problem}
\begin{solution}
	We start with showing that $ F_*(X_p) $ is a derivation at $ F(p) $. By definition we have
	\begin{align*}
		[(F_*(X_p))(f\cdot g)](F(p)) &=  [ X_p((f\cdot g)\circ F) ](p) = [ X_p((f\circ F) \cdot (g\circ F)) ](p)\\
		&= [X_p(f\circ F)](p) \cdot g(F(p)) + f(F(p))\cdot [X_p(g\circ F)](p)\\
		&= [(F_*(X_p))f](F(p)) \cdot g(F(p)) + f(F(p))\cdot [(F_*(X_p))g].
	\end{align*}
	This shows that $ F_*(p) $ is indeed a derivation at $ F(p) $. Furthermore, to prove the linearity, let $ X_p, Y_p \in T_p(N) $. Then
	\[ (F_*(X_p + Y_p))f = (X_p + Y_p)(f\circ F) = X_p(f\circ F) + Y_p(f\circ F) = (F_*(X_p)) f + (F_*(Y_p)) f. \]
\end{solution}


\begin{problem}[Velocity vector versus the calculus derivative (from L. Tu)]
	\label{prob:VelocutyVecotrAndCalculusNotation}
	Let $ c:(a,b) \to \R $ be a curve with target space $ \R $. Verify that $ c'(t) = \dot{c}(t) d/dx|_{c(t)} $.
\end{problem}
\begin{solution}
	Let $ f: \R \to \R $. From definition 
	\[ c'(t_0) f = c_*(\frac{d}{dt}\big|_{t_0})f = \frac{d}{dt}\big|_{t_0}(f\circ c) = \dot{c}(t_0) \frac{df}{dx}, \]
	where we have used the chain rule. Thus
	\[ c'(t_0) = \dot{c}(t) \frac{d}{dx}\big|_{c(t_0)}. \]
\end{solution}

\begin{problem}[Differential of left multiplication (from L. Tu)]
	If $ g $ is a matrix in the general linear group $ GL(n,\R) $, let $ \ell_g: GL(n,\R) \to GL(n,\R) $ be left 
	multiplication by $ g $. Thus $ \ell_g(B) = gB $ for any $ B \in GL(n,\R) $. Since $ GL(n,\R) $ is an open subset of the vector space $ \R^{n\times n} $, the tangent space $ T_g(GL(n,\R)) $ can be identified with $ \R^{n\times n} $. Show that with this identification the differential $ (\ell_g)_{*,I}: T_I(GL(n,\R)) \to T_g(GL(n,\R)) $ is also left multiplication by $ g $. 
\end{problem}
\begin{solution}
	We will use the idea of \autoref{prop:curvesToComputeDifferential} to compute the differential using a curve. Let $ c: (-\epsilon,\epsilon) \to GL(n,\R) $ be a curve such that $ c(0) = I $ and $ c'(0) = X_p $. Then 
	\[ (\ell_g)_{*,I} X_p = (\ell_g\circ c)'(0) = (gc)'(0) = g c'(0) = gX_p. \]
\end{solution}

\begin{problem}[Differential of a map (from L. Tu.)]
	Let $ F: \R^2 \to \R^3 $ be the map
	\[ (u,v,w) = F(x,y) = (x,y,xy). \]
	Let $ p = (x,y) \in \R^2 $. Compute $ F_*(\partial/\partial x|_p) $ as a linear combination of $ \partial/\partial u, \partial/\partial v $, and $ \partial/\partial w $ at $ F(p) $
\end{problem}
\begin{solution}
	We know that $ F_*(\partial/\partial x|_{p} ) \in T_{F(p)}\R^3 $. Letting $ (\R^3,\mathbbm{1}_{\R^3} = (u,v,w)) $ be an atlas for $ \R^3 $, we will have
	\[ F_*(\partial/\partial x|_{p} ) = a \partial/\partial u|_{F(p)} + b\partial/\partial v|_{F(p)} + c\partial/\partial w|_{F(p)}.    \]
	Applying $ u,v,w $ to both sides we will get
	\[ a = F_*(\partial/\partial x|_{p} ) u,\qquad b = F_*(\partial/\partial x|_{p} ) v, \qquad c = F_*(\partial/\partial x|_{p} ) w. \]
	Thus 
	\[ a = 1,\qquad b= 0, \qquad c = 1. \]
	
\end{solution}

\begin{problem}[Differential of a Linear Map]
	Let $ L:\R^n \to \R^m $ be a linear map. For any $ p \in \R^n $, there is a canonical identification of $ T_p(\R^n) $ to $ \R^n $ give by
	\[ \sum a^i \frac{\partial}{\partial  x^i}\big|_{p} \mapsto a = \< a^1,\cdots,a^n \>. \]
	Show that the differential $ L_{*,p}:T_p(\R^n) \to T_{L(p)}\R^m $ is the map $ L:\R^n \to \R^m $ itself with the identification of the tangent spaces as above.
\end{problem}
\begin{solution}
	{\color{red} \noindent TODO: Answer to be added.}
\end{solution}

\begin{problem}[Differential of a Map (from W. Tu)]
	Fix a real number $ \alpha $ and define $ F: \R^2 \to \R^2 $ by 
	\[ 
	\vectt{u}{v} = (u,v) = F(x,y) = \matt{\cos\alpha}{-\sin\alpha}{\sin\alpha}{\cos\alpha}\vectt{x}{y}.
	 \]
	 Let $ X = -y\partial/\partial x + x\partial/\partial y $ be a vector field on $ \R^2 $. If $ p = (x,y) \in \R^2 $ and $ F_*(X_p) = (a\partial/\partial u + b\partial/\partial v)|_{F(p)}  $, find $ a $ and $ b $ in terms of $ x,y $, and $ \alpha $.
\end{problem}
\begin{solution}
	We start with 
	\[ F_*(X_p) = (a\partial/\partial u + b\partial/\partial v)|_{F(p)}.  \]
	Applying $ u $ and $ v $ to both sides we will get
	\[ a = F_*(X_p) u, \qquad b = F_*(X_p) v. \]
	To compute $ a $ we will have
	\[ a = F_*(X_p)u = X_p (u\circ F) = (-y\frac{\partial}{\partial  x}\big|_{p} + x \frac{\partial}{\partial  x}\big|_{p})(x\cos\alpha - y \sin\alpha) = -y\cos\alpha -x\sin\alpha. \]
	Similarly for $ b $ we can write
	\[ b = F_*(X_p)v = X_p (v\circ F) = (-y\frac{\partial}{\partial  x}\big|_{p} + x \frac{\partial}{\partial  x}\big|_{p})(x\sin\alpha + y \cos\alpha) = -y\sin\alpha +x\cos\alpha. \]
\end{solution}

\begin{problem}[Transition matrix for coordinate vectors (from L. Tu)]
	Let $ x,y $ be the standard coordinates on $ \R^2 $, and let $ U $ be the open set
	\[ U = \R^2 - \set{(x,0)| x\geq 0}.\]
	On $ U $ the polar coordinates $ r,\theta $ are uniquely defined by
	\[ x = r\cos\theta, \qquad y = r\sin\theta \]
	where $ r>0 $ and $ 0<\theta<2\pi $.
	Find $ \partial/\partial r $ and $ \partial/\partial \theta $ in terms of $ \partial/\partial x $ and $ \partial/\partial y $.
\end{problem}
\begin{solution}
	We need to find the change of basis matrix. We can either use the idea of \autoref{summary:BasesFor TangentSpace} or do the computations manually. We will do the latter. We expand $ \partial/\partial r $
	\[ \partial/\partial r = a \partial/\partial x + b \partial/\partial y. \]
	Applying both sides to $ x $ and $ y $ respectively we will get
	\[ a = \partial x/\partial r = \cos\theta,  \qquad b = \partial y/\partial r = \sin\theta. \]
	Similarly for $ \partial/\partial \theta $ we can write
	\[ \partial/\partial \theta = c \partial/\partial x + d \partial/\partial y. \]
	Applying both sides to $ x $ and $ y $ respectively we will get
	\[ c = \partial x/\partial \theta = -r\sin\theta, \qquad \partial y/\partial \theta =r\cos\theta. \]
\end{solution}

\begin{problem}[Velocity vector (from L. Tu)]
	Let $ p = (x,y) $ be a point in $ \R^2 $. Then 
	\[ c_p(t) = \matt{\cos 2t}{-\sin 2t}{\sin 2t}{\cos 2t}\vectt{x}{y}, \qquad t \in \R,\]
	is a curve with initial point $ p \in \R^2 $. Compute the velocity vector $ c_p'(0) $.
\end{problem}
\begin{solution}
	We can either use the result of the \autoref{prop:VelocityVectorInLocalCorodinate} or we can do a manual calculation, which we will do the latter. Since $ c_p'(t) \in T_p(\R^2) $, then we can expand it in the bases induced by the local coordinates. 
	\[ c_p'(t) = c_{*,p}(\frac{\partial}{\partial  t }\big|_{0})  = a \frac{\partial}{\partial  x}\big|_{p} + b \frac{\partial}{\partial  y }\big|_{p}.\]
	Now applying $ x $ to both sides we will get
	\[ a = c_{*,p}(\frac{\partial}{\partial  t }\big|_{0})x = \frac{\partial}{\partial  t }\big|_{0} (x\circ c_p(t)) = -2 x\sin 2t  - 2y\cos 2t.  \]
	Similarly, by applying $ y $ to both sides we will get
	\[ b = c_{*,p} (\frac{\partial}{\partial  t }\big|_{0}) y = \frac{\partial}{\partial  t }\big|_{0} (y\circ c_p(t)) = 2x\cos 2t - 2y\sin 2t.\]
\end{solution}

\begin{problem}[Tangent Space to a Product]
	If $ M $ and $ N $ are manifolds, let $ \pi_1: M\times N \to M $ and $ \pi_2: M\times N \to N $ be the two projections. Prove that for $ (p,q) \in M\times N $,
	\[ (\pi_{1*},\pi_{2*}) : T_{(p,q)}(M\times N) \to T_p M \times T_q N \]
	is an isomorphism.
\end{problem}
\begin{solution}
	Consider the map
	\[ (\pi_1,\pi_2) : M\times N \to M\times N. \]
	This is indeed a diffeomorphism, and in fact $ (\pi_1,\pi_2) = \mathbbm{1}_{M\times N} $. This implies that $ (\pi_{1*},\pi_{2*}) $ is an isomorphism of vector spaces.
\end{solution}
\begin{carefull}
	I am not sure if my question above is a correct one. However, I have provided a more direct proof below.
\end{carefull}
\textbf{Second solution:} We will show this by explicitly constructing a basis for $ T_{(p,q)}(M\times N) $ as well as $ T_pM \times T_qN $ and we will show that the map $ (\pi_{1*},\pi_{2*}) $ caries the bases to bases, hence is an isomorphism of vector spaces. 

Let $ (U,\phi=(x^1,\cdots,x^n)) $ be a coordinate chart containing $ p \in M$ and $ (V,\psi=(y^1,\cdots,y^n)) $ be a coordinate chart containing $ q \in N $. Then a coordinate chart for $ M\times N $ that contains $ (p,q) $ would be $ (U\times V, (\phi\circ\pi_1, \psi\circ\pi_2) = (\bar{x}^1,\cdots,\bar{x}^n,\bar{y}^1,\cdots,\bar{y}^m)) $. Observe that 
\[ \pi_{1*}(\frac{\partial}{\partial  \bar{x}^i}\big|_{(p,q)}) = a^1_i \frac{\partial}{\partial  x^1}\big|_{p} + \cdots + a^n_i \frac{\partial}{\partial  x^n}\big|_{p}.  \]
By applying $ x^j $ to both sides we get
\[ a^j_i = \pi_{1*}(\frac{\partial}{\partial  \bar{x}^i}\big|_{(p,q)}) x^j = \frac{\partial}{\partial  \bar{x}^i}\big|_{(p,q)}(x^i\circ \pi_1) = \delta_{ij}. \] 
Thus we can write
\[  \pi_{1*}(\frac{\partial}{\partial  \bar{x}^i}\big|_{(p,q)}) = \frac{\partial}{\partial  x^i}\big|_{p}, \]
and by a similar calculation for $ \pi_{2*} $ we get
\[  \pi_{1*}(\frac{\partial}{\partial  \bar{y}^i}\big|_{(p,q)}) = \frac{\partial}{\partial  y^i}\big|_{p}. \]
On the other hand, observe that a basis for $ T_{(p,q)}(M\times N) $ is 
\[ \set{\frac{\partial}{\partial  \bar{x}^1}\big|_{(p,q)},\cdots,\frac{\partial}{\partial  \bar{x}^n}\big|_{(p,q)},\frac{\partial}{\partial  \bar{y}^1}\big|_{(p,q)},\cdots,\frac{\partial}{\partial  \bar{y}^m}\big|_{(p,q)}} \]
and a basis for $ T_p(M)\times T_q(N) $ is 
\[ \set{(\frac{\partial}{\partial  x^1}\big|_{p},0),\cdots,(0,\frac{\partial}{\partial  x^n}\big|_{p},0),\frac{\partial}{\partial  y^1}\big|_{q}, (0,\frac{\partial}{\partial  y^m}\big|_{q})} \]
We can now see that the map $ (\pi_{1*},\pi_{2*}) $ maps these two bases to each other, thus it is an isomorphism of vector spaces.

\begin{problem}[Differentials of multiplication and inverse (from L. Tu)]
	Let $ G $ be a Lie group with multiplication map $ \mu:G\times G \to G $, inverse map $ \iota: G \to G $, and identity element $ e $. 
	\begin{enumerate}[(a)]
		\item Show that the differential at the identity of the multiplication map $ \mu $ is addition:
		\begin{align*}
			\mu_{*,(e,e)}: T_eG \times T_eG \to T_e G, \qquad \mu_{*,(e,e)}(X_e,Y_e) = X_e + Y_e
		\end{align*}
		\item Show that the differential at the identity of $ \iota $ is the negative
		\[ \iota_{*,e}: T_eG \to T_eG, \qquad \iota_{*,e}(X_e) = -X_e. \]
	\end{enumerate}
\end{problem}
\begin{solution}
	$ \ $\\
	\begin{enumerate}[(a)]
		\item We will use the idea of \autoref{prop:curvesToComputeDifferential} to compute the differential. Let $ c:(-\epsilon,\epsilon) \to G\times G $ for some $ \epsilon>0 $, where $ c(t) = (g_1(t),g(t)) $ such that we have $ x(0) = (e,e) $ and $ c'(0)=(X_e,Y_e)$. Then we can write
		\[ \mu_{*,(e,e)} (X_e,Y_e) = (\mu \circ c)'(0) = (g_1g_2)'(0) = g_1'(0)g_2(0) + g_1(0)g_2'(0) = X_e + Y_e, \]
		where we have used the fact that $ (\cdot)' $ is a derivation.
		\item We again start with the idea of \autoref{prop:curvesToComputeDifferential}. Let $ c:(-\epsilon,\epsilon) \to G $ for some $ \epsilon>0 $ such that $ c(0) = e $ and $ c'(0) = X_e $. Thus we can write
		\[ \iota_{*,e} X_e = (\iota \circ c)'(0). \]
		Now we aim at finding $ (\iota \circ c)'(0) $ by using the identity
		\[ \mu(c(t), (\iota \circ c)(t)) = e, \quad \text{for $ t\in (-\epsilon,\epsilon) $}. \]
		Since the function above (which is from $ \R $ to $ G $) is constant, thus its derivation will be zero. I.e.
		\[ (c \cdot (\iota\circ c))(0) = c'(0)\underbrace{\iota(c(0))}_{e} + \underbrace{c(0)}_e(\iota\circ c)'(0) = 0 \]
		where we have used the fact that $ \iota(e) = e $ and $ (\cdot)' $ is a derivation. This implies that 
		\[ (\iota \circ c)'(0) = -X_p. \]
	\end{enumerate}
\end{solution}

\begin{problem}[Transforming vectors to coordinate vectors]
	Let $ X_1,\cdots,X_n $ be $ n $ vector fields on an open subset $ U $ of a manifold of dimension $ n $. Suppose that at $ p \in U $, the vectors $ (X_1)_p,\cdots,(X_n)_p $ are linearly independent. Show that there is a chart $ (V,x^1,\cdots,x^n) $ about $ p $ such that $ (X_i)_p = (\partial/\partial x^i)_p $ for $ i = 1,\cdots,n $.
\end{problem}
\begin{solution}
	Let $ (V,y^1,\cdots,y^n) $ be a coordinate chart containing $ p $. We expand the tangent vectors $ (X_1)_p,\cdots,(X_n)_p $ in terms of the bases $ \partial/\partial y^1|_{p},\cdots,\partial/\partial y^n|_{p} $ and arrange the coordinates as the columns of a matrix $ P $. Symbolically we can write
	\[ \left[(X_1)_p\ (X_2)_p\ \cdots\ (X_n)_p\right] = \left[\partial/\partial y^1|_{p}\ \partial/\partial y^2|_{p}\ \cdots\ \partial/\partial y^n|_{p}\right] P.
	 \]
	Since $ (X_i)_p $ are linearly independent, so $ P $ is non singular and is invertible. Thus we can write
	\[ \left[ \partial/\partial y^1|_{p}\ \partial/\partial y^2|_{p} \cdots\ \partial/\partial y^n|_{p} \right] = \left[ (X_1)_p\ (X_2)_p\ \cdots\ (X_n)_p \right] \inv{P}. \]
	Define
	\[ x^j = \sum_i\ [\inv{P}]_{ji}\ y^i. \]
	Since $ \inv{P} $ is non singular and $ (y^1,\cdots,y^n) $ is a coordinate maps, this implies that $ (x^1,\cdots,x^n) $ is also a coordinate map, thus $ (V,(x^1,\cdots,x^n)) $ is also a chart. From the change of bases rule for the differential in charts we know that
	\[ \partial/\partial y_i = \sum_j \frac{\partial x_j}{\partial y_i} \frac{\partial}{\partial x_j} \]
	where from the definition $ x^j $
	\[ \partial/\partial y^i = \sum_j \left[ \inv{P} \right]_{ji} \frac{\partial }{\partial x^j}. \]
	Symbolically we can write this as
	\[ [ \partial/\partial y^1\ \partial/\partial y^2\ \cdots\ \partial/\partial y^n ] = [\partial/\partial x^1\ \partial/\partial x^2\ \cdots\ \partial/\partial x^n] \inv{P}. \]
	By multiplying both sides as $ P $ we will get
	\[ [\partial/\partial x^1\ \partial/\partial x^2\ \cdots\ \partial/\partial x^n] =  [ \partial/\partial y^1\ \partial/\partial y^2\ \cdots\ \partial/\partial y^n ]P  \]
	Thus we can conclude that 
	\[ [\partial/\partial x^1\ \partial/\partial x^2\ \cdots\ \partial/\partial x^n] = \left[(X_1)_p\ (X_2)_p\ \cdots\ (X_n)_p\right]. \]
\end{solution}


\begin{problem}[Local Maxima (form L. Tu)]
	A real valued function $ f:M\to \R $ on a manifold is said to have a local maximum at $ p \in M $ if there is a neighborhood $ U $ of $ p $ such that $ f(p) \geq f(q) $ for all $ q \in U $.
	\begin{enumerate}[(a)]
		\item Prove that if a differentiable function $ f:I\to \R $ defined on an open interval $ I $ has a local maximum at $ p \in I $ then $ f'(p) = 0 $.
		\item Prove that a local maximum of a $ C^\infty $ function $ f:M\to \R $ is a critical point of $ f $.
	\end{enumerate}
\end{problem}
\begin{solution}
	$\ $
	\begin{enumerate}[(a)]
		\item Let $ f $ has a local maximum at $ p \in I $. Thus $ f(p) \leq f(x) $ for all $ x $ sufficiently close to $ p $ from left. Thus
		\[ f'(p) = \lim_{x \to p^-} \frac{f(x) - f(p)}{x-p} \leq 0. \]
		On the other hand for all $ x $ sufficiently close to $ p $ from left we have $ f(p) \geq f(x) $. This implies that 
		\[ f'(p) = \lim_{x \to p^+} \frac{f(x) - f(p)}{x-p} \geq 0. \]
		Combining the two inequalities above we will get
		\[ f'(p) = 0. \]
		\item Let $ p \in M $ be a local maximum of a function $ f $. Let $ X_p \in T_pM $ be any tangent vectors, and $ c $ be a smooth curve with velocity vector $ X_p $ that starts from $ p $. Then 
		\[ X_p f = (f\circ c)'(0). \]
		Note that $ f\circ c $ is a real function from $ \R\to \R $ that has a local maxima at $ t = 0 $, thus $ (f\circ c)'(0) = 0 $. Let $ (U,\phi = (x^1,\cdots,x^n)) $ be any coordinate chart. From the fact above we know that 
		\[ \frac{\partial}{\partial  x^i}\big|_{p} f = 0 \qquad \text{for $ i = 1,2,\cdots,n $}. \]
		So by \autoref{prop:CriticalPointOfARealMap} we conclude that $ p $ is a critical point.
	\end{enumerate}
\end{solution}


\begin{problem}[Maxima Integral Curve (from L. Tu)]
	Let $ X = x^2 d/dx $ be a vector field on the real number line $ \R $. Fine maximal integral curve of $ X $ starting at $ x =2 $.
\end{problem}
\begin{solution}
	Let $ c:(a,b) \to \R $ given as $ c(t) = x(t) $ such curve such that $ c(0) = 2 $. By \autoref{prop:curvesToComputeDifferential} we can write
	\[ c'(t) = \dot{c}(t) \frac{d}{dt} = \dot{x} \frac{d}{dx}. \]
	For $ c $ to an integral curve we need to have
	\[ c'(t) = X|_{c(t)} \qquad \text{for } t\in (a,b).  \]
	Thus we will have
	\[ \dot{x} = x^2, \qquad x(0) = 2. \]
	By solving this initial value problem with separation of variables method, we will get
	\[ x(t) = \frac{-2}{2t+1}. \]
	The maximal domain for this curve is $ t \in (-\infty, -1/2) $.
\end{solution}


\begin{problem}[Recovering vector field from a global flow (from L. Tu)]
	Let $ F: \R\times \R^2 \to \R^2 $ be a global flow on $ \R^2 $ give by
	\[ F\left( t, \vectt{x}{y} \right) = \matt{\cos t}{-\sin t}{\sin t}{\cos t} \vectt{x}{y}. \]
	Recover the vector field that generates this global flow.
\end{problem}
\begin{solution}
	Since $ F $ is a global flow, then for any given $ p = \vectt{x}{y} \in \R^2 $, $ F(t,p) $ will be an integral curve starting at $ p $. Thus its velocity vector at $ p $ will give the vector field at $ p $. The velocity vector of $ F $ at $ p $ is given by
	\[ X_p = \frac{\partial F }{\partial t}\big|_{0}(0,p) = \matt{-\sin t}{-\cos t}{\cos t}{-\sin t}\big|_{0} \vectt{x}{y} = -y\frac{\partial }{\partial x} + x \frac{\partial }{\partial y}. \]
	Thus the generating vector field is given as
	\[ X = -y \frac{\partial }{\partial x } + x \frac{\partial }{\partial y}. \]
\end{solution}