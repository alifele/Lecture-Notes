\section{Logic}

In the past few weeks, I have been studying mathematical logic from the book  "A first course in logic" by Hedman and mathematical proof theory by from the book "PLP: An introduction to mathematical proof". That's why I had some time to have some deep thoughts on the concept of logic and its role in our today world. Here in this text I will write some of my thoughts.

\subsection{Logic and Real World Applications}

It is obvious that we are living with logic and using it rules very intuitive in our every day life and I am dare enough to claim that we are using logic more than any other tool in our life time! However, when it comes to the mathematical logic and the discussions like "propositional logic", "first-order logic", "second-order logic", and etc, one might feel that these concepts are so abstract to be used directly in a every day life. Specially, when one observe some stuff that do not make sense with our daily experience in using the intuitive logic. the "implication" operator that is represented by "$ \rightarrow $" symbol is a very good example. We tend to use this operator in our daily life when there is a causality between two things. For example:


\begin{equation*}
	\text{it is raining outside} \rightarrow \text{the grasses are wet}
\end{equation*}

But in symbolic logic (mathematical logic), the implication operator has the following truth table:

\begin{table}[h!]
	\centering
	\begin{tabular}{|c|c|c|}
		\hline
		A & B & $A \rightarrow B$ \\ \hline
		0 & 0 & 1                 \\ \hline
		0 & 1 & 1                 \\ \hline
		1 & 0 & 0                 \\ \hline
		1 & 1 & 1                 \\ \hline
	\end{tabular}
	\caption{Truth table for the implication}
	\label{tab:truthTable}
\end{table}


The second row of the table above do not make sense at all! The second row simply means:


\begin{equation*}
	\text{Aristotle is woman} \rightarrow \text{2 equals 2}
\end{equation*}

This sentence is logically true! In the mathematical logic the statement "a false statement implies true statement" is true. So when one studies logic and observes this kind of contradictions with common sense, then he might think that logic is useful only when we want to proof something with a mathematical rigor. But the truth is that without mathematical logic, the device that you are using to read this text would disappear suddenly! The whole foundation of computer science and digital electronics is mathematical logic.\\d

On in his research life can easily come up with ideas that are not consistent with the common sense (not in a terrible way) which might force that person might abandon that idea completely. But my observation is that such ideas can be developed with enough care and wisdom which has happened in the case of relation between computer science and mathematical logic. Think about yourself! When you grab a logic book and study, there are multiple possible scenarios but I think the most probable one is to abandon the subject altogether after seeing some contradictions with common sense. Or in the best scenario, you might accept those kinds of "contradictions with common sense" and follow the text and do all of exercises and proofs and etc. \\

The logic is quite hard to understand in the first try (because it can be against our daily intuition). But after getting used to it we might be comfortable with it and remain in the small scale logic evaluating and understanding the small statements. At this point, when you consider that the foundation of a computer is logic, then you might feel that building a computer out of true/false statements (that some of them are very challenging even in small scales) might be EXTREMELY hard. But the truth is that adding a little bit of engineering flavor to logic can tunnel you from small scale logic to a computer! \\


The whole magic happens with the concept of \textbf{"level of abstraction"} and \textbf{"consistent visualization system"}. The second one might seem trivial but it is not like that at all. Few days ago I was trying to build a visualization system for the computing graph (that can represent the complex arthritics in form of computing graphs) and I found out that maintain the right amount of detail and complexity can be a very challenging task. \\


It would be a very good experience to evaluate the two concepts in the development of processors out of logic. So let's start with the consistent visualization system.

\subsection{Consistent Visualization for Digital Systems }






