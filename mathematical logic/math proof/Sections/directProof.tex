\section{Direct Proof}
It is often the case in the mathematical proof that we start with assuming the hypothesis of an implication is true and conclude with the truth of the conclusion of the implication. The general structure of those proves will be all similar. Suppose that we want to prove that the implication 
\[ P \imply Q \]
is true. When know that if $ P $ is false, then the statement will automatically be true. But if $ P $ is true, then the implication will be true only when the $ Q $ is true. So if we start by assuming that $ P $ is true and \emph{through linking smaller true implications} we arrive at the conclusion, then by using modus ponens we can infer that conclusion of the original implication is also true. In other words, to prove $ P \imply Q $ is true, we need to construct the following \textbf{true} implications:
\begin{align*}
	P &\imply P_1 \quad \text{is true} \\
	P_1 &\imply P_2 \quad \text{is true} \\
	P_2 &\imply P_3 \quad \text{is true} \\
	 \vdots & \\
	P_n &\imply Q \quad \text{is true}
\end{align*}

Then we can connect these statements with AND operator and have:
\begin{equation}\label{equ:babyImpliactionsImplyBigImplication}
	(P \imply P_1) \wedge (P_1 \imply P_2) \wedge \cdots \wedge (P_n \imply Q)) \imply (P \imply Q)
\end{equation}


Here at this point there are two ways to proceed with the proof. In the following section I'll discuss both of them:

\textbf{The first method to proceed with the proof:} In this point of view, we assume that the statement $ P $ is true. Since the first baby implication is also true, then using Modus Ponens we can infer that $ P_1 $ is also true. Then since $ P_1 $ is true and the second baby implication is also true, then using Modus Ponens we can infer that $ P_2 $ is also true. Using the same logic we can infer that $ P_3, P_4, \cdots P_n $ are also true. And for the last time, since the last baby statement is true and $ P_n $ is also true, then we can infer (using Modus Ponens) $ Q $ is also true. So we started with the true hypothesis and we arrived at the true conclusion. So utilizing the truth table of implication we can infer that the implication $ P \imply Q $ is true. 

\textbf{The second method to proceed with the proof:} The golden equation in utilizing this method is \ref{equ:babyImpliactionsImplyBigImplication}. From now on, when I mention RHS and LHS, I mean the right hand side and the left hand side of the equation \ref{equ:babyImpliactionsImplyBigImplication}. Since all of the statements in the LHS is true, then the whole statement in the LHS is also true. And since the statement LHS $ \imply $ RHS is also true (check out the equation \ref{equ:babyImplicationImplyBigImplicationProof}), then we can infer (modus ponens) that the LHS of the implication (i.e. $ P \imply Q $) is also true. So if P is true then Q is also true.

These two methods are not really two different methods. These are two different methods to think about the same thing are are actually two sides of a single coin!

\begin{example}{Is $ n^2 $ Even or Odd?}
	\textbf{Question.} Let $ n $ be an integer. prove that if $ n $ is even, then $ n^2 $ is also even. \\

	\emph{Scratch Paper Stuff.} We are going to use direct proof method. So let's construct the first baby implication. let the statement $ P $ be:
	\[ P \text{: integer n is even} \]
	and let the statement $ P_1 $ be:
	\[ P_1 \text{: n can be written as } n=2k, k \in \mathbb{Z} \]
	From the definition of even numbers we know that the implication $ P \imply P_1 $ is true. Now lets construct the second baby implication. Let the statement $ P_2 $ be:
	\[ P_2 \text{: the square root of } 2k \text{ is } 4k^2  \]
	From the properties of the multiplication of integers we know that if $ a=b $ then $ ac = bc $ for any integer $ c $ and the since the $ a=b=c=2 $ is an special case of this property, then $ P_1 \imply P_2 $ is also true. Let's construct the third baby implication. Let the statement $ P_3 $ be:
	\[ P_3 \text{: the integer } 4k^2 \text{ can be written as } 2(2k^2)\]
	Considering the associativity property of the multiplication (i.e. $ (ab)c = a(bc) $) then we know that the statement $ P_2 \imply P_3 $ is also true.
	And finally, we can utilize the definition of the even number once again and construct the following final baby implication step. Let $ Q $ be the statement
	\[ Q: n=2(2k^2) \text{ is an even number} \]
	Because of the definition of the even number we know that $ P_3 \imply P_4 $ is also true. So in a nutshell we have:
	\begin{align*}
		P &\imply P_1 \text{ is true (definition of even number)} \\
		P_1 &\imply P_2 \text{ is true (property of integers)} \\ 
		P_2 &\imply P_3 \text{ is true (associativity property of integers)} \\
		P_4 &\imply Q \text{ is true (definition of even number)}
	\end{align*}

	By connecting these statements to each other with AND operator, we can write:
	\[ ((P \imply P_1 ) \wedge (P_1 \imply P_2) \wedge (P_2 \imply P_3) \wedge (P_4 \imply Q)) \imply (P \imply Q) \]
	Since all of the terms on the LHS is true, and the whole statement is also true (proved in \ref{equ:babyImplicationImplyBigImplicationProof}) we can infer (modus ponens) that the right hand side is also true. We need a little extra step here to complete the proof: Since the statement $ P $ is true, and the statement $ P \imply Q $ is also true, then we can infer (Modus Ponens) that $ Q $ is also true. \qed \\
	
	\emph{Clean Proof.} Assume $ n \in \mathbb{Z} $ is even. Using the definition of even numbers, we can write n as $ n=2k $ for some integer k. Now squaring the both sides of the equation and factoring 4 we can write: $ n^2 = 4k^2 = 2(2k^2) $. Since $ 2k^2 \in \mathbb{Z} $, it follows from the definition that $ n^2 $ is even. \qed
	
\end{example}

You might be wondering what is the square block (i.e. \qedsymbol) at the end of each proof. This is called ``QED'' which stands for ``quod erat demonstrandum'' which means `` which was to be demonstrated''. There are also other symbols in common use for this purpose that $ \blacklozenge, \blacksquare $ are the most common alternatives.


\subsection{Proof of Inequalities}
Although there are not such a thing as a general recipe for mathematical proof. But we can add some kind of tricks to our tool kit. Here in this section I will describe one of them that might come very handy in dealing with inequalities. To better demonstrate the idea, I would like to proceed with an example.

\begin{example}{Tricks in Working with Inequalities}
	\textbf{Question.} Let $ x,y \in \mathbb{R}$. Then prove $x^2 + y^2 \geq 2xy$ \\
	
	\underline{\emph{Scratch Paper Proof.}} When dealing with inequalities, it is better by studying a difference derived from the inequality. Let's start with studying the expression $ x^2 + y^2 - 2xy $. As soon as we start dealing with this, we observe that we can write it as $ (x-y)^2 $. Since $ x-y \in \mathbb{R} $ then we know that $ (x-y)^2 >0 $. So we got the intuition that if we start with the difference of real numbers $ x,y $, we can square it and then utilize the fact that the square should be positive and then arrive at the conclusion. Let's put these insights into good words and have more cleaner version of proof that has the appropriate flow of logic. \qed \\
	
	\proof{Proof.} Let $ x,y $ be real numbers. Hence $ (x-y) \in \mathbb{R} $ and we can write $ (x-y)^2 \geq 0 $. Expanding this will give: $ (x-y)^2 = x^2 + y^2 -2xy \geq 0 $. This can be rewritten as: $ x^2 + y^2 \geq 2xy $. \qed
	
\end{example}

So the trick in a nutshell is: To prove and inequality, start with studying a difference (maybe the RHS-LHS). This trick is far from being a general recipe for mathematical proof. There any MANY MANY MANY occasions that this will not work at all. So do not over use any trick in your toolbox.

\begin{example}{Multiplying Inequalities}
	\textbf{Question.} Let $ a,b,c,d \in \mathbb{R} $ and $ a > b > 0 $ and $ c > d > 0 $. Prove $ ac > bd $. \\
	
	\proof{Proof.} Since $ c > 0 $ and $ b > 0 $ then we can multiply $ c $ at the first inequality and $ b $ at the second inequality without changing the direction of the inequality sign. So we will have:
	\begin{align*}
		ac > bc > 0 \\
		bc > bd > 0
	\end{align*}
	So by combining these two inequalities, we can have:
	\[ ac > bd \]
	
	\proof{Scratch Paper Proof.} There is a reason that I am writing the scratch paper proof after the main proof. That is to show that how this kind of proof matches with the logical baby steps. Let the statements $ P_1, P_2 $ be defined as:
	\begin{quote}
		\centering
		$ P_1: a>b>0 $ \\
		$ P_2: c>d>0 $
	\end{quote}
	Let's define the statements $ P $ as an auxiliary statement in the following way:
	\begin{quote}
		\centering
		$ P = P_1 \wedge P_2 $
	\end{quote} 
	utilizing the properties of the inequalities of real numbers (e.g. multiplying a negative number at both sides change the direction of inequality while multiplying a positive number do not change the direction) we can write the following \textbf{true} statements. 
	\begin{quote}
		\centering
		$ P \imply Q_1 $ \\
		$ P \imply Q_2 $ 
	\end{quote}
	in which $ Q_1, Q_2 $ are:
	\begin{quote}
		\centering
		$ Q_1: ac > bc $ \\
		$ Q_2: bc > bd $
	\end{quote}
	Since (trivially!) $ bc=bc $ so we can express the following \textbf{true} statement:
	\begin{quote}
		\centering
		$ Q_1 \wedge Q_2 \imply R $
	\end{quote}
	in which $ R $ is:
	\begin{quote}
		\centering
		$ R: ac>bd $
	\end{quote}
	Note that we have not yet shown that $ P = P_1 \wedge P_2 $ implies Q (although it might trivially make sense). To arrive at the $ P \imply Q $ from the baby steps, implicitly we are using the following tautology:
	\begin{quote}
		\centering
		$ ((P \imply Q_1) \wedge (P \imply Q_2) \wedge (Q_1 \wedge Q_2 \imply R)) \imply (P \imply R) $
	\end{quote}
	This is a tautology because its truth value is always true:
	\begin{center}
		\begin{tabular}{|c|c|c|c|c|c|c|c|c|c|}
			\hline
			P & Q & R & S & $ P \imply R $ & $ P \imply Q $ & $ R \wedge Q \imply S $ & LHS & RHS & LHS $ \imply $ RHS \\
			\hline
			0 & 0 & 0 & 0 & 1 & 1 & 1 & 1 & 1 & 1 \\
			\hline
			0 & 0 & 0 & 1 & 1 & 1 & 1 & 1 & 1 & 1 \\
			\hline
			0 & 0 & 1 & 0 & 1 & 1 & 1 & 1 & 1 & 1 \\
			\hline
			0 & 0 & 1 & 1 & 1 & 1 & 1 & 1 & 1 & 1 \\
			\hline
			0 & 1 & 0 & 0 & 1 & 1 & 1 & 1 & 1 & 1 \\
			\hline
			0 & 1 & 0 & 1 & 1 & 1 & 1 & 1 & 1 & 1 \\
			\hline
			0 & 1 & 1 & 0 & 1 & 1 & 0 & 0 & 1 & 1 \\
			\hline
			0 & 1 & 1 & 1 & 1 & 1 & 1 & 1 & 1 & 1 \\
			\hline
			1 & 0 & 0 & 0 & 0 & 0 & 1 & 0 & 0 & 1 \\
			\hline
			1 & 0 & 0 & 1 & 0 & 0 & 1 & 0 & 1 & 1 \\
			\hline
			1 & 0 & 1 & 0 & 1 & 0 & 1 & 0 & 0 & 1 \\
			\hline
			1 & 0 & 1 & 1 & 1 & 0 & 1 & 0 & 1 & 1 \\
			\hline
			1 & 1 & 0 & 0 & 0 & 1 & 1 & 0 & 0 & 1 \\
			\hline
			1 & 1 & 0 & 1 & 0 & 1 & 1 & 0 & 1 & 1 \\
			\hline
			1 & 1 & 1 & 0 & 1 & 1 & 0 & 0 & 0 & 1 \\
			\hline
			1 & 1 & 1 & 1 & 1 & 1 & 1 & 1 & 1 & 1 \\
			\hline
		\end{tabular}
	\end{center}
	
\end{example}


Since I am in my very early phase of writing proofs for mathematical statements, I am going to do some proofs here (the questions are from the PLP book) to practice mathematical writing and thinking.


\begin{example}{Even or Odd!}
	\textbf{Question.} prove the following statements.
	\begin{enumerate}
		\item The product of two odd numbers is odd
	\end{enumerate}

	\proof{Proof 1.} \\
	Let $ a,b $ be two odd integers. So by the definition of odd numbers we can write them as: $ a = 2k+1, b = 2l+1 $ for some integers $ k,l $. By adding two integers and some simplification we will have: $ a+b = 2(k+l+1) $. Since $ k+l+1 $ is an integer, it follows from the definition even numbers that $ a+b $ is even. \qed \\
	
	\proof{Logic Flow of the Proof 1.}
	To make life easier, let's rewrite the statement that we want to prove:
	\begin{quote}
		\centering
		$ T: s $ is the product of two odd numbers $ a,b $ \\
		$ U: s $ is odd.
	\end{quote}
	so in this prove we want to evaluate the truth value of the following statement:
	\[ T \imply U \]
	
	You might have guessed that the statement $ T $ does not seem to be an atomic statement (it is composite of smaller statement). So we can not directly start with the statement $ T $ and we first need to construct it with atomic sentences. Let $ P_1, P_2 $ be the following statements:
	\begin{quote}
		\centering
		$ P_1: $ Integer $ a $ is odd \\
		$ P_2: $ Integer $ b $ is odd
	\end{quote}
	Considering the definition of the odd numbers we know that the following statements are \textbf{true}
	\begin{quote}
		\centering
		$ I_1: P_1 \imply Q_1 $ (is true)\\
		$ I_2: P_2 \imply Q_2 $ (is true)
	\end{quote}
	in which $ Q_1, Q_2 $ are:
	\begin{quote}
		\centering
		$ Q_1: a = 2k+1, \quad k \in \mathbb{Z} $ \\
		$ Q_2: b = 2l+1, \quad l \in \mathbb{Z} $
	\end{quote}
	Now at this point we utilize the following \textbf{tautology}:
	\begin{quote}
		\centering
		$ ((P_1 \imply Q_1) \wedge (P_2 \imply Q_2)) \imply ((P_1 \wedge P_2) \imply (Q_1 \wedge Q_2))  $
	\end{quote}
	Since $ I_1, I_2 $ are true, using the tautology above along with Modus Ponens, we can infer that the following statement is also true:
	\begin{quote}
		\centering
		$ P_1 \wedge P_2 \imply Q_1 \wedge Q_2 $ (is true)
	\end{quote}
	Now we can construct the statement $ T $ with following \textbf{true} statement (that follows from the properties of the integer arithmetic) 
	\begin{quote}
		\centering
		$ Q_1 \wedge Q_2 \imply T $
	\end{quote}
	with $ T $ defined as before. Following the rules of arithmetic of integers numbers we can write the following \textbf{true} statement.
	\[ T \imply T_1 \text{ (is true)} \]
	in which 
	\begin{quote}
		\centering
		$ T_1: s = 4kl + 2(k+l) + 1 $
	\end{quote}
	and once again utilizing the properties of integer arithmetic, we can write the following \textbf{true} statement:
	\[ T_1 \imply T_2 \text{ (is true)} \]
	in which $ T_2 $ is:
	\begin{quote}
		\centering
		$ T_2: s =  2(2kl + (k+l)) + 1 $
	\end{quote}
	since $ 2kl + (k+l) \in \mathbb{Z} $ we can finally follow the definition of odd number, and write the following \textbf{true} statement:
	\[ T_2 \imply U \]
	in which the statement $ U $ is as defined before. Now it is time to chain the baby statement with AND and infer the original statement. However this should be done in two steps. The first step is to use the following tautology to infer $ P_1 \wedge P_2 \imply T $ is a \textbf{true} statement:
	\begin{quote}
		\centering
		$ (P_1 \wedge P_2 \imply Q_1 \wedge Q_2) \wedge (Q_1 \wedge Q_2 \imply T) \imply (P_1 \wedge P_2) \imply T $
	\end{quote}
	so we can infer that $ P_1 \wedge P_2 \imply T $ is a true statement. Then assuming $ P_1 \wedge P_2 $ is  true and using modus ponens we can infer that $ T $ is also a \textbf{true} statement. So we will have the following conclusion from the first step:
	\begin{quote}
		\centering
		$ T: $ is a true statement
	\end{quote}
	Now here comes the second part: considering the already derived true statement and linking them by AND we can write the following \textbf{tautology}:
	\begin{quote}
		\centering
		$ ((T \imply T_1) \wedge (T_1 \imply T_2) \wedge (T_2 \imply U)) \imply (T \imply U) $ 
	\end{quote}
	Since the LHS of the above tautology is true, then modus ponens says the RHS should be true as well. So in summary we got:
	\begin{quote}
		\centering
		$ T \imply U $ is a true statement.
	\end{quote}
	and since from the step 1 we know that $ T $ is true, then again modus ponens says that $ U $ is also true. \qed
	\newline
	
	\textbf{Important note:} This tedious way of proving such a simple statement, is not the way that we usually do math proof. Here I just wanted to demonstrate the logic proof behind the teeny tiny math proof in the first section. \\
	
	\proof{Another proof with logical flow.} Considering the following statements:
	\begin{quote}
		$ P_1:$ $ a,b $ are two odd numbers, \\
		$ P_2:$ $ a = 2k+1 $ and $ b=2l+1 $ for $ k,l \in \mathbb{Z} $, \\
		$ P_3:$ $ (ab) = (2k+1)(2l+1) = 4kl+2(k+l)+1 = 2q+1 $ for $ q \in \mathbb{Z}, $ \\
		$ P_4:$ $ ab $ is odd.
	\end{quote}
	On the other hand the following implications are true as well:
	\begin{quote}
		$ P_1 \imply P_2 $ is true (definition of odd numbers),
		$ P_2 \imply P_3 $ is true (integers arithmetic properties),
		$ P_3 \imply P_4 $ is true (definition of odd numbers),
	\end{quote}
	and using the following tautology
	\[  ((P_1 \imply P_2) \wedge (P_2 \imply P_3) \wedge (P_3 \imply P_4)) \imply (P_1 \imply P_4)  \]
	along with Modus Ponens to infer that the implication $ P_1 \imply P_2 $ is true.
\end{example}

\subsection{Some Proofs and Examples}


\begin{example}{Even or Odd?}
	\textbf{Question.} Prove that sum of two odd numbers is even. \\
	
	\proof{Proof.} let $ a,b $ be two odd integers. Then there are $ l,k \in \mathbb{Z} $ such that $ a = 2k+1, b = 2l+1 $. Hence \[ a+b = 2k+1+2l+1 = 2(k+l+1) \]
	Since $ k,l $ are integers, $ k+l+1 \in \mathbb{Z} $. So it follows from definition that $ a+b $ odd.
\end{example}

\begin{example}{}
	\textbf{Question.} Let $ n,a,b,x,y \in \mathbb{Z} $. if $ n|a $ and $ n|b $ then $ n|(ax+by) $ \\
	
	\proof{Proof.}
	Assume $ a,b,x,y,n \in \mathbb{Z} $ and $ n|a $, and $ n|b $. There are $ k,l \in \mathbb{Z} $ such that 
	\[ a = nk, \quad b = nl \]
	Multiplying first and second equation is $ x,y $ correspondingly will yield:
	\[ ax = xnk, \quad by = ynl \]
	Adding the two equations and some basic arithmetic will results in: 
	\[ax + by = nkx + nly = n(kx+ly)\]
	Since $ k,lx,y \in \mathbb{Z} $, so $ kx+ly \in \mathbb{Z} $. So it follows from the definition that:
	\[ n|ax + by \] 	\qed \\
	
	\proof{Proof with another wording.} let $ n,a,b,x,y \in \mathbb{Z}$ and assume $ n|a $ and $ n|b $. So there are $ k,l \in \mathbb{Z} $ such that $ a = nk $ and $ b = nl $, hence $ ax = nkx $ and $ by = nly $. Adding two equations will result in $ ax+by = nkx + nly = n (kx+ly) $. Since $ k,l,x,y $ are integers, then so is $ kx+ly $. Then it follows from definition $ n|ax+by $ \qed
\end{example}

\begin{example}{}
	\textbf{Question.} Let $ n,a \in \mathbb{Z} $. Prove that if $ n|a $ and $ n|(a+1) $, then $ n=1 $ or $ n=-1 $ \\
	
	\proof{Proof.} Assume $ n,a \in \mathbb{Z} $ and $ n|a $ and $ n|(a+1) $. So there are $ k,l \in \mathbb{Z} $ such that $ a=nk $ and $ a+1 = nl $. Subtracting the first equation from the second one yields
	\[ 1 = n(l-k) \]
	Since $ (l-k) \in \mathbb{Z} $, we see $ n|1 $. Since $ n $ is an integer, the only possibilities are $ n=1 $ or $ n=-1 $
\end{example}

\begin{example}{}
	\textbf{Question.} Let $ n \in \mathbb{Z} $. If $ 3|(n-4) $, then $ 3|(n^2-1) $ \\
	
	\proof{Proof.} Assume $ n \in \mathbb{Z} $ and $ 3|(n-4) $. Then there is $ k \in  \mathbb{Z} $ such that $ n-4 = 3k $. Thus, $ n=3k+4 $, which implies
	\[ n^2 - 1 = (3k+4)^2 - 1 = 9k^2 + 15 + 24k = 3 (3k^2 + 8k +5)  \]
	Since $ (3k^2 + 8k +5) \in \mathbb{Z} $ so $ 3|(n^2-1) $ \qed \\
	
	\proof{Proof with other wordings}. Let n be an integer and assume $ 3|(n-4) $. So there is an integer k such that $ n-4 = 3k $. Solving for $ n $ yields $ n=3k+4 $. Hence $ n^2 - 1 = 9k^2 + 15 + 24 = 3 (3k^2+5+6) = 3q $. Since $ k $ is an integer then so is $ q = 3k^2+5+6 $. So it follows from definition that $ 3|(n^2-1) $ \qed \\
	
	\proof{Proof with logical steps}. Let the statements $ P_1, P_2, P_3 $ and $ P_3 $ be:
	\begin{quote}
		$ P_1: 3|n-4, \quad n \in \mathbb{Z}, $ \\
		$ P_2: n-4 = 3k, \quad k \in \mathbb{Z}, $ \\
		$ P_3: n = 3k+4,  $ \\
		$ P_4: n^2-1 = 9k^2 + 15 + 24 = 3 (3k^2+5+6) = 3q, \quad q \in \mathbb{Z} $, \\
		$ P_5: 3|(n^2-1) $
	\end{quote}
	The following implications are true:
	\begin{quote}
		$ P_1 \imply P_2 $ is true (definition of divisibility), \\
		$ P_2 \imply P_3 $ is true (integers arithmetic), \\
		$ P_3 \imply P_4 $ is true (integers arithmetic), \\
		$ P_4 \imply P_5 $ is true (definition of divisibility).
	\end{quote}
	By connecting these baby implications with the and operator and then utilizing the corresponding tautology and then using modus ponens we can infer that $ P_1 \imply P_2 $ is true.
\end{example}

\begin{example}{This is a bad proof}
	\textbf{Question.} Consider the following faulty proof. Explain why is that a faulty proof and provide the correct answer.
	\begin{quote}
		\centering
		Let $ a,b $ and $ c $ be integers. if $ a|b $ and $ b|c $, then $ a|c $ \\
	\end{quote}

	
	\proof{Faulty Proof.} Assume $ a,b,c \in \mathbb{Z} $ such that $ a|b $, and $ b|c $. So there is $ k \in \mathbb{Z} $ such that $ b=ak $ and $ c=bk $. Substituting $ b $ in the second equation yields:
	\[c  = ak^2\]
	Since $ k^2 \in \mathbb{Z} $, we see $ a|c $ \qed \newline
	
	\proof{Why is this faulty?} The proof provided above is a faulty proof since it has used the integer $ k $ in two different places. The fact that $ a|b $ and $ b|c $ means that $ b $ is an integer multiple of $ a $, and $ c $ is an integer multiple of $ b $ correspondingly and those integer multiples are not necessarily the same. So the proof is not correct although the logic and its flow is correct. The correct proof is as follows:
	
	\proof{Proof.} Assume $ a,b,c \in \mathbb{Z}$ and $ a|b $ and $ b|c $. So there are $ k,l \in \mathbb{Z} $ s.t. $ b=ka $ and $ c=bl $. By substituting $ b $ from first equation in the second one, we will get $ c=akl $. Since $ a,k \in \mathbb{Z} $, so $ ak \in \mathbb{Z} $ and we see $ a|c $
	
\end{example}

\begin{example}{}
	\textbf{Question.} The floor function, denoted by $ \lfloor x \rfloor $, is defined to be the function that takes a real number $ x $ and returns the greatest integer less than or equal to $ x $. This is also sometimes called the greatest integer function. Using this definition prove that
	\[ x \in \mathbb{Z} \Leftrightarrow \floor{x} = x \] \\
	
	\proof{Proof: ($ \Leftarrow $ direction).} Let's start with the backward implication which is simpler. Assume $ \floor{x} = x $. Based on the definition of the floor function, it returns integer numbers as output. So $ \floor{x} = x \in \mathbb{Z} $.
	
	\proof{Proof: ($ \Rightarrow $ direction).} Assume $ x \in \mathbb{Z} $. From the definition of the floor function, we can infer $ \floor{x} \leq x $. On the other hand, since $ x \in \mathbb{Z} $, so $ x \leq x $. In other words, $ x $ is an integer that is less than or equal to $ x $. Moreover, by definition, $ \floor{x} $ is the largest integer less than $ x $. Combining this fact with the $ x \leq x $ inequality yields $ \floor{x} \geq x $. So $ \floor{x} \leq x \leq \floor{x} $, which implies $ \floor{x} = x $. \qed
	
\end{example}


\begin{example}{}
	\textbf{Question.} We call a number $ n $ an \emph{integer root} if $ n^k=m $ for some $ k \in \mathbb{N} $ and $ m \in \mathbb{Z} $. For example $ \sqrt{7} $ is a root because $ (\sqrt{7})^2 = 7 $. However, $ \frac{5}{3} $ is not an integer root. Prove that if $ a $ and $ b $ are integer roots, then so is $ ab $. \\
	
	\proof{Proof.} Assume $ a,b \in \mathbb{Z} $ are integer roots. So there are $ k,l \in \mathbb{N} $ and $ m,n \in \mathbb{Z} $ such that $ a^k = m $ and $ b^l=n $. Raising the both sides of the first equation to the power $ l $ and second equation to the power $ k $ will yield $ a^{kl} = m^l $ and $ b^{kl} = n^k $. By multiplying two sides of equations we can write:
	\[ (ab)^{lk} = m^l n^k \] 
	Since $ k,l \in \mathbb{N} $ and $ m,n \in \mathbb{Z} $, $ m^l n^k \in \mathbb{Z} $ and $ lk \in \mathbb{N} $. So we can conclude that ab is also an integer root. \qed
\end{example}

\begin{example}{Faulty Proof}
	\textbf{Question.} Consider the following faulty proof. Determine why is the proof faulty and then provide the correct proof.
	\begin{quote}
		\centering
		Let $ x $ be a positive real number. If $ x<1 $ then $ 1 < \frac{3x+2}{5x} $
	\end{quote}
	
	\proof{Faulty Proof.} Let $ x $ be positive. Then by multiplying the inequality 
	\[ 1 < \frac{3x+2}{5} \]
	by $ 5x $, which is positive, we obtain
	\[ 5x < 3x + 2 \]
	Collecting like terms, we have $ 2x < 2 $, and finally dividing by 2, we have $ x < 1 $. \\
	
	\proof{Discussion.} This is a faulty proof because the flow of the logic is not correct. Usually, in working with inequalities, sometimes it is beneficial to start with the conclusion and arrive at the hypothesis. Although doing this will give us some hint that what is the best starting position, we need to note that this flow of logic is not a proof and we need to re write the proof after getting some hints.
	
	\proof{Correct Proof.} Let $ x $ is a positive real number. Assume $ x < 1 $. By multiplying both sides in 2 and adding $ 3x $ to both sides we will get $ 5x < 3x+2 $. Since $ 5x \neq 0 $ and $ 5x > 0 $, so dividing both sides by $ 5x $ we will have: $ 1 < \frac{3x+2}{5x} $ which is the conclusion. \qed
\end{example}



\begin{example}{Square roots}
	\proof{Question.} Without using calculus show that for two positive real numbers $ x,y $, 
	\[ y>x \imply \sqrt{y}>\sqrt{x}. \] \\
	
	\proof{Proof.} Let $ x,y $ be two positive real numbers such that $ 0<x<y $. Then $ y-x>0 $. Then $ (\sqrt{y} - \sqrt{x})(\sqrt{y} + \sqrt{x}) > 0 $. Since $ x,y $ are two positive real numbers, so $ (\sqrt{y} + \sqrt{x}) > 0 $. Then we conclude $ \sqrt{y} - \sqrt{x} >0 $ \qed
\end{example}

\begin{example}{square roots again!}
	\proof{Question.} Prove that for every positive real number we have
	\begin{equation*}
		\sqrt{ab}< \frac{a+b}{2}.
	\end{equation*} \\

	\proof{Scratch work stuff}. Let's begin with the inequality that we want to show and do some algebra with it.
	\begin{align*}
		&\sqrt{ab} < \frac{a+b}{2},\\
		&ab < \frac{(a+b)^2}{4},\\
		&4ab < a^2 + b^2 + 2ab,\\
		&0 < (a-b)^2.
	\end{align*}
	
	Note that this is not the correct flow of the logic for proof. However this scratch work gives us the hint that it might be better to start with the statement $ (a-b)^2 > 0 $. \\
	
	\proof{Proof.} Let $ a,b \in \mathbb{R} $ and with out the loss of generality we can assume that $ a>b $ hence $ a-b >0 $. By squaring both sides of the inequality we will have $ (a-b)^2 > 0 $ which yields $ a^2 + b^2 -2ab + 4ab > 4ab $. In the last step we have added the real number $ 4ab $ to the both sides of the inequality. So $ a^2 + b^2 + 2ab > 4ab $ hence $ (a+b)^2>4ab $. Since the square root function is a positive and increasing function, then we can apply it to the both sides of the inequality with out changing the direction of the inequality:
	\[ a+b > 2 \sqrt{ab}, \] which then yields:
	\[ \sqrt{ab} < \frac{a+b}{2}. \]
\end{example}


\begin{example}{square roots once again!}
	\proof{Question.} Show that for $ x,y \in \mathbb{R} $ that $ x,y \geq 0 $ we have: 
	\[\sqrt{x+y} \leq \sqrt{x} + \sqrt{y}.  \]
	
	\proof{Scratch paper work}. Let's start with the inequality that we want to show and do some algebra to see what happens:
	\begin{align*}
		&\sqrt{x+y} \leq \sqrt{x} + \sqrt{y}, \\
		&x+y \leq x + y + 2\sqrt{xy}, \\
		&0 \leq \sqrt{xy}.
	\end{align*}
	So it seems reasonable to start with the statement $ \sqrt{xy} \geq 0 $.
	
	\proof{Proof.} Let $ x,y \in \mathbb{R} $ and $ x,y \geq 0 $. Then $ xy \geq 0 $. Since square root function is a positive and increasing function then we can apply it to the both sides of the inequality and get: $ \sqrt{xy} \geq 0 $, hence $ 2 \sqrt{xy} \geq 0 $. Since we can add any real number to both sides of any inequality we do so with $ x+y $ which yields $ x+y+2\sqrt{xy} \geq x+y $. This equation can be written as: $ (\sqrt{x} + \sqrt{y})^2 \geq x+y $ and again by taking the square root of the both sides we will arrive at $ (\sqrt{x} + \sqrt{y}) \geq \sqrt{x+y}. $ \qed
\end{example}









