\section{Direct Proof}
It is often the case in the mathematical proof that we start with assuming the hypothesis of an implication is true and conclude with the truth of the conclusion of the implication. The general structure of those proves will be all similar. Suppose that we want to prove that the implication 
\[ P \imply Q \]
is true. When know that if $ P $ is false, then the statement will automatically be true. But if $ P $ is true, then the implication will be true only when the $ Q $ is true. So if we start by assuming that $ P $ is true and \emph{through linking smaller true implications} we arrive at the conclusion, then by using modus ponens we can infer that conclusion of the original implication is also true. In other words, to prove $ P \imply Q $ is true, we need to construct the following \textbf{true} implications:
\begin{align*}
	P &\imply P_1 \quad \text{is true} \\
	P_1 &\imply P_2 \quad \text{is true} \\
	P_2 &\imply P_3 \quad \text{is true} \\
	 \vdots & \\
	P_n &\imply Q \quad \text{is true}
\end{align*}

Then we can connect these statements with AND operator and have:
\begin{equation}\label{equ:babyImpliactionsImplyBigImplication}
	(P \imply P_1) \wedge (P_1 \imply P_2) \wedge \cdots \wedge (P_n \imply Q)) \imply (P \imply Q)
\end{equation}


Here at this point there are two ways to proceed with the proof. In the following section I'll discuss both of them:

\textbf{The first method to proceed with the proof:} In this point of view, we assume that the statement $ P $ is true. Since the first baby implication is also true, then using Modus Ponens we can infer that $ P_1 $ is also true. Then since $ P_1 $ is true and the second baby implication is also true, then using Modus Ponens we can infer that $ P_2 $ is also true. Using the same logic we can infer that $ P_3, P_4, \cdots P_n $ are also true. And for the last time, since the last baby statement is true and $ P_n $ is also true, then we can infer (using Modus Ponens) $ Q $ is also true. So we started with the true hypothesis and we arrived at the true conclusion. So utilizing the truth table of implication we can infer that the implication $ P \imply Q $ is true. 

\textbf{The second method to proceed with the proof:} The golden equation in utilizing this method is \ref{equ:babyImpliactionsImplyBigImplication}. From now on, when I mention RHS and LHS, I mean the right hand side and the left hand side of the equation \ref{equ:babyImpliactionsImplyBigImplication}. Since all of the statements in the LHS is true, then the whole statement in the LHS is also true. And since the statement LHS $ \imply $ RHS is also true (check out the equation \ref{equ:babyImplicationImplyBigImplicationProof}), then we can infer (modus ponens) that the LHS of the implication (i.e. $ P \imply Q $) is also true. So if P is true then Q is also true.

These two methods are not really two different methods. These are two different methods to think about the same thing are are actually two sides of a single coin!

\begin{example}{Is $ n^2 $ Even or Odd?}
	\textbf{Question.} Let $ n $ be an integer. prove that if $ n $ is even, then $ n^2 $ is also even. \\

	\emph{Scratch Paper Stuff.} We are going to use direct proof method. So let's construct the first baby implication. let the statement $ P $ be:
	\[ P \text{: integer n is even} \]
	and let the statement $ P_1 $ be:
	\[ P_1 \text{: n can be written as } n=2k, k \in \mathbb{Z} \]
	From the definition of even numbers we know that the implication $ P \imply P_1 $ is true. Now lets construct the second baby implication. Let the statement $ P_2 $ be:
	\[ P_2 \text{: the square root of } 2k \text{ is } 4k^2  \]
	From the properties of the multiplication of integers we know that if $ a=b $ then $ ac = bc $ for any integer $ c $ and the since the $ a=b=c=2 $ is an special case of this property, then $ P_1 \imply P_2 $ is also true. Let's construct the third baby implication. Let the statement $ P_3 $ be:
	\[ P_3 \text{: the integer } 4k^2 \text{ can be written as } 2(2k^2)\]
	Considering the associativity property of the multiplication (i.e. $ (ab)c = a(bc) $) then we know that the statement $ P_2 \imply P_3 $ is also true.
	And finally, we can utilize the definition of the even number once again and construct the following final baby implication step. Let $ Q $ be the statement
	\[ Q: n=2(2k^2) \text{ is an even number} \]
	Because of the definition of the even number we know that $ P_3 \imply P_4 $ is also true. So in a nutshell we have:
	\begin{align*}
		P &\imply P_1 \text{ is true (definition of even number)} \\
		P_1 &\imply P_2 \text{ is true (property of integers)} \\ 
		P_2 &\imply P_3 \text{ is true (associativity property of integers)} \\
		P_4 &\imply Q \text{ is true (definition of even number)}
	\end{align*}

	By connecting these statements to each other with AND operator, we can write:
	\[ ((P \imply P_1 ) \wedge (P_1 \imply P_2) \wedge (P_2 \imply P_3) \wedge (P_4 \imply Q)) \imply (P \imply Q) \]
	Since all of the terms on the LHS is true, and the whole statement is also true (proved in \ref{equ:babyImplicationImplyBigImplicationProof}) we can infer (modus ponens) that the right hand side is also true. We need a little extra step here to complete the proof: Since the statement $ P $ is true, and the statement $ P \imply Q $ is also true, then we can infer (Modus Ponens) that $ Q $ is also true. \qed \\
	
	\emph{Clean Proof.} Assume $ n \in \mathbb{Z} $ is even. Using the definition of even numbers, we can write n as $ n=2k $ for some integer k. Now squaring the both sides of the equation and factoring 4 we can write: $ n^2 = 4k^2 = 2(2k^2) $. Since $ 2k^2 \in \mathbb{Z} $, it follows from the definition that $ n^2 $ is even. \qed
	
\end{example}

You might be wondering what is the square block (i.e. \qedsymbol) at the end of each proof. This is called ``QED'' which stands for ``quod erat demonstrandum'' which means `` which was to be demonstrated''. There are also other symbols in common use for this purpose that $ \blacklozenge, \blacksquare $ are the most common alternatives.


\subsection{Proof of Inequalities}
Although there are not such a thing as a general recipe for mathematical proof. But we can add some kind of tricks to our tool kit. Here in this section I will describe one of them that might come very handy in dealing with inequalities. To better demonstrate the idea, I would like to proceed with an example.

\begin{example}{Tricks in Working with Inequalities}
	\textbf{Question.} Let $ x,y \in \mathbb{R}$. Then prove $x^2 + y^2 \geq 2xy$ \\
	
	\underline{\emph{Scratch Paper Proof.}} When dealing with inequalities, it is better by studying a difference derived from the inequality. Let's start with studying the expression $ x^2 + y^2 - 2xy $. As soon as we start dealing with this, we observe that we can write it as $ (x-y)^2 $. Since $ x-y \in \mathbb{R} $ then we know that $ (x-y)^2 >0 $. So we got the intuition that if we start with the difference of real numbers $ x,y $, we can square it and then utilize the fact that the square should be positive and then arrive at the conclusion. Let's put these insights into good words and have more cleaner version of proof that has the appropriate flow of logic. \qed \\
	
	\proof{Proof.} Let $ x,y $ be real numbers. Hence $ (x-y) \in \mathbb{R} $ and we can write $ (x-y)^2 \geq 0 $. Expanding this will give: $ (x-y)^2 = x^2 + y^2 -2xy \geq 0 $. This can be rewritten as: $ x^2 + y^2 \geq 2xy $. \qed
	
\end{example}

So the trick in a nutshell is: To prove and inequality, start with studying a difference (maybe the RHS-LHS). This trick is far from being a general recipe for mathematical proof. There any MANY MANY MANY occasions that this will not work at all. So do not over use any trick in your toolbox.

\begin{example}{Multiplying Inequalities}
	\textbf{Question.} Let $ a,b,c,d \in \mathbb{R} $ and $ a > b > 0 $ and $ c > d > 0 $. Prove $ ac > bd $. \\
	
	\proof{Proof.} Since $ c > 0 $ and $ b > 0 $ then we can multiply $ c $ at the first inequality and $ b $ at the second inequality without changing the direction of the inequality sign. So we will have:
	\begin{align*}
		ac > bc > 0 \\
		bc > bd > 0
	\end{align*}
	So by combining these two inequalities, we can have:
	\[ ac > bd \]
	
	\proof{Scratch Paper Proof.} There is a reason that I am writing the scratch paper proof after the main proof. That is to show that how this kind of proof matches with the logical baby steps. Let the statements $ P_1, P_2 $ be defined as:
	\begin{quote}
		\centering
		$ P_1: a>b>0 $ \\
		$ P_2: c>d>0 $
	\end{quote}
	Let's define the statements $ P $ as an auxiliary statement in the following way:
	\begin{quote}
		\centering
		$ P = P_1 \wedge P_2 $
	\end{quote} 
	utilizing the properties of the inequalities of real numbers (e.g. multiplying a negative number at both sides change the direction of inequality while multiplying a positive number do not change the direction) we can write the following \textbf{true} statements. 
	\begin{quote}
		\centering
		$ P \imply Q_1 $ \\
		$ P \imply Q_2 $ 
	\end{quote}
	in which $ Q_1, Q_2 $ are:
	\begin{quote}
		\centering
		$ Q_1: ac > bc $ \\
		$ Q_2: bc > bd $
	\end{quote}
	Since (trivially!) $ bc=bc $ so we can express the following \textbf{true} statement:
	\begin{quote}
		\centering
		$ Q_1 \wedge Q_2 \imply R $
	\end{quote}
	in which $ R $ is:
	\begin{quote}
		\centering
		$ R: ac>bd $
	\end{quote}
	Note that we have not yet shown that $ P = P_1 \wedge P_2 $ implies Q (although it might trivially make sense). To arrive at the $ P \imply Q $ from the baby steps, implicitly we are using the following tautology:
	\begin{quote}
		\centering
		$ ((P \imply Q_1) \wedge (P \imply Q_2) \wedge (Q_1 \wedge Q_2 \imply R)) \imply (P \imply R) $
	\end{quote}
	This is a tautology because its truth value is always true:
	\begin{center}
		\begin{tabular}{|c|c|c|c|c|c|c|c|c|c|}
			\hline
			P & Q & R & S & $ P \imply R $ & $ P \imply Q $ & $ R \wedge Q \imply S $ & LHS & RHS & LHS $ \imply $ RHS \\
			\hline
			0 & 0 & 0 & 0 & 1 & 1 & 1 & 1 & 1 & 1 \\
			\hline
			0 & 0 & 0 & 1 & 1 & 1 & 1 & 1 & 1 & 1 \\
			\hline
			0 & 0 & 1 & 0 & 1 & 1 & 1 & 1 & 1 & 1 \\
			\hline
			0 & 0 & 1 & 1 & 1 & 1 & 1 & 1 & 1 & 1 \\
			\hline
			0 & 1 & 0 & 0 & 1 & 1 & 1 & 1 & 1 & 1 \\
			\hline
			0 & 1 & 0 & 1 & 1 & 1 & 1 & 1 & 1 & 1 \\
			\hline
			0 & 1 & 1 & 0 & 1 & 1 & 0 & 0 & 1 & 1 \\
			\hline
			0 & 1 & 1 & 1 & 1 & 1 & 1 & 1 & 1 & 1 \\
			\hline
			1 & 0 & 0 & 0 & 0 & 0 & 1 & 0 & 0 & 1 \\
			\hline
			1 & 0 & 0 & 1 & 0 & 0 & 1 & 0 & 1 & 1 \\
			\hline
			1 & 0 & 1 & 0 & 1 & 0 & 1 & 0 & 0 & 1 \\
			\hline
			1 & 0 & 1 & 1 & 1 & 0 & 1 & 0 & 1 & 1 \\
			\hline
			1 & 1 & 0 & 0 & 0 & 1 & 1 & 0 & 0 & 1 \\
			\hline
			1 & 1 & 0 & 1 & 0 & 1 & 1 & 0 & 1 & 1 \\
			\hline
			1 & 1 & 1 & 0 & 1 & 1 & 0 & 0 & 0 & 1 \\
			\hline
			1 & 1 & 1 & 1 & 1 & 1 & 1 & 1 & 1 & 1 \\
			\hline
		\end{tabular}
	\end{center}
	
\end{example}


Since I am in my very early phase of writing proofs for mathematical statements, I am going to do some proofs here (the questions are from the PLP book) to practice mathematical writing and thinking.


\begin{example}{Even or Odd!}
	\textbf{Question.} prove the following statements.
	\begin{enumerate}
		\item The product of two odd numbers is odd
		\item The sum of two odd numbers is even
		\item The sum of two even numbers is even
		\item The sum of and even and and odd number is an odd number
	\end{enumerate}

	\proof{Proof 1.} Let $ a,b $ be two odd integers. So by the definition of odd numbers we can write them as: $ a = 2k+1, b = 2l+1 $ for some integers $ k,l $. By adding two integers and some simplification we will have: $ a+b = 2(k+l+1) $. Since $ k+l+1 $ is an integer, it follows from the definition even numbers that $ a+b $ is even.
	
	\proof{Logic Flow of Proof 1.}
	Since we know that the following statement is a tautology:
	\[ ((P_1 \imply Q) \wedge (P_2 \imply S)) \imply ((P_1 \wedge P_2) \imply (R \wedge S)) \equiv P \imply (R \wedge S) \]
	in which $ P = P_1 \wedge P_2$
	\begin{quote}
		\centering 
		$ P \imply Q $ is true
	\end{quote}
	in which:
	\begin{quote}
		\centering
		$ P_1: $ The integer a is even \\
		$ P_2: $ The integer b is even \\
		$ P: P_1 \wedge P_2 $ \\
		$ P_1: a=2k+1, b=2l+1,$ 
	\end{quote}
	
\end{example}