\section{Fundamentals}
In this section we will review some basic definitions in the mathematical proof and mathematical logic.

\begin{defbox}{Axiom}
	Axiom is a mathematical statement whose truth is accepted without proof.
\end{defbox}

For example the followings are some well-known axioms in mathematics:

\begin{itemize}
	\item Kolmogorov axioms (axioms of probability)
	\item Axioms of the Euclidean geometry: For every line $l$ and point $P$ that is not on the line, there exists only one line $l'$ that contains the point $P$ and is parallel to the line $l$.
\end{itemize}

\begin{defbox}{Theorem}
	A true mathematical statement whose truth can be verified using mathematical proof and following mathematical proof.
\end{defbox}

However, the mathematicians reserve the word theorem for true mathematical statements that is significant and very important. For instance the fact that $2 + 3 = 5$ is a true mathematical statement whose truth can be verified using mathematical proof. However, since it is not a significant results, it is not common to call it a theorem. Instead, alternative words are used like: proposition, results, fact, observation.  
