\documentclass[11pt,a4paper]{article}
\usepackage[T1]{fontenc}
\usepackage[left=2cm, right=2cm, top=2cm, bottom=2cm]{geometry}
\usepackage{graphicx}
\usepackage{mathtools}
\usepackage{amssymb}
\usepackage{amsthm}
\usepackage{thmtools}
\usepackage{xcolor}
\usepackage{nameref}
\usepackage[colorlinks=true, linkcolor=blue, citecolor=cyan]{hyperref}
\usepackage{natbib} 
\usepackage{tkz-graph}
\usepackage{placeins}
\usepackage{tikz}
\usetikzlibrary{arrows}
\usepackage{tikz-cd}
\usepackage{quiver}


\newcommand{\grad}{\operatorname{grad}}
\newcommand{\curl}{\operatorname{curl}}
\renewcommand{\div}{\operatorname{div}}
\newcommand{\img}{\operatorname{img}}

\newcommand{\Obj}{\operatorname{Obj}}
\newcommand{\Morph}{\operatorname{Morph}}


\newcommand{\boxedR}[1]{\fcolorbox{red}{white}{#1}}
\newcommand{\boxedG}[1]{\fcolorbox{green}{white}{#1}}

\newcommand{\xc}[1]{%
	\tikz[baseline=(n.base)]{
		\node[inner sep=0pt] (n) {$#1$};
		\draw[red,thick] (n.west) -- (n.east);
	}%
}



\theoremstyle{definition}
\newtheorem{definition}{Definition}
\newtheorem{prob}{Problem Statemnt}
\newtheorem{example}{Example}


\theoremstyle{remark}
\newtheorem{remark}{Remark}

\title{Hausdorff Spaces and Categorical Products}
\author{Ali Fele Paranj}



\begin{document}
	\maketitle
	
	\begin{abstract}
		Do not think too hard. This is a short note about some intuitive remarks on the Hausdorff spaces and the Categorical products, in a disjoint way!
	\end{abstract}
	
	\section{Categorical Product}
	We start with a definition.
	\begin{definition}
		Let  $ C $ be a category. ``The'' categorical product of $ X,Y \in \Obj(C) $, if it exists, is $ X\times Y \in \Obj(C) $ along with two morphisms $ \pi_1: X\times Y \to X $ and $ p_2:X\times Y \to Y $ with the following universal property: For any other $ Z\in \Obj(C) $, along with two maps $ f_1:Z\to X $ and $ f_2:Z\to Y $, there exists a unique morphism $ f: Z\to X\times Y $ such that the following diagram commutes.
		
		% https://q.uiver.app/#q=WzAsNCxbMiwxLCJYXFx0aW1lcyBZIl0sWzMsMCwiWCJdLFszLDIsIlkiXSxbMCwxLCJaIl0sWzAsMSwiXFxwaV8xIiwxXSxbMCwyLCJcXHBpXzIiLDFdLFszLDAsImYiLDEseyJzdHlsZSI6eyJib2R5Ijp7Im5hbWUiOiJkYXNoZWQifX19XSxbMywyLCJmXzIiLDEseyJjdXJ2ZSI6Mn1dLFszLDEsImZfMSIsMSx7ImN1cnZlIjotMn1dXQ==
		\[\begin{tikzcd}
			&&& X \\
			Z && {X\times Y} \\
			&&& Y
			\arrow["{f_1}"{description}, curve={height=-12pt}, from=2-1, to=1-4]
			\arrow["f"{description}, dashed, from=2-1, to=2-3]
			\arrow["{f_2}"{description}, curve={height=12pt}, from=2-1, to=3-4]
			\arrow["{\pi_1}"{description}, from=2-3, to=1-4]
			\arrow["{\pi_2}"{description}, from=2-3, to=3-4]
		\end{tikzcd} \]
		
		The categorical product of $ X $ and $ Y $ is unique up to isomorphism. In fact, more is true. They are unique up to unique isomorphism.
	\end{definition}
	
	
	
\end{document}