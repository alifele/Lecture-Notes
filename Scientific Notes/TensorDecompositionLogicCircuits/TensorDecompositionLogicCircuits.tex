\documentclass[11pt,a4paper]{article}
\usepackage[T1]{fontenc}
\usepackage[left=2cm, right=2cm, top=2cm, bottom=2cm]{geometry}
\usepackage{graphicx}
\usepackage{mathtools}
\usepackage{amssymb}
\usepackage{amsthm}
\usepackage{thmtools}
\usepackage{xcolor}
\usepackage{nameref}
\usepackage{hyperref}
\usepackage{karnaugh-map}

\usepackage{nicematrix}


\title{Tensor Decomposition For Logic Circuits}
\author{Ali Fele Paranj}
\begin{document}
	\maketitle
	Let $ A $ represent a boolean variable. Then observe that
	\[ A + \overline A = 1. \]
	We can use this fact to simplify the logical circuits. For instance to implement $ f(A,B,C,D) = \overline A B C D + A \overline B C D  $ as it is we need more logic gates to implement its equivalent $ f(A,B,C,D) = CD $. Karnaugh map makes such simplification possible for any logic function. For instance,  consider the following truth table: 
	
	\begin{center}
		\begin{NiceTabular}{|c|c|c|c|c|} 
			\hline
			A & B & C & D & F(A,B,C,D) \\
			\hline
			0 & 0 & 0 & 0 & 0 \\
			0 & 0 & 0 & 1 & 0 \\
			0 & 0 & 1 & 0 & 0 \\
			0 & 0 & 1 & 1 & 0 \\
			0 & 1 & 0 & 0 & 0 \\
			0 & 1 & 0 & 1 & 0 \\
			0 & 1 & 1 & 0 & 1 \\
			0 & 1 & 1 & 1 & 0 \\
			1 & 0 & 0 & 0 & 1 \\
			1 & 0 & 0 & 1 & 1 \\
			1 & 0 & 1 & 0 & 1 \\
			1 & 0 & 1 & 1 & 1 \\
			1 & 1 & 0 & 0 & 1 \\
			1 & 1 & 0 & 1 & 1 \\
			1 & 1 & 1 & 0 & 1 \\
			1 & 1 & 1 & 1 & 0 \\
			\hline
		\end{NiceTabular}
	\end{center}
	We can write this function as
	\[ F(A,B,C,D) = \overline A BC \overline D + A \overline{BCD} + A\overline{BC}D + A\overline B C \overline D + A\overline B CD + AB \overline{CD} + AB\overline{C}D + ABC\overline{D}. \]
	To decompose 4-linear form and then use the identity above simplify the function we can use the Karnaugh map.
	\begin{center}
		\begin{karnaugh-map}[4][4][1][$CD$][$AB$]
			\minterms{6,8,9,10,11,12,13,14}
			\autoterms[0]
			\implicant{8}{10}
			\implicant{12}{9}
			\implicant{6}{14}
		\end{karnaugh-map}
	\end{center}
	The grouped boxes above shows which terms can be grouped to make a simpler term. For instance, the green box above show the following terms that can be grouped easily:
	\begin{align*}
		&AB\overline{CD} + AB\overline{C}D + A\overline{BCD} + A\overline{BC}D\\
		&= AB\overline C (D+\overline D) + A\overline{BC}(D+\overline D)\\
		&= A\overline C(B+\overline B) \\
		&= A\overline C.
	\end{align*}
	With a similar reasoning, the red box results in the term $ A\overline B $ and the yellow box results in the term $ BC\overline D $. So the simplified function will be
	\[ F(A,B,C,D) = A\overline C + A\overline B + BC\overline D. \]

	
	

	
\end{document}