\documentclass[11pt,a4paper]{article}
\usepackage[T1]{fontenc}
\usepackage[left=2cm, right=2cm, top=2cm, bottom=2cm]{geometry}
\usepackage{graphicx}
\usepackage{mathtools}
\usepackage{amssymb}
\usepackage{amsthm}
\usepackage{thmtools}
\usepackage{xcolor}
\usepackage{nameref}
\usepackage[colorlinks=true, linkcolor=blue, citecolor=cyan]{hyperref}
\usepackage{natbib} 
\usepackage{tkz-graph}
\usepackage{placeins}


\newcommand{\grad}{\operatorname{grad}}
\newcommand{\curl}{\operatorname{curl}}
\renewcommand{\div}{\operatorname{div}}
\newcommand{\img}{\operatorname{img}}
\renewcommand{\span}{\operatorname{span}}



\theoremstyle{definition}
\newtheorem{definition}{Definition}


\theoremstyle{remark}
\newtheorem{remark}{Remark}

\title{Some Intuitions on the Continuous Maps}
\author{Ali Fele Paranj}



\begin{document}
	
	\maketitle
	\begin{abstract}
		In this document I will provide an intuitive example on the notion of continuous maps.
	\end{abstract}
	
	To define a continuous map, we first need to have some topological spaces (at least one). A topological space is the minimum setting required to talk about the notion of continuoity. We can also have higher level structures (like metric spaces, and vector spaces) that are more special example of the topological spaces, but in this writing we will stick to the general topological spaces. A topological space is a set with a notion of topology, which is a collection of subsets of the set that are closed under union and finite intersection. These sets are called the open sets.
	
	
	\begin{definition}
		A topological space is a pair $ (X,\mathcal{T}) $, where $ X $ is a set and $ \mathcal{T} \subseteq \mathcal{P} $ is a subset of the power set of $ X $ that is closed under arbitrary union and finite intersection. The elements of $ \mathcal{T} $ are called open sets.
	\end{definition}
	
	
	\begin{definition}
		A map between two topological spaces $ f: (X,\mathcal{T}_X) \to (Y,\mathcal{T}_Y) $ is a continuous map if for all $ A \in \mathcal{T}_Y $ we have $ f^{-1}(A) \in \mathcal{T}_X $.
	\end{definition}
	
	It might be hard to see the intuitive meaning behind the notion of a continuous map. To make a good intuitive example, we first need to construct a very simple topological space. To do this first consider the following remark.
	
	\begin{remark}
		Let $ X $ be a finite set, and let $ P $ be a partition on the set (i.e. a collection of subsets of $ X $ that are disjoint and their union is the whole set). Then $ P $ is a subbasis for a topology on $ X $, and the topology that it generates, i.e. $ \mathcal{T} $ is the collection of all unions of elements of $ P $. Note that in general, then topology that a subbasis generates is the arbitrary union of the finite intersections of the elements of $ P $, but because $ P $ is a partition, their elements are disjoint.
	\end{remark}
	
	Consider the following diagram. In the left subplot, there are two sets: A set consisting of point $ \{A,B,C,D\} $, and a set containing different color values. The set on top is also a topological spaces, where the topology is generated by the subbasis denoted by the dashed circles. I.e. the subbasis consists of the sets $ S = \{ \{A\}, \{B,C\}, \{D\} \} $. And on the right subplots there is a mapping from the set of circles to the set of colors (i.e. coloring of the circles). The way that we have drawn the color bar, implicitly implies that we are assuming some kind of topology on that set (similar to the standard topology of the real line). But we are going to be explicit here and define a topology by defining a partition on the topology.
	
	\begin{figure}[h!]
	\centering
	
	
	
	% Gradient Info
	
	\tikzset {_3sybf9svc/.code = {\pgfsetadditionalshadetransform{ \pgftransformshift{\pgfpoint{0 bp } { 0 bp }  }  \pgftransformrotate{0 }  \pgftransformscale{2 }  }}}
	\pgfdeclarehorizontalshading{_msh882t58}{150bp}{rgb(0bp)=(0.82,0.01,0.11);
		rgb(37.5bp)=(0.82,0.01,0.11);
		rgb(43.973214285714285bp)=(0.97,0.91,0.11);
		rgb(49.24107142857143bp)=(0.72,0.91,0.53);
		rgb(55.58035714285714bp)=(0.11,0.97,0.94);
		rgb(62.00892857142857bp)=(0.8,0.29,0.89);
		rgb(100bp)=(0.8,0.29,0.89)}
	
	% Gradient Info
	
	\tikzset {_s6rolseyi/.code = {\pgfsetadditionalshadetransform{ \pgftransformshift{\pgfpoint{0 bp } { 0 bp }  }  \pgftransformrotate{0 }  \pgftransformscale{2 }  }}}
	\pgfdeclarehorizontalshading{_2kvz2ka7v}{150bp}{rgb(0bp)=(0.82,0.01,0.11);
		rgb(37.5bp)=(0.82,0.01,0.11);
		rgb(43.973214285714285bp)=(0.97,0.91,0.11);
		rgb(49.24107142857143bp)=(0.72,0.91,0.53);
		rgb(55.58035714285714bp)=(0.11,0.97,0.94);
		rgb(62.00892857142857bp)=(0.8,0.29,0.89);
		rgb(100bp)=(0.8,0.29,0.89)}
	\tikzset{every picture/.style={line width=0.75pt}} %set default line width to 0.75pt        
	
	\begin{tikzpicture}[x=0.75pt,y=0.75pt,yscale=-1,xscale=1]
		%uncomment if require: \path (0,300); %set diagram left start at 0, and has height of 300
		
		%Shape: Circle [id:dp33643947845182054] 
		\draw  [fill={rgb, 255:red, 155; green, 155; blue, 155 }  ,fill opacity=1 ] (40,45) .. controls (40,36.72) and (46.72,30) .. (55,30) .. controls (63.28,30) and (70,36.72) .. (70,45) .. controls (70,53.28) and (63.28,60) .. (55,60) .. controls (46.72,60) and (40,53.28) .. (40,45) -- cycle ;
		%Shape: Circle [id:dp0443477789203186] 
		\draw  [fill={rgb, 255:red, 155; green, 155; blue, 155 }  ,fill opacity=1 ] (120,45) .. controls (120,36.72) and (126.72,30) .. (135,30) .. controls (143.28,30) and (150,36.72) .. (150,45) .. controls (150,53.28) and (143.28,60) .. (135,60) .. controls (126.72,60) and (120,53.28) .. (120,45) -- cycle ;
		%Shape: Circle [id:dp9214696658988042] 
		\draw  [fill={rgb, 255:red, 155; green, 155; blue, 155 }  ,fill opacity=1 ] (81.5,108.5) .. controls (81.5,100.22) and (88.22,93.5) .. (96.5,93.5) .. controls (104.78,93.5) and (111.5,100.22) .. (111.5,108.5) .. controls (111.5,116.78) and (104.78,123.5) .. (96.5,123.5) .. controls (88.22,123.5) and (81.5,116.78) .. (81.5,108.5) -- cycle ;
		%Shape: Circle [id:dp13696565141434136] 
		\draw  [fill={rgb, 255:red, 155; green, 155; blue, 155 }  ,fill opacity=1 ] (170,115) .. controls (170,106.72) and (176.72,100) .. (185,100) .. controls (193.28,100) and (200,106.72) .. (200,115) .. controls (200,123.28) and (193.28,130) .. (185,130) .. controls (176.72,130) and (170,123.28) .. (170,115) -- cycle ;
		%Shape: Ellipse [id:dp12864986832842984] 
		\draw  [dash pattern={on 4.5pt off 4.5pt}] (82.3,131.88) .. controls (65.55,121.98) and (66.7,89.04) .. (84.86,58.31) .. controls (103.02,27.58) and (131.32,10.69) .. (148.06,20.58) .. controls (164.81,30.48) and (163.66,63.42) .. (145.5,94.15) .. controls (127.34,124.88) and (99.05,141.77) .. (82.3,131.88) -- cycle ;
		%Shape: Ellipse [id:dp1610627082678362] 
		\draw  [dash pattern={on 4.5pt off 4.5pt}] (43.7,64) .. controls (32.49,57.38) and (28.66,43.1) .. (35.16,32.12) .. controls (41.65,21.13) and (56,17.6) .. (67.21,24.22) .. controls (78.42,30.85) and (82.25,45.12) .. (75.75,56.11) .. controls (69.26,67.09) and (54.91,70.63) .. (43.7,64) -- cycle ;
		%Shape: Ellipse [id:dp12974777917337055] 
		\draw  [dash pattern={on 4.5pt off 4.5pt}] (172.7,135) .. controls (161.49,128.38) and (157.66,114.1) .. (164.16,103.12) .. controls (170.65,92.13) and (185,88.6) .. (196.21,95.22) .. controls (207.42,101.85) and (211.25,116.12) .. (204.75,127.11) .. controls (198.26,138.09) and (183.91,141.63) .. (172.7,135) -- cycle ;
		%Shape: Rectangle [id:dp6743641105872487] 
		\draw  [draw opacity=0][shading=_msh882t58,_3sybf9svc] (20,200.03) -- (20,190) -- (220,190) -- (220,200.03) -- cycle ;
		%Shape: Circle [id:dp41982950014639475] 
		\draw  [fill={rgb, 255:red, 226; green, 160; blue, 109 }  ,fill opacity=1 ] (310,45.25) .. controls (310,36.96) and (316.71,30.25) .. (325,30.25) .. controls (333.28,30.25) and (340,36.96) .. (340,45.25) .. controls (340,53.53) and (333.28,60.25) .. (325,60.25) .. controls (316.71,60.25) and (310,53.53) .. (310,45.25) -- cycle ;
		%Shape: Circle [id:dp8259512278629788] 
		\draw  [fill={rgb, 255:red, 80; green, 227; blue, 194 }  ,fill opacity=1 ] (390,45.25) .. controls (390,36.96) and (396.71,30.25) .. (405,30.25) .. controls (413.28,30.25) and (420,36.96) .. (420,45.25) .. controls (420,53.53) and (413.28,60.25) .. (405,60.25) .. controls (396.71,60.25) and (390,53.53) .. (390,45.25) -- cycle ;
		%Shape: Circle [id:dp07277556260180118] 
		\draw  [fill={rgb, 255:red, 245; green, 235; blue, 135 }  ,fill opacity=1 ] (351.5,108.75) .. controls (351.5,100.46) and (358.21,93.75) .. (366.5,93.75) .. controls (374.78,93.75) and (381.5,100.46) .. (381.5,108.75) .. controls (381.5,117.03) and (374.78,123.75) .. (366.5,123.75) .. controls (358.21,123.75) and (351.5,117.03) .. (351.5,108.75) -- cycle ;
		%Shape: Circle [id:dp6294612143917628] 
		\draw  [fill={rgb, 255:red, 189; green, 16; blue, 224 }  ,fill opacity=1 ] (440,115.25) .. controls (440,106.96) and (446.71,100.25) .. (455,100.25) .. controls (463.28,100.25) and (470,106.96) .. (470,115.25) .. controls (470,123.53) and (463.28,130.25) .. (455,130.25) .. controls (446.71,130.25) and (440,123.53) .. (440,115.25) -- cycle ;
		%Shape: Ellipse [id:dp7357850148564777] 
		\draw  [dash pattern={on 4.5pt off 4.5pt}] (352.29,132.12) .. controls (335.55,122.23) and (336.69,89.29) .. (354.85,58.56) .. controls (373.02,27.82) and (401.31,10.93) .. (418.06,20.83) .. controls (434.81,30.73) and (433.66,63.66) .. (415.5,94.39) .. controls (397.34,125.13) and (369.04,142.02) .. (352.29,132.12) -- cycle ;
		%Shape: Ellipse [id:dp5587385382701016] 
		\draw  [dash pattern={on 4.5pt off 4.5pt}] (313.7,64.25) .. controls (302.49,57.63) and (298.66,43.35) .. (305.15,32.36) .. controls (311.64,21.38) and (325.99,17.84) .. (337.21,24.47) .. controls (348.42,31.09) and (352.24,45.37) .. (345.75,56.35) .. controls (339.26,67.34) and (324.91,70.88) .. (313.7,64.25) -- cycle ;
		%Shape: Ellipse [id:dp4946869974787813] 
		\draw  [dash pattern={on 4.5pt off 4.5pt}] (442.7,135.25) .. controls (431.49,128.63) and (427.66,114.35) .. (434.15,103.36) .. controls (440.64,92.38) and (454.99,88.84) .. (466.21,95.47) .. controls (477.42,102.09) and (481.24,116.37) .. (474.75,127.35) .. controls (468.26,138.34) and (453.91,141.88) .. (442.7,135.25) -- cycle ;
		%Shape: Rectangle [id:dp7288880698372671] 
		\draw  [draw opacity=0][shading=_2kvz2ka7v,_s6rolseyi] (290,200.28) -- (290,190.25) -- (490,190.25) -- (490,200.28) -- cycle ;
		%Curve Lines [id:da9868746504728607] 
		\draw    (325,60.25) .. controls (301.72,83.52) and (314.22,158.9) .. (310.26,188.26) ;
		\draw [shift={(310,190)}, rotate = 279.63] [color={rgb, 255:red, 0; green, 0; blue, 0 }  ][line width=0.75]    (10.93,-3.29) .. controls (6.95,-1.4) and (3.31,-0.3) .. (0,0) .. controls (3.31,0.3) and (6.95,1.4) .. (10.93,3.29)   ;
		%Curve Lines [id:da6841947081825557] 
		\draw    (455,130.25) .. controls (431.72,153.52) and (470.16,164.07) .. (479.47,188.49) ;
		\draw [shift={(480,190)}, rotate = 252.07] [color={rgb, 255:red, 0; green, 0; blue, 0 }  ][line width=0.75]    (10.93,-3.29) .. controls (6.95,-1.4) and (3.31,-0.3) .. (0,0) .. controls (3.31,0.3) and (6.95,1.4) .. (10.93,3.29)   ;
		%Curve Lines [id:da5128158053664607] 
		\draw    (366.5,123.75) .. controls (376.05,142.61) and (358.96,160.76) .. (350.51,188.3) ;
		\draw [shift={(350,190)}, rotate = 286.14] [color={rgb, 255:red, 0; green, 0; blue, 0 }  ][line width=0.75]    (10.93,-3.29) .. controls (6.95,-1.4) and (3.31,-0.3) .. (0,0) .. controls (3.31,0.3) and (6.95,1.4) .. (10.93,3.29)   ;
		%Curve Lines [id:da01085903440845759] 
		\draw    (405,60.25) .. controls (406.22,79.11) and (416.33,156.32) .. (419.8,188.11) ;
		\draw [shift={(420,190)}, rotate = 263.92] [color={rgb, 255:red, 0; green, 0; blue, 0 }  ][line width=0.75]    (10.93,-3.29) .. controls (6.95,-1.4) and (3.31,-0.3) .. (0,0) .. controls (3.31,0.3) and (6.95,1.4) .. (10.93,3.29)   ;
		
		% Text Node
		\draw (49.25,35.5) node [anchor=north west][inner sep=0.75pt]   [align=left] {A};
		% Text Node
		\draw (91.25,99.5) node [anchor=north west][inner sep=0.75pt]   [align=left] {B};
		% Text Node
		\draw (129.25,35.5) node [anchor=north west][inner sep=0.75pt]   [align=left] {C};
		% Text Node
		\draw (178.75,106) node [anchor=north west][inner sep=0.75pt]   [align=left] {D};
		% Text Node
		\draw (319.25,35.75) node [anchor=north west][inner sep=0.75pt]   [align=left] {A};
		% Text Node
		\draw (361.25,99.75) node [anchor=north west][inner sep=0.75pt]   [align=left] {B};
		% Text Node
		\draw (399.25,35.75) node [anchor=north west][inner sep=0.75pt]   [align=left] {C};
		% Text Node
		\draw (448.75,106.25) node [anchor=north west][inner sep=0.75pt]   [align=left] {D};
		
		
	\end{tikzpicture}
\end{figure}
	
	\FloatBarrier
	
	Now consider the following diagrams on which we have defined two explicit topologies on the color space by providing the partitions. It is easy to check that the with the topology on the right subplot, the coloring map is not continuous, but with the topology on the left subplot the coloring map is continuous.
	
	
	\begin{figure}[h!]
	\centering
	
	
	% Gradient Info
	
	\tikzset {_jupf8dp9u/.code = {\pgfsetadditionalshadetransform{ \pgftransformshift{\pgfpoint{0 bp } { 0 bp }  }  \pgftransformrotate{0 }  \pgftransformscale{2 }  }}}
	\pgfdeclarehorizontalshading{_oxikznzed}{150bp}{rgb(0bp)=(0.82,0.01,0.11);
		rgb(37.5bp)=(0.82,0.01,0.11);
		rgb(43.973214285714285bp)=(0.97,0.91,0.11);
		rgb(49.24107142857143bp)=(0.72,0.91,0.53);
		rgb(55.58035714285714bp)=(0.11,0.97,0.94);
		rgb(62.00892857142857bp)=(0.8,0.29,0.89);
		rgb(100bp)=(0.8,0.29,0.89)}
	
	% Gradient Info
	
	\tikzset {_rrrrdzg1t/.code = {\pgfsetadditionalshadetransform{ \pgftransformshift{\pgfpoint{0 bp } { 0 bp }  }  \pgftransformrotate{0 }  \pgftransformscale{2 }  }}}
	\pgfdeclarehorizontalshading{_f2np8rf0m}{150bp}{rgb(0bp)=(0.82,0.01,0.11);
		rgb(37.5bp)=(0.82,0.01,0.11);
		rgb(43.973214285714285bp)=(0.97,0.91,0.11);
		rgb(49.24107142857143bp)=(0.72,0.91,0.53);
		rgb(55.58035714285714bp)=(0.11,0.97,0.94);
		rgb(62.00892857142857bp)=(0.8,0.29,0.89);
		rgb(100bp)=(0.8,0.29,0.89)}
	\tikzset{every picture/.style={line width=0.75pt}} %set default line width to 0.75pt        
	
	\begin{tikzpicture}[x=0.75pt,y=0.75pt,yscale=-1,xscale=1]
		%uncomment if require: \path (0,300); %set diagram left start at 0, and has height of 300
		
		%Shape: Circle [id:dp8686502462251218] 
		\draw  [fill={rgb, 255:red, 226; green, 160; blue, 109 }  ,fill opacity=1 ] (70,64.97) .. controls (70,56.68) and (76.72,49.97) .. (85,49.97) .. controls (93.28,49.97) and (100,56.68) .. (100,64.97) .. controls (100,73.25) and (93.28,79.97) .. (85,79.97) .. controls (76.72,79.97) and (70,73.25) .. (70,64.97) -- cycle ;
		%Shape: Circle [id:dp5552463587832126] 
		\draw  [fill={rgb, 255:red, 80; green, 227; blue, 194 }  ,fill opacity=1 ] (150,64.97) .. controls (150,56.68) and (156.72,49.97) .. (165,49.97) .. controls (173.28,49.97) and (180,56.68) .. (180,64.97) .. controls (180,73.25) and (173.28,79.97) .. (165,79.97) .. controls (156.72,79.97) and (150,73.25) .. (150,64.97) -- cycle ;
		%Shape: Circle [id:dp21656446486952063] 
		\draw  [fill={rgb, 255:red, 245; green, 235; blue, 135 }  ,fill opacity=1 ] (111.5,128.47) .. controls (111.5,120.18) and (118.22,113.47) .. (126.5,113.47) .. controls (134.78,113.47) and (141.5,120.18) .. (141.5,128.47) .. controls (141.5,136.75) and (134.78,143.47) .. (126.5,143.47) .. controls (118.22,143.47) and (111.5,136.75) .. (111.5,128.47) -- cycle ;
		%Shape: Circle [id:dp6549202348048928] 
		\draw  [fill={rgb, 255:red, 189; green, 16; blue, 224 }  ,fill opacity=1 ] (200,134.97) .. controls (200,126.68) and (206.72,119.97) .. (215,119.97) .. controls (223.28,119.97) and (230,126.68) .. (230,134.97) .. controls (230,143.25) and (223.28,149.97) .. (215,149.97) .. controls (206.72,149.97) and (200,143.25) .. (200,134.97) -- cycle ;
		%Shape: Ellipse [id:dp6213542116896975] 
		\draw  [dash pattern={on 4.5pt off 4.5pt}] (112.3,151.84) .. controls (95.55,141.95) and (96.7,109.01) .. (114.86,78.28) .. controls (133.02,47.55) and (161.32,30.66) .. (178.06,40.55) .. controls (194.81,50.45) and (193.66,83.38) .. (175.5,114.12) .. controls (157.34,144.85) and (129.05,161.74) .. (112.3,151.84) -- cycle ;
		%Shape: Ellipse [id:dp18690290078640293] 
		\draw  [dash pattern={on 4.5pt off 4.5pt}] (73.7,83.97) .. controls (62.49,77.35) and (58.66,63.07) .. (65.16,52.09) .. controls (71.65,41.1) and (86,37.56) .. (97.21,44.19) .. controls (108.42,50.81) and (112.25,65.09) .. (105.75,76.08) .. controls (99.26,87.06) and (84.91,90.6) .. (73.7,83.97) -- cycle ;
		%Shape: Ellipse [id:dp10224302620382786] 
		\draw  [dash pattern={on 4.5pt off 4.5pt}] (202.7,154.97) .. controls (191.49,148.35) and (187.66,134.07) .. (194.16,123.09) .. controls (200.65,112.1) and (215,108.56) .. (226.21,115.19) .. controls (237.42,121.81) and (241.25,136.09) .. (234.75,147.08) .. controls (228.26,158.06) and (213.91,161.6) .. (202.7,154.97) -- cycle ;
		%Shape: Rectangle [id:dp3411009771047675] 
		\draw  [draw opacity=0][shading=_oxikznzed,_jupf8dp9u] (50,220) -- (50,209.97) -- (250,209.97) -- (250,220) -- cycle ;
		%Curve Lines [id:da09977618835584368] 
		\draw    (85,79.97) .. controls (61.73,103.25) and (74.23,178.62) .. (70.27,207.98) ;
		\draw [shift={(70,209.72)}, rotate = 279.63] [color={rgb, 255:red, 0; green, 0; blue, 0 }  ][line width=0.75]    (10.93,-3.29) .. controls (6.95,-1.4) and (3.31,-0.3) .. (0,0) .. controls (3.31,0.3) and (6.95,1.4) .. (10.93,3.29)   ;
		%Curve Lines [id:da6318930308826853] 
		\draw    (215,149.97) .. controls (191.73,173.25) and (230.16,183.8) .. (239.47,208.21) ;
		\draw [shift={(240,209.72)}, rotate = 252.07] [color={rgb, 255:red, 0; green, 0; blue, 0 }  ][line width=0.75]    (10.93,-3.29) .. controls (6.95,-1.4) and (3.31,-0.3) .. (0,0) .. controls (3.31,0.3) and (6.95,1.4) .. (10.93,3.29)   ;
		%Curve Lines [id:da09627170903749671] 
		\draw    (126.5,143.47) .. controls (136.06,162.34) and (118.96,180.48) .. (110.51,208.02) ;
		\draw [shift={(110,209.72)}, rotate = 286.14] [color={rgb, 255:red, 0; green, 0; blue, 0 }  ][line width=0.75]    (10.93,-3.29) .. controls (6.95,-1.4) and (3.31,-0.3) .. (0,0) .. controls (3.31,0.3) and (6.95,1.4) .. (10.93,3.29)   ;
		%Curve Lines [id:da7334611428136537] 
		\draw    (165,79.97) .. controls (166.23,98.84) and (176.34,176.05) .. (179.8,207.83) ;
		\draw [shift={(180,209.72)}, rotate = 263.92] [color={rgb, 255:red, 0; green, 0; blue, 0 }  ][line width=0.75]    (10.93,-3.29) .. controls (6.95,-1.4) and (3.31,-0.3) .. (0,0) .. controls (3.31,0.3) and (6.95,1.4) .. (10.93,3.29)   ;
		%Shape: Boxed Bezier Curve [id:dp6591612969439882] 
		\draw  [dash pattern={on 1.5pt off 1.5pt}]  (102.5,230.5) .. controls (97.25,221.5) and (96.25,211.5) .. (102.5,200.5) ;
		%Shape: Boxed Bezier Curve [id:dp22021352768956315] 
		\draw  [dash pattern={on 1.5pt off 1.5pt}]  (92.05,200.52) .. controls (97.38,209.47) and (98.46,219.47) .. (92.31,230.52) ;
		%Shape: Boxed Bezier Curve [id:dp17420743375708936] 
		\draw  [dash pattern={on 1.5pt off 1.5pt}]  (190.93,200.04) .. controls (196.25,208.99) and (197.34,218.98) .. (191.18,230.04) ;
		%Shape: Boxed Bezier Curve [id:dp21373559969089517] 
		\draw  [dash pattern={on 1.5pt off 1.5pt}]  (200.57,229.49) .. controls (195.2,220.56) and (194.07,210.57) .. (200.18,199.49) ;
		%Shape: Circle [id:dp2071167052111076] 
		\draw  [fill={rgb, 255:red, 226; green, 160; blue, 109 }  ,fill opacity=1 ] (350,64.45) .. controls (350,56.17) and (356.71,49.45) .. (365,49.45) .. controls (373.28,49.45) and (380,56.17) .. (380,64.45) .. controls (380,72.73) and (373.28,79.45) .. (365,79.45) .. controls (356.71,79.45) and (350,72.73) .. (350,64.45) -- cycle ;
		%Shape: Circle [id:dp5086147230828475] 
		\draw  [fill={rgb, 255:red, 80; green, 227; blue, 194 }  ,fill opacity=1 ] (430,64.45) .. controls (430,56.17) and (436.71,49.45) .. (445,49.45) .. controls (453.28,49.45) and (460,56.17) .. (460,64.45) .. controls (460,72.73) and (453.28,79.45) .. (445,79.45) .. controls (436.71,79.45) and (430,72.73) .. (430,64.45) -- cycle ;
		%Shape: Circle [id:dp1132042015881487] 
		\draw  [fill={rgb, 255:red, 245; green, 235; blue, 135 }  ,fill opacity=1 ] (391.5,127.95) .. controls (391.5,119.67) and (398.21,112.95) .. (406.5,112.95) .. controls (414.78,112.95) and (421.5,119.67) .. (421.5,127.95) .. controls (421.5,136.23) and (414.78,142.95) .. (406.5,142.95) .. controls (398.21,142.95) and (391.5,136.23) .. (391.5,127.95) -- cycle ;
		%Shape: Circle [id:dp5225952989546916] 
		\draw  [fill={rgb, 255:red, 189; green, 16; blue, 224 }  ,fill opacity=1 ] (480,134.45) .. controls (480,126.17) and (486.71,119.45) .. (495,119.45) .. controls (503.28,119.45) and (510,126.17) .. (510,134.45) .. controls (510,142.73) and (503.28,149.45) .. (495,149.45) .. controls (486.71,149.45) and (480,142.73) .. (480,134.45) -- cycle ;
		%Shape: Ellipse [id:dp008210969696711645] 
		\draw  [dash pattern={on 4.5pt off 4.5pt}] (392.29,151.33) .. controls (375.55,141.43) and (376.69,108.49) .. (394.85,77.76) .. controls (413.02,47.03) and (441.31,30.14) .. (458.06,40.03) .. controls (474.81,49.93) and (473.66,82.87) .. (455.5,113.6) .. controls (437.34,144.33) and (409.04,161.22) .. (392.29,151.33) -- cycle ;
		%Shape: Ellipse [id:dp15440925630698055] 
		\draw  [dash pattern={on 4.5pt off 4.5pt}] (353.7,83.45) .. controls (342.49,76.83) and (338.66,62.55) .. (345.15,51.57) .. controls (351.64,40.58) and (365.99,37.05) .. (377.21,43.67) .. controls (388.42,50.3) and (392.24,64.57) .. (385.75,75.56) .. controls (379.26,86.54) and (364.91,90.08) .. (353.7,83.45) -- cycle ;
		%Shape: Ellipse [id:dp5043296278196521] 
		\draw  [dash pattern={on 4.5pt off 4.5pt}] (482.7,154.45) .. controls (471.49,147.83) and (467.66,133.55) .. (474.15,122.57) .. controls (480.64,111.58) and (494.99,108.05) .. (506.21,114.67) .. controls (517.42,121.3) and (521.24,135.57) .. (514.75,146.56) .. controls (508.26,157.54) and (493.91,161.08) .. (482.7,154.45) -- cycle ;
		%Shape: Rectangle [id:dp7051919783039517] 
		\draw  [draw opacity=0][shading=_f2np8rf0m,_rrrrdzg1t] (330,219.48) -- (330,209.45) -- (530,209.45) -- (530,219.48) -- cycle ;
		%Curve Lines [id:da37843887018525946] 
		\draw    (365,79.45) .. controls (341.72,102.73) and (354.22,178.11) .. (350.26,207.46) ;
		\draw [shift={(350,209.2)}, rotate = 279.63] [color={rgb, 255:red, 0; green, 0; blue, 0 }  ][line width=0.75]    (10.93,-3.29) .. controls (6.95,-1.4) and (3.31,-0.3) .. (0,0) .. controls (3.31,0.3) and (6.95,1.4) .. (10.93,3.29)   ;
		%Curve Lines [id:da9854017683047185] 
		\draw    (495,149.45) .. controls (471.72,172.73) and (510.16,183.28) .. (519.47,207.69) ;
		\draw [shift={(520,209.2)}, rotate = 252.07] [color={rgb, 255:red, 0; green, 0; blue, 0 }  ][line width=0.75]    (10.93,-3.29) .. controls (6.95,-1.4) and (3.31,-0.3) .. (0,0) .. controls (3.31,0.3) and (6.95,1.4) .. (10.93,3.29)   ;
		%Curve Lines [id:da13498803888212696] 
		\draw    (406.5,142.95) .. controls (416.05,161.82) and (398.96,179.96) .. (390.51,207.51) ;
		\draw [shift={(390,209.2)}, rotate = 286.14] [color={rgb, 255:red, 0; green, 0; blue, 0 }  ][line width=0.75]    (10.93,-3.29) .. controls (6.95,-1.4) and (3.31,-0.3) .. (0,0) .. controls (3.31,0.3) and (6.95,1.4) .. (10.93,3.29)   ;
		%Curve Lines [id:da046756128412226206] 
		\draw    (445,79.45) .. controls (446.22,98.32) and (456.33,175.53) .. (459.8,207.32) ;
		\draw [shift={(460,209.2)}, rotate = 263.92] [color={rgb, 255:red, 0; green, 0; blue, 0 }  ][line width=0.75]    (10.93,-3.29) .. controls (6.95,-1.4) and (3.31,-0.3) .. (0,0) .. controls (3.31,0.3) and (6.95,1.4) .. (10.93,3.29)   ;
		%Shape: Boxed Bezier Curve [id:dp7139533309729651] 
		\draw  [dash pattern={on 1.5pt off 1.5pt}]  (434.32,230) .. controls (429.07,221) and (428.07,211) .. (434.32,200) ;
		%Shape: Boxed Bezier Curve [id:dp4178108981730697] 
		\draw  [dash pattern={on 1.5pt off 1.5pt}]  (425.43,200.04) .. controls (430.75,208.99) and (431.84,218.98) .. (425.68,230.04) ;
		%Shape: Boxed Bezier Curve [id:dp8000489551931103] 
		\draw  [dash pattern={on 1.5pt off 1.5pt}]  (470.92,199.52) .. controls (476.25,208.48) and (477.33,218.47) .. (471.18,229.52) ;
		%Shape: Boxed Bezier Curve [id:dp738862646285132] 
		\draw  [dash pattern={on 1.5pt off 1.5pt}]  (480.56,228.97) .. controls (475.2,220.04) and (474.07,210.05) .. (480.18,198.97) ;
		
		% Text Node
		\draw (79.25,55.47) node [anchor=north west][inner sep=0.75pt]   [align=left] {A};
		% Text Node
		\draw (121.25,119.47) node [anchor=north west][inner sep=0.75pt]   [align=left] {B};
		% Text Node
		\draw (159.25,55.47) node [anchor=north west][inner sep=0.75pt]   [align=left] {C};
		% Text Node
		\draw (208.75,125.97) node [anchor=north west][inner sep=0.75pt]   [align=left] {D};
		% Text Node
		\draw (359.25,54.95) node [anchor=north west][inner sep=0.75pt]   [align=left] {A};
		% Text Node
		\draw (401.25,118.95) node [anchor=north west][inner sep=0.75pt]   [align=left] {B};
		% Text Node
		\draw (439.25,54.95) node [anchor=north west][inner sep=0.75pt]   [align=left] {C};
		% Text Node
		\draw (488.75,125.45) node [anchor=north west][inner sep=0.75pt]   [align=left] {D};
		
		
	\end{tikzpicture}
\end{figure}
	
	
	
	
\end{document}