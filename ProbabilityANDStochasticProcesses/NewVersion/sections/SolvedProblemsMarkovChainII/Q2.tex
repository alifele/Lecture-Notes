\begin{problem}
	There are 6 coins on a table, each showing heads (H) or tails (T). In each step we 
	\begin{itemize}
		\item Select uniformly one of the coins. 
		\item If it is heads, toss it and replace on the table (with random side).
		\item If it sis tails, toss it. If it comes up heads, leave it at that. If it comes up tails, toss it a second time, and leave the result as it is.
		Let $X_n$ be the number of heads showing after $n$ such steps. Answer the following questions
		\begin{enumerate}[(a)]
			\item Determine the transition probabilities for this Markov chain.
			\item Draw the transition diagram and write the transition matrix.
			\item What is $\prob(X_2 = 4| X_0=5)$?
		\end{enumerate}
	\end{itemize}
\end{problem}
\begin{solution}
	\begin{enumerate}[(a)]
		\item To compute the transition probabilities, we need to perform the first step analysis. Let the events \[I = \set{X_1 = a+1},\qquad S = \set{X_1 = a},\qquad D = \set{X_1 = a-1},\]
		where $0 \leq a \leq 6$ is the number of heads. So to compute the transition probabilities, we need to compute
		\[ P(a,a+1) = \prob_a(I), \qquad P(a,a) = \prob_a(S), \qquad P(a,a-1) = \prob_a(D). \]
		We start with $\prob_a(I)$. Let $ST$ be the event where the selected coin is tails, and $SH$ be the event where the selected coin is heads. These two events are disjoint and the probability of their union is 1, thus
		\[ \prob_a(I) = \underbrace{\prob_a(I|SH)}_{\text{see Eq (2.I.1)}}\underbrace{\prob_a(SH)}_{\frac{a}{6}} + \underbrace{\prob_a(I|ST)}_{\text{see Eq (2.I.2)}}\underbrace{\prob_a(ST)}_{\frac{6-a}{6}}. \tag{2.I}\]
		Note that if we start with $a$ coins heads, then the chance we choose a random coin from the table and find it heads is $\frac{a}{6}$, hence $\prob_a(SH) = \frac{a}{6}$, and $\prob_a(ST) = \frac{6-a}{6}$. Now we need to expand the remaining terms with appropriate conditioning. Let $TT$ be the event where we toss a coin and find it tails and $TH$ be the event where we toss a coin and find it heads. Thus we can write
		\[ \prob_a(I|SH) = \underbrace{\prob_a(I|SH,TH)}_{0}\underbrace{\prob_a(TS)}_\frac{1}{2} + \underbrace{\prob_a(I|SH,TT)}_{0}\underbrace{\prob_a(TT)}_\frac{1}{2}. \tag{2.I.1}  \]
		Note that $\prob_a(TT) = \prob_a(TH) = \frac{1}{2}$, since the coin tossing is fair. Also, note that $\prob_a(I|SH,TH)=\prob_a(I|SH,TT)=0$ since if we select a heads, and then toss it, finding it either heads or tails will not increase the total number of heads on the table. Similarly, for the other term in $(2.1)$ we have
		\[ \prob_a(I|ST) = \underbrace{\prob_a(I|ST,TH)}_{1}\underbrace{\prob_a(TH)}_{\frac{1}{2}} + \underbrace{\prob_a(I|ST,TT)}_\text{see Eq (2.I.3)}\underbrace{\prob_a(TT)}_{\frac{1}{2}}. \tag{2.I.2} \]
		Now we need to expand the remaining terms in the equation above.
		\[ \prob_a(I|ST,TT) = \underbrace{\prob_a(I|ST,TT,TH)}_{1}\prob_a(TH) + \underbrace{\prob_a(I|ST,TT,TT)}_{0}\prob_a(TT) = \frac{1}{2}. \tag{2.I.3}  \]
		Putting all together we can write
		\[ \boxed{P(a,a+1) = \prob_a(I) = \frac{6-a}{8}}. \]
		Similarly, we can compute other transition probabilities. For instance for $\prob_a(S)$ we can write
		\[ \prob_a(S) = \underbrace{\prob_a(S|SH)}_{\text{see Eq (2.S.1)}}\underbrace{\prob_a(SH)}_{\frac{a}{6}} + \underbrace{\prob_a(S|ST)}_{\text{see Eq (2.S.2)}}\underbrace{\prob_a(ST)}_{\frac{6-a}{6}}. \tag{2.S}\]
		and for the remaining terms we can write
		\[ \prob_a(S|SH) = \underbrace{\prob_a(S|SH,TH)}_{1}\underbrace{\prob_a(TS)}_\frac{1}{2} + \underbrace{\prob_a(S|SH,TT)}_{0}\underbrace{\prob_a(TT)}_\frac{1}{2}, \tag{2.S.1}  \]
		and
		\[ \prob_a(S|ST) = \underbrace{\prob_a(S|ST,TH)}_{0}\underbrace{\prob_a(TH)}_{\frac{1}{2}} + \underbrace{\prob_a(S|ST,TT)}_\text{see Eq (2.S.3)}\underbrace{\prob_a(TT)}_{\frac{1}{2}}. \tag{2.S.2} \]
		And for the remaining term above
		\[ \prob_a(S|ST,TT) = \underbrace{\prob_a(S|ST,TT,TH)}_{0}\prob_a(TH) + \underbrace{\prob_a(S|ST,TT,TT)}_{1}\prob_a(TT) = \frac{1}{2}. \tag{2.S.3}  \]
		and by putting all together we will get
		\[\boxed{ P(a,a) =  \prob_a(S) = \frac{6+a}{24}}. \]
		Finally, since $\prob_a(I\cup S \cup D) = 1$, and $I,S,D$ are mutually disjoint, we can write 
		\[ \prob_a(D) = 1 - (\prob_a(I) + \prob_a(S)), \] 
		hence
		\[ \boxed{P(a,a-1) = \prob_a(D) = \frac{a}{12}}. \]
		so the transition probabilities are as calculated.
		
		\item The transition diagram is plotted below. 
		
		\begin{center}
			\begin{tikzpicture}[->,>=stealth',shorten >=1pt,auto,node distance=1.9cm,
				semithick,scale=0.4]
				\tikzstyle{every state}=[circle, draw, fill=orange!50,
				inner sep=0pt, minimum width=10pt]
				
				\node[state] (A)              {$0$};
				\node[state]         (B) [right of=A] {$1$};
				\node[state]         (C) [right of=B] {$2$};
				\node[state]         (D) [right of=C] {$3$};
				\node[state]         (E) [right of=D] {$4$};
				\node[state]         (F) [right of=E] {$5$};
				\node[state]         (G) [right of=F] {$6$};
				
				\path (A) edge [bend left] node {$p_{01}$} (B)
				(B) edge [bend left] node {$p_{10}$} (A)
				(A) edge [loop left] node {$p_{00}$} (A)
 				
				(B) edge [bend left] node {$p_{12}$} (C)
				(B) edge [loop above] node {$p_{11}$} (C)
				(C) edge [bend left] node {$p_{21}$} (B)
				
				(C) edge [bend left] node {$p_{23}$} (D)
				(C) edge [loop below] node {$p_{22}$} (D)
				(D) edge [bend left] node {$p_{32}$} (C)
				
				(D) edge [bend left] node {$p_{34}$} (E)
				(D) edge [loop above] node {$p_{33}$} (E)
				(E) edge [bend left] node {$p_{43}$} (D)
				
				(E) edge [bend left] node {$p_{45}$} (F)
				(E) edge [loop below] node {$p_{44}$} (E)
				(F) edge [bend left] node {$p_{54}$} (E)
				
				(F) edge [bend left] node {$p_{56}$} (G)
				(F) edge [loop above] node {$p_{55}$} (E)
				(G) edge [bend left] node {$p_{65}$} (F)
				(G) edge [loop right] node {$p_{66}$} (G);
			\end{tikzpicture}
		\end{center}
		And the transition matrix is
		\[  M =
		\begin{pmatrix}
			1/4 & 3/4 & 0 & 0 & 0 & 0 & 0 \\
			1/12 & 7/24 & 5/8 & 0 & 0 & 0 & 0 \\
			0 & 1/6 & 1/3 & 1/2 & 0 & 0 & 0 \\
			0 & 0 & 1/4 & 3/8 & 3/8 & 0 & 0 \\
			0 & 0 & 0 & 1/3 & 5/12 & 1/4 & 0 \\
			0 & 0 & 0 & 0 & 5/12 & 11/24 & 1/8 \\
			0 & 0 & 0 & 0 & 0 & 1/2 & 1/2
		\end{pmatrix}
		 \]
		 
		 \item $\prob(X_2 = 4|X_0=5)$ is the second transition probability $P_2(5,4)$. To compute this, we need to fine the element in the 6-th row and 5-th column in the $M^2$ matrix, which is basically the inner product between the vectors formed by the 6-th row and the 5-th column.
		 \[ P_2(5,4) = (\frac{5}{12})^2 + \frac{11}{24}\cdot\frac{5}{12} = \frac{35}{96} \]
		 which after simplification becomes
		 \[ \boxed{P_2(5,4) = \frac{35}{96}}. \]
		\qed

		
		
		
	\end{enumerate}
	
\end{solution}