\chapter{Probability by Le Gall}

\section{Integration of Measurable Functions}

The following properties become very useful in the following chapters. 
\begin{proposition}
	Let $ f $ be a \emph{non-negative} measurable function.
	\begin{enumerate}[(i)]
		\item We have
		\[ \int f d\mu = 0 \biImp f = 0,\ \mu\ a.e. \]
	\end{enumerate}
\end{proposition}
\begin{proof}
	For the proof we will have
	\begin{enumerate}[(i)]
		\item Define 
		\[ A_n = \set{f \geq 1/n}. \]
		Then by the Markov inequality
		\[ \mu(A_n) = \mu(\set{f \geq 1/n}) \leq \frac{\int f\d\mu}{1/n} = 0. \]
		Also observe that $ \set{f > 0} = \bigcup_n \set{f \geq 1/n} = \bigcup_n A_n $. Thus
		\[ \mu(\bigcup_n A_n) \leq \sum_n \mu(A_n) = 0. \]
		This implies $ \mu(\set{f>0}) = 0 $, thus $ f = 0 $ almost everywhere.
		
	\end{enumerate}
\end{proof}
\begin{remark}
	For proof for part (i) in the proposition above, I also have the following proof using the conditional expectation. But I don't know how that translates to the integrals. Note that we have $ \E{X} = 0 $ and we want to show $ \prob(X = 0) = 1 $. We can write
	\begin{align*}
		0 = \E{X} &= \E{X\I_{X=0} + X\I_{X\neq 0}} = \E{X\I_{X=0}} + \E{X\I_{X\neq0}} \\
		&= \underbrace{\E{X|X=0}\E{\I_{X=0}}}_{0} + \E{X|X\neq 0} \E{\I_{X\neq 0}} \\
		&= \E{X|X\neq 0} \E{\I_{X\neq 0}}.
	\end{align*}
	Observe that we have $ \E{X|X\neq 0} > 0 $ since $ X $ is a non-negative random variable. Thus we need to have $ \E{I_{X\neq 0}} = \prob(X\neq0) = 1 $.
\end{remark}