\chapter{Lamplighter Group }


\begin{summary}[Solution to Exercise 1 Page 312 of GeomOfficeHour]
	In this summary box, I will derive a formula to multiply two elements of the lamplighter group represented in a tuple, where the first entry is the set of all ``on'' lamps, and the second entry is the current position of the lamplighter. Let $ g,h \in L_2 $ with
	\[ g = (T_g, p_g), \qquad h = (T_h, p_h), \]
	where $ T_g, T_h $ are the set of all on lamps for element $ g $ and $ h $ respectively, and $ p_g, p_h $ denote the current position of the lamplighter in each group element. Then 
	\[ g+h = (T_g \Delta (T_h+p_g), p_g+p_h). \]
	For instance if
	\[ g=(\set{-6,-1,4,5,6},-2), \qquad h = (\set{2,6}, 6), \]
	then we have
	\[ g+h = (\set{-6,-1,4,5,6}\Delta\set{0,4}, 4) = (\set{-6,-1,0,5,6},4), \]
	which matches with the final answer for the worked example in the textbook.
\end{summary}