\chapter{Review of the Category of Groups}


\section{Summary}
\subsection{Review of basic groups}
\begin{summary}[Random Notes]
	$  $\\
	\begin{enumerate}[(i)]
		\item In example 2.1.11 L\"{o}h, we discuss the notion of the isometry groups. One good example for such group is the group of unitary matrices. The reason is that these matrices preserve the inner product, and hence the norm induced by the inner product. Indeed $ \inv{U} = U\dagger $, so
		\[ \norm{U a} = \langle Ua, Ua \rangle^{1/2} = \langle UU^\dagger a, a\rangle^{1/2} = \langle a,a \rangle =  \norm{a}. \]
		
		
		\item Dihedral group is the symmetry group (isometry group) of regular polygon. A bit of interpretation is needed to clarify we mean here. When we say, the symmetry group of a regular polygon, implicitly, we consider the regular polygon as a metric space. And one natural question is, what is the metric? We can let the metric be the one inherited by restricting the Euclidean metric to the polygon (assuming it is embedded in $ \R^n $), or we can use other notions of metric, like the word metric (i.e. minimum number of edges between two nodes of interest). 
		
		
		\item In proposition 2.1.19, part (2) in L\"{o}h, we prove that every group $ G $ is isomorphic to $ \Aut(X) $ for some object $ X $ in some category $ C $. When I was thinking on this problem, initially, I thought the following construction would also do the job: Let $ C $ be the category of all sets, and $ X $ be the underlying set of $ G $. But this is not true. I.e. $ \Aut(X) $ is not the same as $ G $. But rather $ G $ is a subgroup of $ \Aut(X) $. This is already discussed in Proposition 2.1.19 (Cayley's theorem) where we showed that every group is some subgroup of some symmetric group. $ \Aut(X) $ in the category of all sets is the same as $ S_X $, i.e. the symmetric group of $ X $.
		
		\item Automorphsism group is one way to associate meaningful groups to interesting objects.
		
		
	
	\end{enumerate}
\end{summary}

\begin{observation}
	Proposition 2.1.19 states that $ \Aut(X) $ for any object $ X $ in any category is a group. This group has special names in different categories as summarized below:
	\begin{enumerate}[(a)]
		\item Let $ C $ be the category of all sets, where the morphisms are the set-theoretic mappings. Then $ \Aut(X) $ is the Symmetric group $ S_X $.
		\item In the category of all metric spaces, where the morphisms are the isometric embeddings, then for any object $ X $, $ \Aut(X) $ is the symmetry (isometry) group of $ X $.
	\end{enumerate}
\end{observation}


\begin{summary}[Interesting examples]
	\begin{enumerate}[(a)]
		\item $ \SLNR \lhd \GLNR $. So $ \GLNR/\SLNR $ is a group, and in fact
		\[ \GLNR/\SLNR \cong (\R,\times). \]
		Because $ \det(g\cdot \SLNR) = \det(g) \in \R $. Having an isomorphism between $ \GLNR/\SLNR $ and $ (\R,\times) $ means that given any coset in $ \GLNR/\SLNR $ we can come up with a real number, and vise versa in a way that the group structure is preserved. I.e. if we multiply two cosets and get a third one, the third one maps to a real number that is equal to the multiplication of the numbers to which the first two cosets are mapped.
	\end{enumerate}
\end{summary}

\begin{observation}[Important characterization of equality of two cosets]
	Trying to see if an operation is well defined (specially over the quotient structure of groups or vector spaces), can sometimes be very confusing. There is a very easy characterization of the equality of two cosets, that can be very useful in showing the well-definedness of definition. We will describe it in two settings: vector spaces, and groups.
	\begin{enumerate}[(a)]
		\item 
		Let $ W \subset V $ be a subspace of the vector space $ V $. Then for $ u,v \in V $ two cosets $ u + W $ and $ v+W $ are equal if and only if $ u-v\in W $ (or equivalently $ v-w\in W, $ as $ W $ is a subspace). In other words
		\[ u + W = v + W \quad \text{iff} \quad u - v \in W. \]
		
		\item Let $ H \subset G $ by \emph{any} subgroup of the group $ G $. Then for $ a,b \in G $, two cosets $ aH $ and $ bH $ are equal if and only if $ \inv{a} b \in H $ (or equivalently $ \inv{b}a \in H $, as $ H $ is a subgroup). In simple words
		\[ aH = bH \quad \text{iff}\quad  \inv{a}b \in H. \]
	\end{enumerate}
	\begin{proof}
		We will show the proof for the case of groups. For the forward direction we assume $ aH = bH $. So $ \exists h_1,h_2 \in H $ such that $ ah_1 = b h_2 $. We can write this as $ \inv{a} b = h_1\inv{h_2} $. Since $ H $ is a subgroup, then $ \inv{h_2}\in H $ and $ h_1\inv{h_2}\in  H $. This implies that $ \inv{a}b \in H $. For the converse, we assume $ \inv{a}b  \in H$. Since $ H $ is a subgroup, then $ \inv{\inv{a}b} = \inv{b}a \in H $ as well. Let $ ah_1 \in aH $ for some $ h_1 \in H $. Then $ ah_1 = b\inv{b}ah_1 = bh'h_1 $ for some $ h'\in H $. So $ ah_1 \in bH $. Let $ bh_2 \in bH $. Then $ a\inv{a}bh_2 = ah''h_2 \in aH $. So $ bh_2 \in aH $.
	\end{proof}
	
	The characterization above helps us to quickly see why the quotient multiplication for $ G/N $ is well-defined when $ N $ is a normal subgroup of $ G $, i.e. $ N \lhd G $. Let $ a_1N = a_2N $ and $ b_1N = b_2N $. To show the well-definedness of the multiplication we need to show $ (a_1b_1)N = (a_2b_2)N $, or equivalently, from the characterization above $ (a_1b_1)^{-1} (a_2b_2)\in N $. To see this we start with the characterization of $ a_1N = a_2N $ and $ b_1N = b_2N $. These will result in $ \inv{a_1}a_2 \in N $ and $ \inv{b_1}b_2 \in N $ respectively. Indeed
	\[ (a_1b_1)^{-1}(a_2b_2) = \inv{b_1}\inv{a_1}a_2b_2 = b_1^{-1} n_1 b_2 = b_1^{-1} n_1 b_1\inv{b_1} b_2 = n_2 \inv{b_1}b_2 \in N.   \]
\end{observation}


\begin{observation}[An explicit example of not well-defined multiplication for quotient with non-normal subgroups]
	We construct an explicit example. Consider $ S_3 $, the permutation group of $ \set{1,2,3} $ as
	\[ \sigma_0 = (1,2,3),\ \sigma_1 = (1,3,2),\ \sigma_2=(2,1,3),\ \sigma_3 =(2,3,1),\ \sigma_4 = (3,1,2),\ \sigma_5 = (3,2,1). \]
	The multiplication table will be
	\[ 
	\begin{matrix}
		\cdot\ \ \vline & \sigma_0 & \sigma_1 & \sigma_2 & \sigma_3 & \sigma_4 & \sigma_5 \\
		\hline
		\sigma_0 \vline & \sigma_0 & \sigma_1 & \sigma_2 & \sigma_3 & \sigma_4 & \sigma_5 \\
		\sigma_1 \vline & \sigma_1 & \sigma_0 & \sigma_4 & \sigma_5 & \sigma_2 & \sigma_3 \\
		\sigma_2 \vline & \sigma_2 & \sigma_3 & \sigma_0 & \sigma_1 & \sigma_5 & \sigma_4 \\
		\sigma_3 \vline & \sigma_3 & \sigma_2 & \sigma_5 & \sigma_4 & \sigma_0 & \sigma_1 \\
		\sigma_4 \vline & \sigma_4 & \sigma_5 & \sigma_1 & \sigma_0 & \sigma_3 & \sigma_2 \\
		\sigma_5 \vline & \sigma_5 & \sigma_4 & \sigma_3 & \sigma_2 & \sigma_1 & \sigma_0 
	\end{matrix}
	 \]
	 The cyclic subgroups will be
	 \[ \set{\sigma_0,\sigma_1},\quad \set{\sigma_0,\sigma_2},\quad \set{\sigma_0,\sigma_3,\sigma_3^2},\quad \set{\sigma_0,\sigma_5}. \]
	 The subgroup $ H = \set{\sigma_0, \sigma_1} $ is not a normal subgroup because $ \sigma_2 \sigma_1 \inv{\sigma_2} = \sigma_2 \sigma_1 \sigma_2 = \sigma_2 \sigma_4 = \sigma_5 \notin H $. So we expect $ G/H $ not to have a group structure. This is true because
	 \[ G/H = \set{\set{\sigma_0,\sigma_1}, \set{\sigma_2,\sigma_3},\set{\sigma_4,\sigma_5}}, \]
	 where
	 \[ \sigma_0 H  = \sigma_1 H = \set{\sigma_0,\sigma_1},\quad \sigma_2H = \sigma_3 H = \set{\sigma_2,\sigma_3},\quad \sigma_4H=\sigma_5H = \set{\sigma_4,\sigma_5}.\]
	 But
	 \[ (\sigma_0 H)\cdot (\sigma_2 H) = \sigma_2 H \quad (\sigma_1H)\cdot(\sigma_3 H) = \sigma_5 H, \]
	 but
	 $ \sigma_2 H \neq \sigma_3 H $.
	 
	 
	 \begin{example}[Quotient of $ S_3 $]
	 	From what we had above, we see that $ N = \set{\sigma_0, \sigma_3,\sigma_3^2} $ is the normal subgroup of $ G $. So it is interesting to ask what kind of group $ G/N $ will be? To answer this question, we need to calculate the cosets $ N $. A simple calculation shows
	 	\[ \sigma_0N = \sigma_3N = \sigma_4N = \set{\sigma_0,\sigma_3,\sigma_4}, \qquad  \sigma_1N =\sigma_2N = \sigma_5 N = \set{\sigma_1,\sigma_2,\sigma_5}.  \]
	 	Let $ a = \sigma_0 N $ and $ b = \sigma_1N $. Then it is easy to see $ aa = a,\ ab = b,\ ba = b,\ bb = a $. So we can say that 
	 	\[ G/N \cong S_2, \qquad \text{or alternatively} \qquad G/N \cong \Z_2. \]
	 \end{example}
	 
	 
	 \begin{example}[Demonstration of the Cayley's theorem]
	 	It is very easy to see the Cayley's theorem (Proposition 2.1.9) from the multiplication table above. Each line of this multiplication table can be thought of as a permutation of the elements of $ G $. For instance, the line corresponding to $ \sigma_2 $ can be thought of as the following permutation
	 	\[ \begin{pmatrix}
	 		0 & 1 & 2 & 3 & 4 & 5 \\
	 		2 & 3 & 0 & 1 & 5 & 4 
	 	\end{pmatrix} \]
	 	and the line corresponding to $ \sigma_3 $:
	 	\[ \begin{pmatrix}
	 		0 & 1 & 2 & 3 & 4 & 5 \\
	 		3 & 2 & 5 & 4 & 0 & 1
	 	\end{pmatrix}. \]
	 	Multiplying $ \sigma_2 \sigma_3 $ according to the multiplication table will result in $ \sigma_2 \sigma_3 = \sigma_1 $, and composing the permutations above we will get
	 	\[ 
	 	\begin{pmatrix}
	 		0 & 1 & 2 & 3 & 4 & 5 \\
	 		2 & 3 & 0 & 1 & 5 & 4 
	 	\end{pmatrix} 
	 	\begin{pmatrix}
	 		0 & 1 & 2 & 3 & 4 & 5 \\
	 		3 & 2 & 5 & 4 & 0 & 1
	 	\end{pmatrix}
	 	=
	 	\begin{pmatrix}
	 		0 & 1 & 2 & 3 & 4 & 5 \\
	 		1 & 0 & 4 & 5 & 2 & 3
	 	\end{pmatrix},
	 	 \]
	 	 which is precisely the permutation corresponding to $ \sigma_1 $.
	 \end{example}
	 
	 \begin{example}[Learning from my mistake!]
	 	As demonstrated in the example above, we have
	 	\[ \set{g\cdot(\square): g\in G} \overset{G}{\subseteq} S_G. \]
	 	For a while, I was thinking the permutations defined in this way are actually automorphisms and I was thinking the group on the left hand side is a subgroup of $ \Aut(G) $. But a simple calculation proved me that I am wrong. That is because for $ a,b \in G $, $ g\cdot(ab) \neq (ga)\cdot(gb) $. However, consider the following permutations
	 	\[ \set{g\cdot(\square)\cdot \inv{g}: g\in G}, \]
	 	then it is very easy to check that for $ a,b\in G $
	 	\[  g(ab)\inv{g} = ga\inv{g} g b \inv{g} = (ga\inv{g})(gb\inv{g}),  \]
	 	and it preserves the group structure of $ G $, and after checking that it is indeed a bijection we can conclude that it is a subgroup of $ \Aut(G) $. It is also easy to check that $ \set{g\cdot\square\cdot \inv{g}: g\in G} $ is also a normal subgroup of $ \Aut{G} $. That is because for $ \sigma \in \Aut(G) $, we have
	 	\[ \sigma\circ f_g \circ \inv{\sigma} = f_h, \]
	 	where $ f_g(\square) = g\square\inv{g}, $ and $ f_h(\square) = h\square\inv{h} $, for $ h = \sigma(g) $. That is because
	 	\[  \sigma\circ f_g \circ \inv{\sigma}(a) = \sigma(g(\inv{\sigma}a)\inv{g}) = \sigma(g) a \sigma(\inv{g}) = ha\inv{h} = f_h, \]
	 	where we have used the fact if $ \sigma $ is an homomorphism, then $ \sigma(\inv{g}) = \inv{\sigma(g)} $.
	 	So the correct inclusion will be
	 	\[ \set{g\cdot\square\cdot\inv{g}: g\in G} \lhd \Aut(G) \overset{G}{\subseteq} S_G, \]
	 	and
	 	\[ \set{g\cdot\square: g\in G} \overset{G}{\subseteq} S_G, \]
	 	(as demonstrated in the Cayley's theorem).
	 \end{example}
\end{observation}
\begin{example}
	$ S_3 $ as in the observation box above, can be generated by $ S = \set{\sigma_1, \sigma_2} $. 
\end{example}



\begin{theorem}
	Kernel of group homomorphisms is normal subgroup of the domain group. Conversely, any normal subgroup, is the kernel of some group homomorphism.
\end{theorem}
\begin{proof}
	Let $ \phi: G\to H $ be a group homomorphism. Let $ n \in \ker \phi $ and let $ g\in G $ be any group element. Then
	\[ \phi(gn\inv{g}) = \phi(g) \phi(n) \phi(\inv{g}) = \phi(g) \phi(\inv{g}) = \phi(g\inv{g}) = \phi(e) = e. \]
	So $ gn\inv{g} \in \ker \phi $. So $ \ker \phi \lhd G $.
	
	For the converse, let the group homomorphism be the canonical projection map to the quotient group $ G/N $. I.e. $ \phi: G\to G/N $ given by $ g \mapsto gN $ (here we only considered the left cosets).
\end{proof}
\begin{remark}
	Recall that from linear algebra, the kernel of a linear map is a subspace, and the quotient of a vector space with any subspace is a vector space. However, in the case of groups, the quotient of a group with only a normal subgroup is a group. By the theorem above, then for any group homomorphism $ \phi $, $ G/\ker \phi $ is a group.
\end{remark}


\begin{summary}[Outer and Inner Automorphism Group]
	Let $ G $ be a group, and consider its symmetric group $ S_G $. Although all of the elements of $ S_G $ are bijections, but they fail to preserve the group structure of $ G $. However, those elements of $ S_G $ that are also homomorphisms, indicated by $ \Aut(G) $, is the group of automorphisms of $ G $. Now consider the map
	\[ \Phi: G \to \Aut(G), \quad g \mapsto f_g, \]
	where $ f_g(h) = g\cdot h\cdot \inv{g} $. It is easy to check that $ f_g $ as defined above is indeed an automorphism. To see this, first we show that $ f_g $ is homomorphism. Let $ h_1,h_2\in G $, then 
	\[ f_g(h_1h_2) = g\cdot h_1h_2 \cdot\inv{g} = g\cdot h_1\cdot \inv{g} g \cdot h_2\cdot \inv{g} = f_g(h_1) f_g(h_2).  \]
	To show $ f_g $ is injective, let $ h \in \ker f_g $. Then $ f_g(h) = g\cdot h \cdot \inv{g} = e $. multiplying both side by $ g $ we will get $ gh = g $. (I am still thinking how to show $ hg = g $). So we can conclude that $ h =e $. For the surjectivity, let $ a\in G $. Then we want to find $ h\in G $ such that $ g\cdot h \cdot \inv{g} = a $. $ h $ is simply $ h = \inv{g}\cdot a g $. So $ f_g $ is indeed an automorphism.
	
	Furthermore, $ \Phi(G) $ is a normal subgroup of $ \Aut(G) $. That is because let $ f_g \in \Phi(G)$, and let $ \tau \in \Aut(G) $. Then
	\[( \tau\circ f_g \inv{\tau})(h) = \tau(g\cdot \tau^{-1}(h)\cdot \inv{g}) = \tau(g) \cdot \operatorname{id}_G h \cdot \inv{\tau(G)}.  \]
	So $ f_g = f_h $ for $ h = \tau g $. Because $ \Phi(G) $ is a normal subgroup, then we can talk about the quotient group $ G/\Phi(G) $. There are special names for these groups:
	\[ \Phi(G):\text{ inner automorphisms}, \qquad G/\Phi(G):\text{ outer automorphisms}. \]
\end{summary}

\subsection{Generators of Groups}
\begin{summary}[The generators of $ S_A $]
	Let $ A = \set{1,\cdots,n} $ and $ S_A $ be its symmetric group. Then we claim that $ S_A $ can be generated by 
	\[ \set{(1\mapsto2), (1\mapsto 3), \cdots, (1\mapsto j), \cdots, (1\mapsto n)}, \]
	where $ (i\mapsto j) $ is a transposition. Observe that we can generate any transposition by
	\[ (i\mapsto j) = (i\mapsto 1) \circ (1\mapsto j) \circ (i\mapsto 1). \]
	And now from these arbitrary transpositions we can generate any permutation.
\end{summary}

