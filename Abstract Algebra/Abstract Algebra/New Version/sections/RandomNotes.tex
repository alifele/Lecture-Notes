\chapter{Random Notes}

\section{Interesting Observations from Roman}

\begin{observation}[Geometric Interpretation of Dual Vectors]
	The notion of the dual space of a vector space is somewhat abstract and one usually struggles to have a geometric realization of the functionals and dual spaces. Here, I provide a very interesting point of view. Let $ V $  be a finite dimensional vectors space. Then every $ f \in V^* $ is characterized by a hyperplane $ H $ such that $ H = \operatorname{ker} f $. 
	
	\noindent With this point of view, $ f(x) = 0 $ corresponds to the fact that $ x \in  H $. Also, it is very straight forward to see the following properties of functionals with this geometric point of view.
	\begin{proposition}
		\begin{enumerate}[(a)]
			\item If $ f(x) \neq 0 $ then 
			\[ V = \langle x \rangle \oplus \ker f. \]
			\item For every $ x\in V $ there exists $ f \in V^* $ such that $ f(x) \neq 0 $.
			\item For $ x\in V $, $ f(x) = 0 $ for all $ f\in V^* $ implies $ x = 0 $.
			\item 
		\end{enumerate}
	\end{proposition}
	\begin{proof}(Geometric interpretation)
		\begin{enumerate}[(a)]
			\item If $ x\notin H $ for some hyperplane $ H $, then 
			\[ V = \langle x \rangle \oplus H. \]
			\item Given any point of the space, there is some hyperplane that misses that particular point.s
			\item The only point that belongs to all hyperplanes is the origin.
		\end{enumerate}
	\end{proof}
\end{observation}


