\chapter{Halmos Solution Manual}


\section{Fields}

\begin{observation}
	In a group we can only add one element an integer number of times with itself. For instance we can only have $ \alpha + \alpha + \cdot + \alpha = m\alpha  $ for some $ m \in \N $. However, field is a generalization of group in the sense that we can have more general many times addition with itself for every element. For instance we can have $ q \alpha  $ where $ q $ is not necessarily an integer. The same is true in vector spaces. The fact that we can multiply a vector by some element of the underlying field (i.e. the scalar) shows this. 
\end{observation}


\begin{problem} 
	\begin{solution}
		\begin{enumerate}[(a)]
			\item Holds because $ (F,0,+) $ is an abelean group.
			\item Since $ (\mathcal{F},0,+) $ is an abelian group, then $ \alpha $ has an inverse (i.e. $ -\alpha $). Add this to both sides of the equation.
			\item We can write
			\begin{align*}
				\alpha + (\beta - \alpha) &= \alpha + (\beta + (-\alpha)) \\
				&= \alpha + \beta + (-\alpha) && \text{(distributivity of multiplication)} \\
				&= \alpha + (-\alpha) + \beta = \beta && \text{($ (F,0,+) $ is abelian group)}
			\end{align*}
			\item We can write
			\begin{align*}
				\alpha\cdot0 = \alpha \cdot (0 + 0) = \alpha\cdot 0 + \alpha\cdot 0 = 2\alpha\cdot 0.
			\end{align*}
			Adding the inverse of $ \alpha\cdot 0 $ to both sides we will get
			\[ \alpha\cdot0 = 0. \]
			\item We can use the distributivity of multiplication and write
			\[ (-1)\alpha + \alpha = (-1 + 1)\alpha = 0\cdot\alpha = 0. \]
			Adding the inverse of $ \alpha $ to both sides we will get
			\[ (-1)\alpha = -\alpha. \]
			
			\item We can write
			\begin{align*}
				(-\alpha)(-\beta) &= (-1(\alpha))(-\beta)  && \text{(property proved above)} \\
				&= ((-1)(\alpha))((-1)(\beta)) && \text{(property proved above)} \\
				&= (-1)(-1)\alpha\beta && \text{(associativity of product)} \\
				&= -(-1 )\alpha \beta = \alpha\beta
			\end{align*}
			
			\item If both $ \alpha $ and $ \beta $ are zero, then it follows that $ \alpha\beta = 0 $. If one of them is not zero, WLOG we can assume $ \beta \neq 0 $, then $ \beta^{-1} $ exists, and multiplying it on the both sides we will get 
			\[ \alpha = 0. \]
			{\color{red} \noindent In a second run, I observed that my proof above might be wrong. I am somehow using the conclusion to prove the hypothesis.}
		\end{enumerate}
	\end{solution}
\end{problem}


\begin{problem}
	\begin{solution}
		\begin{enumerate}[(a)]
			\item No. $ (F,0,+) $ is not a group. $ (F\backslash\set{0},1,\cdot) $ is not a group.
			\item No. $ (F\backslash\set{0},1,\cdot) $ is not a group. The set of all integers is a Ring with identity.
			\item Yes. For instance, consider the set of all integers $ \Z $. Let $ \phi:\Z\to \Q $ be a bijection. Define the addition and multiplication as
			\[ m \oplus n = \inv{\phi}(\phi(m) + \phi(n)),\qquad m\odot n = \inv{\phi}(\phi(m)\cdot\phi(n)). \]
			Let $ z_1 = \inv{\phi}(1) $ and $ z_0= \inv{\phi}(0) $. Then we claim that $ (\Z,z_0,z_1,\oplus,\odot) $ is a field. It is straightforward to check that $ (\Z,z_0,\oplus) $, and $ (\Z\backslash\set{z_0},z_1,\odot) $ are groups, and the distributitivty law holds. For instance, to check for the associativity of addition let $ m,n,l \in \Z $. Then we have
			\begin{align*}
				(m\oplus n)\oplus l &= \inv{\phi}(\phi(m\oplus n) + \phi(l)) \\
				&= \inv{\phi}(\phi(m) + \phi(n) + \phi(l)) \\
				&= \inv{\phi}(\phi(m) + \phi(n\oplus l)) \\
				&= m \oplus (n\oplus l),
			\end{align*}
			where we have used the fact that $ \phi(m\oplus n) = \phi(m) + \phi(n) $.
		\end{enumerate}
	\end{solution}
\end{problem}


\begin{problem}
	\begin{solution}
		\begin{enumerate}[(a)]
			\item We can solve this part with different levels of abstraction. But we want to use the fact that the multiplicative group $ U(n) $ (i.e. the set of numbers in $ \Z_n $ that are prime relative to $ n $) is a group under multiplication (see example 3.3 Lee Abstract Algebra). When $ n $ is prime, then $ U(n) $ contains all numbers $ 1,\cdots,n-1 $. Thus when $ n $ is prime, $ (\Z_n\backslash\set{0}, \cdot) $ and $ (\Z_n,+) $ are both groups, and since the distribution law holds, it follows that $ \Z_n $ is a field if and only if $ n $ is a prime number. 
			\item $ -1 = 4 $ in $ \Z_5 $.
			\item $ \frac{1}{3} = 5 $ in $ \Z_7 $.
		\end{enumerate}
	\end{solution}
\end{problem}
\begin{observation}
	The field $ \Z_p $ has characteristic $ p $.
\end{observation}


\begin{problem}
	\begin{solution}
		First, observe that if the cardinality of the underlying set of the field is infinite, then we have $ \underbrace{1+\cdots + 1}_m = m\cdot 1 \neq 0 $ for all $ m\in \N $.
		However, when $ F $ is finite, then 
		\[ F\text{ is finite} \implies m\cdot 1 =0 \text{ for some } m\in\N. \]
		We can see this by contrapositive. If $ m\cdot 1 \neq 0 $ for all $ m\in \N $ then $ F $ has at least $ \N $ many elements. We want to show that the smallest such $ m $ is prime. Assume otherwise. Then $ m=pq $ for some $ p,q\neq 0 $. Then 
		\[ 0 = m\cdot 1 = pq\cdot 1 = pq. \]
		Since every field is an integral domain (has no zero divisors) (see Theorem 8.9 Lee Abstract Algebra), then it implies that $ p=0 $  or $ q=0 $, this is a contradiction.
	\end{solution}
\end{problem}

\begin{problem}
	\begin{solution}
		\begin{enumerate}[(a)]
			\item Yes. It is easy to check that $ (\Q(\sqrt{2}),0,+) $ and $ (\Q(\sqrt{2})\backslash\set{0},1,\cdot) $ are groups. The associativity and closedness of the operators can be shown directly. For instance
			\begin{align*}
				(\alpha + \beta\sqrt{2})(\eta + \gamma\sqrt{2}) = (\alpha\eta  + 2\beta\gamma )+ \sqrt{2}(\alpha\gamma + \beta\eta),
			\end{align*}
			hence the multiplication is closed. Also, $ 0,1 \in \Q $ are the same as $ 0,1\in \Q(\sqrt{2}) $. Also, it is easy to check that the additive inverse of $ \alpha + \sqrt{2}\beta $ is $ -\alpha - \sqrt{2}\beta $. And the multiplicative inverse is easy to calculate and follows from the observation that
			\[ (\alpha + \beta\sqrt{2}) \cdot (\frac{\alpha - \beta\sqrt{2}}{\alpha^2 - 2\beta^2}) =1. \]
			So the multiplicative inverse of $ \alpha + \beta\sqrt{2} $ is 
			\[ \frac{\alpha - \beta\sqrt{2}}{\alpha^2 - 2\beta^2}. \]
			\item No. Because $ 1+\sqrt{2} $ has no inverse of the form $ \alpha + \beta\sqrt{2} $ where $ \alpha,\beta \in \Z $. 
		\end{enumerate}
	\end{solution}
\end{problem}

\begin{problem}
	\begin{solution}
		\begin{enumerate}[(a)]
			\item No. Not every polynomial has an inverse with integer coefficients. For instance, $ p= 2x^2 -1 $ should be multiplied by
			\[ -1 + 2x^2 - 4x^4 + 8x^6 - \cdots \]
			to get the $ 1 $ polynomial. But the expression above is not a polynomial.
			\item No. The same problem above. The set of all polynomials with integer or real coefficients forms a commutative Ring. 
		\end{enumerate}
	\end{solution}
\end{problem}

\begin{problem}
	\begin{solution}
		\begin{enumerate}[(a)]
			\item The addition part of OK! I.e. $ (F,(0,0),+) $ forms an abelian group. However, $ (F\backslash\set{(0,0)},\mathds{1},\cdot) $ does not form a group as defined above. Because by the provided definition of multiplication we need to have $ (\alpha,\beta)\mathds{1} = (\alpha,\beta) $ that implies that the only choice for $ \mathds{1} $ is
			\[ \mathds{1} = (1,1). \]
			But then the elements $ (0,1) $ and $ (1,0) $ have no multiplicative inverses. This is not the only obstacle though.
			
			\item Yes. This multiplication resolves the obstacles above and $ (F\backslash\set{(0,0)},\mathds{1},\cdot) $ is an abelian group. It is easy to check that the multiplicative identity should be
			\[ \mathds{1} = (1,0). \]
			I.e. this is the only choice that satisfies $ (\alpha,\beta)\cdot \mathds{1} = (\alpha,\beta) $. It is also easy to check that the inverse for a non-zero element $ (\alpha,\beta) $ is
			\[ (\frac{\alpha}{\alpha^2 + \beta^2}, \frac{-\beta}{\alpha^2 + \beta^2}). \]
			
			\item It will lead to the same kind of structure. 
		\end{enumerate}
	\end{solution}
\end{problem}


\section{Vector Spaces}
\begin{problem}
	\begin{solution}
		\begin{enumerate}[(a)]
			\item This follows from $ (V,0,+) $ being an abelian group.
			\item The additive inverse of the zero element in an additive group is itself. So this follows from $ (V,0,+) $ being an abelian group.
			\item We can write
			\[ \alpha\cdot0 = \alpha\cdot(0 + 0) = \alpha\cdot 0 + \alpha\cdot 0 \]
			Since the set of vectors is an additive abelian group, we can add the inverse of $ \alpha\cdot 0 $ to both sides and get
			\[ \alpha\cdot 0 = 0. \]
			\item We can write
			\[ 0\cdot x = (0+0)\cdot x = 0\cdot x + 0\cdot x. \]
			Since the set of vectors is an additive abelian group, then adding the inverse of $ 0\cdot x $ to both sides we will get
			\[ 0\cdot x = 0. \]
			\begin{remark}
				Note that in the expression above, the zero on the LHS is the zero element of the field, and the zero on the RHS is the zero element of the vector field. 
			\end{remark}
			\item {\color{red} \noindent Still thinking. I was trying to do a similar proof as for problem 1 part (g), but I realized that my proof for that part is also not correct.}
			
			\item We can add $ x = 1\cdot x $ to $ (-1)x $. Then using the distributivity law we can write
			\[ x + (-1)x = 0. \]
			Adding the additive inverse of $ x $ to both sides we will get
			\[ (-1)x = -x. \]
			
			\item We can write
			\[ y + (x-y) = 1\cdot y + 1\cdot (x-y) = 1\cdot(y+x+(-y)) = 1\cdot x = x. \]
 		\end{enumerate}
	\end{solution}
\end{problem}

\begin{problem}
	The elements of $ \Z_p^n $ are the n-tuples, or equivalently the set of all functions $ f:[p]\to \Z_p $ where $ [n]= \set{1,2,\cdots,n} $. There are $ p^n $ such functions. 
\end{problem}


\begin{problem}
	\begin{solution}
		No. One of the immediate problems that I can see is that the scalar $ 1 $ does not interact nicely with the vectors. I.e. in the vector space axioms we have $ 1\cdot x = x $ for all $ x\in V $. However, in the definition above we have $ 1\cdot(\xi_1,\xi_2) = (1\xi_1,0) = (\xi_1,0) \neq (\xi_1,\xi_2).    $
	\end{solution}
\end{problem}

\begin{problem}
	\begin{solution}
		We assume that the vector space $ \C^3 $ is defined on $ \C $ rather than $ \R $.
		\begin{enumerate}[(a)]
			\item No. While the vector space $ (V,0,+) $ forms a group, but the scalars does not behave nicely. For instance $ i\cdot(r_1,\xi_2,\xi_3) = (ir_1,\xi_2,\xi_3)  $ and the first argument is not longer a real number.
			\item Yes.
			\item No. because $ (\xi_1,0,\xi_2) + (0,\tilde\xi_2,\tilde\xi_3) = (\xi_1,\tilde\xi_2,\xi_2+\tilde\xi_3) $ and neither its first or second argument is zero.
			\item The vector space $ (V,0,+) $ forms a group. The addition is closed: Let $ (\xi_1,\xi_2,\xi_3) $ and $ (\tilde\xi_1,\tilde\xi_2,\tilde\xi_3) $ be in the subspace. Then 
			\begin{align*}
				\alpha (\xi_1,\xi_2,\xi_3) + \beta(\tilde\xi_1,\tilde\xi_2,\tilde\xi_3) = (\alpha\xi_1+\beta\tilde\xi_1,\alpha\xi_2+\beta\tilde\xi_2,\xi_3+\tilde\xi_3).
			\end{align*}
			Since
			\[ \alpha(\xi_1 + \xi_2) + \beta (\tilde\xi_1 + \tilde\xi_2) = 0, \]
			then the sum is also in the subspace, and the addition is closed. Also, in additive inverse of $ (\xi_1,\xi_2,\xi_3) $ is
			\[ (-\xi_1,-\xi_2,-\xi_3) \]
			where since $ -(\xi_1+\xi_2) = 0 $ it implies that the inverse is also in the subspace. It is also easy to check that the scalars behave nicely with the vector space.
			\item No. This subspace does not contain the origin $ (0,0,0) $.
		\end{enumerate}
	\end{solution}
\end{problem}

\begin{problem}
	\begin{solution}
		The answers to this question depends on the set where the coefficients of the polynomial belongs, as well as the scalars on which the vector space is defined. We assume that the coefficients are complex numbers and the scalars are also complex numbers. 
		\begin{enumerate}[(a)]
			\item Yes.
			\item Yes. It is easier to see this if we identify each such polynomial with a 4-tuple $ (a_0,a_1,a_2,a_3) $, where these coordinates record the coefficients of the monomials $ 1,x^1,x^2$, and $ x^3 $ respectively. Then the subset of interest is 
			\[ \tilde V = \set{(a_0,a_1,a_2,a_3): a_1 = 2a_0}. \]
			This is subspace. Because the structure $ (\tilde V,+,0) $ is a group (it is easy to check that under addition of two such tuples the resulting tuple still satisfies $ \xi_1 = 2\xi_0 $). Also, it is straightforward to see that $ (0,0,0,0) \in \tilde V $. And the inverse of $ (a_0,2a_0,a_2,a_3) $ is $ (-a_0,-2a_0,-a_2,-a_3) $.
			\item No. let $ x\in V $. So $ x(t) \geq 0 $ for $ t\in [0,1] $. Let $ \alpha = 2 $ be a scalar. Then $ \alpha x(t) \leq 0 $ for $ t\in [0,1] $. So $ \alpha x(t) \notin V $.
			
			\item Yes. The specified condition leads to 
			\[ a_0 + a_1t + a_2t^2 + a_3t^3 = a_0 + a_1(1-t) + a_2(1-t)^2 + a_3(1-t)^3.  \]
			This simplifies to 
			\[ (2t-1)a_1 + (t^2-(1-t)^2)a_2 + (t^3 - (1-t)^3)a_3 = 0. \]
			Observe that
			\[ t^2 - (1-t)^2 = 2t-1. \]
			Also
			\[ t^3 - (1-t)^3 = 2t^3 - 3t^2 + 3t - 1. \]
			So we will have
			\[ (2t-1)a_1 + (2t-1)a_2 + (2t^3-3t^2+3t-1)a_3 = 0. \]
			This is a hyperplane passing through the origin in $ \R^4 $ (if we identify each polynomial with a 4-tuple). This defines a subspace. Note: We could have said this without simplifying the coefficients, and I did that for no good reason!
		\end{enumerate}
	\end{solution}
\end{problem}

\section{Bases}
\begin{observation}
	Consider the vector space $ \C^1 $. Let $ z_1,z_2 \in \C^1 $ be any non-zero vectors. Then $ z_1,z_2 $ are linearly dependent vectors. The reason that I am highlighting this is that I was putting too much emphasis on the looking at $ \C^1 $ as $ \R^2 $ that I was not expecting to see that any two non-zero vectors in $ \C^1 $ is linearly dependent (which is definitely false, in $ \R^2 $ as not any two non-zero vectors are linearly dependent). Then I realized that it is the magic of scalars that makes the difference. If the vector space of consideration is $ (\C^1,\R^1) $, i.e. the scalars are real numbers, then not every two vectors in $ \C^1 $ is linearly dependent. For instance the vectors $ 1 $ and $ i $ are linearly independent (as there are no ways to multiply $ 1 $ at a real number and get $ i $). But in the case of $ (\C^1, \C) $ any two vectors are linearly dependent. Because in the example above we can multiply $ i $ be $ -i $ and get $ 1 $. I.e. in the second case the scalars are not only for enlarging the vectors, but also to rotate them.
\end{observation}



