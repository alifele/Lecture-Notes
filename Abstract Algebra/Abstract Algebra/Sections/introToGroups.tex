Here in this section we will study one of the simplest algebraic structures, groups. We grab a set, define a binary operator on it, impose four basic rules and see what we can deduce. Since the algebra is an abstract way to study the interaction of different things with each other, (two things in a sets interact and produce another element of the set), we expect to see the group theory stuff in our real life a lot. As an instance the solution the Rubik's cube is a problem in group theory.

In this chapter, we will study this algebraic structure with a number of examples along with deducing some important properties. 

\section{An Important Example}
Here in this section we will go through an example that serves as a very great motivation to study the structures so-called groups. Let $S = \{1,2,3\}$. Then define the one-to-one and onto permutation function $\sigma: S \rightarrow S$. An example of such function is $\sigma(1) = 2, \sigma(2)=1,$ and $\sigma(3)=3$. Since every function is a kind of relation, so we can represent this function as a subset of the Cartesian product $S \times S$. Then the function $\sigma$ will be $\sigma = \{ (1,2),(2,1),(3,3) \}$. However, we can define the following notation to represent this function in a more clear way:
\[ \sigma = \begin{pmatrix}
	1 & 2 & 3 \\
	2 & 1 & 3
\end{pmatrix} \]
This is just one out of 6 (why?) such function that we can define. The set of all permutation functions on the set $S$ will be:
\[ F = \{ 
	\begin{pmatrix}
	1 & 2 & 3 \\
	1 & 2 & 3
	\end{pmatrix},
	\begin{pmatrix}
		1 & 2 & 3 \\
		2 & 1 & 3
	\end{pmatrix},
	\begin{pmatrix}
		1 & 2 & 3 \\
		1 & 3 & 2
	\end{pmatrix},
	\begin{pmatrix}
		1 & 2 & 3 \\
		3 & 2 & 1
	\end{pmatrix},
	\begin{pmatrix}
		1 & 2 & 3 \\
		2 & 3 & 1
	\end{pmatrix},
	\begin{pmatrix}
		1 & 2 & 3 \\
		3 & 2 & 1
	\end{pmatrix}
	 \}. 
\]

Now let's pick two permutations from $F$ and study them closely:
\[ \sigma = \begin{pmatrix} 1 & 2 & 3 \\ 1 & 3 & 2 \end{pmatrix}, \quad
	\tau = \begin{pmatrix} 1 & 2 & 3 \\ 3 & 2 & 1 \end{pmatrix}. \]
With these two permutations in hand, we can define the composition of two permutations as: \[(\sigma \circ \tau)(n) = \sigma (\tau (n)) \]. Then we will have:
\[\sigma \circ \tau = \begin{pmatrix} 1 & 2 & 3 \\ 2 & 3 & 1 \end{pmatrix} \]
We observe that the composition of two permutations is itself a permutation. So the composition operator as defined above acts like a binary operator on the set $F$ (which were the set of all permutations on $S$). So the set $F$ under the composition has the a \textbf{closure} property. Let's pick another permutation $\omega$ from $F$:
\[ \omega = \begin{pmatrix} 1 & 2 & 3 \\ 2 & 3 & 1 \end{pmatrix} \]
Then we observe that there is a kind of \textbf{associativity law}:
\[ \omega \circ (\sigma \circ \tau) = (\omega \circ \sigma) \circ \tau = 
\omega \circ \begin{pmatrix} 1 & 2 & 3 \\ 2 & 3 & 1 \end{pmatrix} = 
\begin{pmatrix} 1 & 2 & 3 \\ 2 & 1 & 3 \end{pmatrix} \circ \tau
= \begin{pmatrix} 1 & 2 & 3 \\ 3 & 1 & 2 \end{pmatrix}.\]
Also, we can observe that the permutation \[  \lambda = \begin{pmatrix} 1 & 2 & 3 \\ 1 & 2 & 3 \end{pmatrix} \] is kind of special and acts like an \textbf{identity element}. Since its permutation with every other permutation produces the same permutation, i.e. \[ \sigma \circ \lambda = \lambda \circ \sigma = \sigma. \] Also, we can see that for every permutation $\alpha \in F$, we can find another permutation $\beta \in F$ that acts like an \textbf{inverse}, because \[\alpha \circ \beta = \beta \circ \alpha = \lambda. \] Note that the set of all permutation functions on the set $S$, is in fact the set of all bijective (injective and surjective) functions from $S$ to $S$. So we can show that for a bijective function, and inverse always exists. 

The set of bijective functions on the set $S = \{ 1,2,\cdots,n\}$ is a great representation of the algebraic structure that we can it a \textbf{group}.

\begin{defbox}
	The set of all permutations of the set $S = \{1,2,\cdots,n\}$ under composition operator is called a  \textbf{symmetric group} on $n$ letters and is denoted by $S_n$.
\end{defbox}

The observations that we had above, are not by accident and we can put those things into a more formal structure and define group.

\begin{example}{Number of permutations on $S_n$}
	\textit{Question.} How many permutations we can define the set $S_n$? In $S_5$, how many permutations $\alpha$ satisfy $\alpha(w) = 2$? \\
	
	\textit{Answer.} Let $S_n = \{ 1,2,\cdots,n \}$. Then the element $1 \in S_n$, can be mapped onto $n$ possible numbers. The element $2 \in S_n$, can be mapped to one of $n-1$ remaining elements, and so on. So the total number of permutations on $S_n$ will be $n!$.
	
	For the second part of the question, the answer will be very similar to the reasoning above. Look at the diagram below:
	\[ \sigma = \begin{pmatrix} 1 & 2 & 3 & 4 & 5 \\ \Box & 2 & \Box & \Box & \Box \end{pmatrix}. \]
	Then for the very first box, there are 4 choices, for the second box there remains 3 choices, and so on. So the total number of permutations with $\alpha(2) =2$ will be $4! = 24$.
\end{example}

\begin{example}{Permutations on $S_5$}
	\textit{Question.} Let $H$ be the set of all permutations $\alpha \in S_5$ satisfying $\alpha(2)=2$. Which of the properties closure, associativity, identity, and inverse does $H$ enjoy under composition of functions? \\
	
	\textit{Answer.} 
	\begin{itemize}
		\item Closure: Let $\alpha, \beta \in H$. Then $(\alpha \beta)(2) = \alpha(\beta(2)) = \alpha(2) = 2$, and $(\beta \alpha) (2) = \beta(\alpha(2)) = \beta (2) = 2$. So $\forall \alpha, \beta \in H,$ both $\alpha\beta$ and $\beta\alpha$ are in $H$.
		\item Associativity: Since associativity law holds for $S_n$ under composition, then it holds for $H$ under composition as well.
		\item Inverse: Since all elements of $H$ are bijective, then there exist inverses. But the thing that we need to check is that if the inverse still lives in the $H$ as well? The answer is yes. Since, from the properties of the inverse function we have $\alpha^{-1} (\alpha(2))=2$. On the other hand $\alpha(2)=2$. So $\alpha^{-1}(2) = 2$. So $\alpha^[-1] \in H$.
		\item Identity: Since the identity of $S_5$ is \[ i = \begin{pmatrix} 1 & 2 & 3 & 4 & 5 \\ 1 & 2 & 3 & 4 & 5 \end{pmatrix}, \] and $i(2)=2$, then $i \in H$.
		So $H$ holds all of the required properties.
	\end{itemize}
\end{example}

\begin{example}{The set of all functions on $S_n$}
	\textit{Question.} Consider the set of all functions from $A = \{1,2,3,4,5\}$ to $A = \{1,2,3,4,5\}$. Which of the properties closure, associativity, identity, and inverse does this set enjoy under composition of functions?\\
	
	\textit{Answer.}
	\begin{itemize}
		\item Closure: The composition of two functions from $A$ to $A$ is always a function from $A$ to $A$. So the closure holds.
		\item Associativity: Associativity holds for the composition of functions in general. So the associativity also holds for this case.
		\item Identity: The element $i$ acts as the identity element for the set of all function from $A$ to $A$:  \[ i = \begin{pmatrix} 1 & 2 & 3 & 4 & 5 \\ 1 & 2 & 3 & 4 & 5 \end{pmatrix} \]
	\end{itemize}
\end{example}
