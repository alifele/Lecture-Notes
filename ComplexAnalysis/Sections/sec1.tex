%\section[Fundamental Concepts]{\hyperlink{toc}{Fundamental Concepts}}
\section{Fundamentals}

Complex numbers can be thought as an extension to the real number system in which we have a solution for the $x^2 +1 = 0 $ equation. There are some mathematically rigorous ways to construct the complex numbers from real numbers. However, those mathematically rigorous ways came out only recently (20th century) and there were not around when the first ideas of complex numbers were forming around 18th century.  For that reason I have not discussed the detailed mathematical construction of the complex numbers here. \\

As discussed earlier, complex numbers system is a system in which we have a solution for the $x^2+1=0$ equation which is represented as $i = \sqrt{-1}$. A complex number is written like $z = a+bi$ in which $a,b \in \mathbb{R}$ and the set of all complex numbers is denoted as $\mathbb{C}$. It is easy to check that complex numbers still satisfy the \emph{commutative}, \emph{associative}, and \emph{distributive} properties similar to the real numbers. 






\begin{itemize}
\item The intuition behind the complex variables (from visual complex analysis book)


\end{itemize}




\section{Complex Maps}

\subsection{Linear Map}

\subsection{Inverse Map}

\subsection{Mobius Map}

\subsection{Quadratic Map}

\subsection{Exponential Map}


\section{Calculus for Complex variables}

\subsection{Limit}

\subsection{Continuity}

\subsection{Differentiability}