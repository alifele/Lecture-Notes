\section{Preparation for the Exam}


\subsection{Complex Integration}

There are many ways to evaluate a complex integral, and in this section we will review some of them.

\begin{fact}[Integration by Substitution]
	We can do a contour integration by substitution method similar to what we had in the real analysis. 
	\[ \int_{\gamma} f(z) dz = \int_{t_0}^{t_1} f(\gamma(t)) \gamma'(t) dt, \]
	where $\gamma:\R \to \C$ is a path in the complex plane parameterized by $t \in [t_0,t_1]$.
\end{fact}

This method is the most general method of evaluating the integrals. However, it comes with the generality, the cumbersomeness of this method to evaluate some integrals. It is very easy to get some very messy integrals with this method that is almost impossible to solve analytically. That is why in the future steps we will develop some more theory that will enable us to evaluate the integrals more easily.


\begin{fact}[Fundamental Theorem of Calculus]
	To evaluate the integral 
	\[ \int_\gamma f(z) dz, \]
	if the path $\gamma$ lies in an open connected region $\Omega$ in which $f$ is continuous and has an anti-derivative $F(z)$ (i.e. $F(z)$ is holomorphic at $\Omega$ such that $F'(z) = f(z)$) then we have
	\[ \int_\gamma f(z) dz = F(z_1) - F(z_0), \] 
	where $z_1$ and $z_0$ are the starting and the finishing points of the path.
	
\end{fact}

Then it follows immediately, that if $f$ is continuous in a open, connected region $\Omega$ and posses an anti-derivative, then the contour integral around every closed loop is zero. I.e., for any closed contour $\gamma$ we have
\[ \int_\gamma f(z) dz = 0. \]


\begin{fact}[Cauchy's integral theorem]
	Cauchy's integral theorem (do not confuse with the Cauchy's integral formula!) can be stated in two different ways, from which I prefer the first one. \\
	\textbf{The first form of Cauchy's integral theorem}: Let $f$ be analytic at some open connected region, and $\gamma_1$ and $\gamma_2$ two \emph{loops} that can be continuously deformed to each other. Then we have
	\[ \int_{\gamma_1} f(z) dz = \int_{\gamma_2} f(z) dz. \]
	\textbf{The second form of Cauchy's integral theorem}: Let $f$ be analytic at some open, \emph{simply} connected region, and $\gamma$ any loop in that region. Then 
	\[ \int_\gamma f(z) dz = 0. \]
	
\end{fact}

\begin{exm}
	Evaluate the integral 
	\[ \int_\gamma (z-z_0)^n dz, \] 
	where $\gamma$ is any simple,closed loop winding around $z_0$ in counterclockwise direction.
\end{exm}
\begin{answer}
	When $n\geq0$, then $(z-z_0)^n$ is entire, thus by the Cauchy's integral theorem the integral is zero. However, for $n<0$, $f(z)$ is only analytic in the punctured plane, punctured at $z_0$. Thus Cauchy's integral formula can not make any statements about the value of the integral. However, for $n\leq-2$, the integrand is continuous and has anti-derivative. Thus its integral on any closed contour evaluates to zero. Finally, for $n=-1$, we can follow two methods: 1. To evaluate the integral using the direct method (which is very hard as we don't know the contour explicitly), or using the Fundamental theorem of calculus. We can break the path into two pieces, and on each piece we can find a branch of $Log$ function that is analytic. Thus we can calculate the integral, which will be $2\pi i$.
\end{answer}

\begin{fact}[Cauchy's Integral Formula]
	Let $\gamma$ be a simple, closed, positively oriented loop in an open, simply connected region $\Omega$ and $z_0 \in \Omega$. If $f$ is analytic in this region, then
	\[ f(z_0) = \frac{1}{2\pi i} \int_\gamma \frac{f(z)}{z-z_0} dz. \]
	Also, we have
	\[ f'(z_0) = \frac{1}{2\pi i} \int_\gamma \frac{f(z)}{(z-z_0)^2}dz. \]
	And in general
	\[ f^{n}(z_0) = \frac{n!}{2\pi i} \int_{\gamma} \frac{f(z)}{(z-z_0)^{n+1}} dz. \]
\end{fact}

\begin{fact}[Evaluating integrals with Lemma 1]
	Let $P(z)$ be a polynomial of degree at least 2. The 
	\[ \int_{\gamma}\frac{1}{P(z)} dz = 0, \]
	for $\gamma$ a positively oriented closed path.
\end{fact}
\begin{exm}
	Instead of proof, we want to show case the reasoning in a specific example. We want to evaluate the integral
	\[ I = \int_{\gamma} \frac{1}{z^2(z-1)^3} dz, \]
	where $\gamma$ is define as $\abs{z} = R > 2$. 
\end{exm}
\begin{answer}
	First, note that all of the zeros of the polynomial fall inside the closed loop $\gamma$, thus we can easily inflate the loop to go to infinity. To take use of this, we first approximate the integral 
	\[ \abs{\int_{\gamma} \frac{1}{z^2(z-1)^3} dz} \leq \max_{z\in \gamma*} \abs{\frac{1}{z^2(z-1)^3}} L(\gamma) = \frac{2\pi R}{ \min_{z\in \gamma*} \abs{z^2(z-1)^3}}= \frac{2\pi}{R(R-1)^3}.\]
	Now letting $R\to\infty$, by the squeeze theorem we have
	\[ \abs{I} = 0 \implies \int_{\gamma} \frac{1}{z^2(z-1)^3} dz \]
\end{answer}





























