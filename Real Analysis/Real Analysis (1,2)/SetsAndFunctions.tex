\chapter{Sets and Mappings}

\begin{remark}
	Let $\R_+$ denote the real number greater than or equal to zero\footnote{This note is from Segel, undergraduate analysis.}. The we can view the association $x \mapsto x^2$ as a map from $R$ to $\R_+$. When viewed so, the map is the surjective. Thus it is a reasonable convention not to identify this map with the map $f:\R \to \R$ defined by the same formula. To be completely accurate, we should therefore denote the set of arrival and the set of departure of the map into our notation, and for instance write
	\[ F^S_T: S \to T, \]
	instead of our $f: S \to T$ notation. In practice, this notation is too clumsy, so that we omit the indices $S, T.$ However, the reader should keep in mind the distinction between the maps 
	\[ f^\R_{\R_+}: \R \to \R_+ \quad \text{and} \quad f_\R^\R: \R \to \R  \]
	both defined by the association $x \mapsto x^2$. The first map is surjective, while the second one is not. Similarly the maps
	\[ f_{\R_+}^{R_+}: \R_+ \to \R_+ \quad \text{and} \quad f_\R^{\R_+}: \R_+ \to \R \]
	defined by the same association $x \mapsto x^2$ are injective.
\end{remark}

\begin{remark}

\end{remark}



\section{Problems}
\begin{problem}
	Let $X,Y$ be sets and $B \subseteq Y$. Prove
	\[ \inv{f}(B^c) = \inv{f}(B)^c. \]
\end{problem}

\begin{proof}
	First we show $\inv{f}(B^c) \subseteq \inv{f}(B)^c$. Let $x \in \inv{f}(B^c)$. Then $f(x) \in B^c \implies f(x) \notin B \implies x \notin \inv{f}(B) \implies x \in \inv{f}(B)^c$. For the other direction, assume $x \in \inv{f}(B)^c$. Then $x \notin \inv{f}(B) \implies f(x) \notin B \implies f(x) \in B^c \implies x \in \inv{f}(B^c)$. 
\end{proof}

\begin{problem}
	Suppose both $f_1, f_2: X\to Y$ are continuous, where $X$ and $Y$ are Hausdorff topological spaces. Then prove that for any $Q \subseteq X$ we have
	\[ [f_1(q) = f_2(q) \quad \forall q\in Q ] \implies [f_1(x) = f_2(x) \quad \forall x\in\closure{Q}].\]
\end{problem}
\begin{proof}
	We will prove by contrapositive. The contrapositive will be
	\[ [\exists x \in \closure{Q} \st f_1(x) \neq f_2(x)] \implies [\exists q \in Q \st f_1(q) \neq f_2(q)].  \]
	Let $x\in \closure{Q}$ such that $f_1(x) \neq f_2(x)$. $x \in \closure{Q}$ implies $x \in Q$ or $x\in Q'$. In the former case, there is nothing to prove the consequent of the statement. Thus let $x \in Q'$. Also let $U,V \in \mathcal{T}_Y$ such that $f_1(x)\in U$ and $f_2(x) \in V$ and $U\cap V = \emptyset$. Since $f_1$ and $f_2$ are continuous, then $\inv{f_1}(U)$ and $\inv{f_2}(V)$ are both open and contains $x$. Since $\inv{f_1}(U) \cap \inv{f_2}(V)$ is open, then $\exists G \in \mathcal{T}$ such that $x \in G \subseteq \inv{f_1}(U) \cap \inv{f_2}(V)$. Furthermore, since $x \in G'$, then $\exists q \in Q \st q \in G$. Thus $f_1(q) \in U$ and $f_2(q) \in V$ and since $U \cap V = \emptyset$ then $f_1(q) \neq f_2(q)$ and this completes the proof.
\end{proof}
