\chapter{Useful Tips, Tools, and Tricks for $\R$eal Analysis}


In this chapter we will cover some useful tips and tricks in solving the problem. This is not a complete and comprehensive list of tricks, and the list will grow larger as I encounter more of these in different resources.

\begin{trick}
	There is a way to produce multiplication from squaring and addition/difference. Suppose that a set $S$ is closed under squaring, as well as addition and subtraction. Thus if $a,b \in S$ then $a+b \in S$, $a-b\in S$, thus $(a-b)^2 \in s$ as well as $(a+b)^2\in S$. Thus $(a+b)^2 - (a-b)^2 \in S$ as well. Thus
	\[  (a+b)^2 - (a-b)^2 = 4ab \in S. \]
	Thus we can conclude that being closed under addition, subtraction, and squaring, leads to being closed under multiplication as well.
\end{trick}
One useful use of this trick is in when we try to prove if $f$ and $g$ are Riemann–Stieltjes integrable, then we want prove that $fg$ is also Riemann–Stieltjes integrable. We do so by observing that $f,g$ are RS integrable, then so is $f+g$ and $f-g$ (which are proved by simpler theorem) and also $f^2$ is RS integrable. Thus from the trick above we can conclude that $fg$ is also RS integrable. 













\section{Construction Zone}
Some tricks for real analysis. I will expand this list and also explain each one with its particular use case throughout this chapter


\begin{enumerate}

\item archimedes
\item well ordering principle
\item unpack defn
\item IVT
\item MVT
\item three equiv. defn for cts.
\item sequential compactness
\item heine borel
\item bolzano weierstrass
\item lim $x_n=L$ iff limsup $x_n=\liminf x_n=L$
\item sequential characterization of open, closed sets, and others
\item union of countable set is countable
\item add subtrace and multiply divide
\item define new set or function and play with it
\item construct sequence
\item triangle inequality
\item X compact then f:X->R has max/min in X and f(X) compact
\item work backwards and apply previous results
\item contrapositives
\item denseness of Q in R
\item finite intersection prop
\item cantor intersection thm
\item squeeze thm
\item limit laws
\item monotone convergence thm
\item show seq cauchy (by completeness)
\item diff between squares
\item exploit the defn of convergence
\item consider all cases
\item maybe try proof by contradiction
\item ep ball defn of x in cl(A)
\item if A is subset of B and B closed then cl(A) is subset of B
\item if A is subset of B and A open then A subset of int(A)
\item HTS1-HTS4

\end{enumerate}

