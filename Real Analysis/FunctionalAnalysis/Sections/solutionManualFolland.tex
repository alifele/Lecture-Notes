\chapter{Solution Manual Folland}

Here is a list of theorems that are used in the problem sets.
\begin{proposition}[Some properties]
	\begin{enumerate}
		\item Let $ T \in L(X,X) $. Then $ \norm{T^n} \leq \norm{T}^n $.
	\end{enumerate}
\end{proposition}
\begin{proof}
	\begin{enumerate}[(a)]
		\item We demonstrate the statement for the case where $ n=2 $, and the general result follows by induction. Observe that
		\[ \norm{T^2x} = \norm{T(Tx)} \leq \norm{T} \norm{Tx} \leq \norm{T}^2\norm{x}. \]
		Since $ \norm{T^2} $ is smallest constant $ C $ such that $ \norm{T^2x}\leq C\norm{x} $ for all $ x\in X $, the it follows that $ \norm{T^2} \leq \norm{T}^2 $.
	\end{enumerate}
\end{proof}



\section{Elements of Functional Analysis}


\begin{problem}[Folland: Ch5, P1]
	The essence of the proof is to show that if $ u $ is close to $ u' $ and $ v $ is close to $ v' $ the $ u+v $ is close to $ u'+v' $. This is true because
	\[ \norm{(u+v) - (u'+v')} \leq \norm{u-u'} + \norm{v-v'} < 2\epsilon, \]
	where we choose $ \norm{u-u'} <\epsilon $ and $ \norm{v-v'}<\epsilon $. 
	
	To prove the continuity of the multiplication, we also use a similar idea. Want to show that if $ u $ is close to $ u' $ then $ \alpha u $ is also close to $ \alpha u' $. To see this we can write
	\[ \norm{\alpha u - \alpha u'} = \alpha \norm{u - u'} < \alpha\epsilon \]
	where we choose $ \norm{u-u'} < \epsilon $.
	
	To prove the continuity of the norm we need to show that if $ u $ is close to $ v $ then $ \norm{u} $ is close to $ \norm{v} $ (as real numbers). So we need to prove
	\[ \abs{\norm{u} - \norm{v}} \leq \norm{u - v}. \]
	This is reverse triangle inequality. To prove this we can write
	\[ \norm{y} = \norm{y \pm x} \leq \norm{y-x} +\norm{x}. \]
	This implies
	\[ \norm{y} - \norm{x} \leq \norm{y-x}. \]
	By a similar argument we can also write
	\[ \norm{x} - \norm{y} \leq \norm{y-x}. \]
	These two inequalities implies that 
	\[ \abs{\norm{x}-\norm{y}} \leq \norm{x- y}. \]

\end{problem}

\begin{problem}[Folland: Ch5, P4]
	Briefly, we want to prove that given any $ \epsilon>0 $ we can make $ T $ is close enough to $ S $ and $ x $ close enough to $ y $ such that $ Tx $ is closer to $ Sy $ that $ \epsilon $.
	Fix $ (T,y) \in L(X,Y)\times X $. We want to show continuity at this point. For the given $ \epsilon>0 $ as above if we have $ \norm{T-S} < \epsilon/\norm{y} $ and $ \norm{x-y} <\epsilon/\norm{T} $, then 
	\[ \norm{Tx - Sy} \leq \norm{Tx-Ty} + \norm{Sy-Ty} \leq \norm{T}\norm{x-y} + \norm{S-T}\norm{y} < 2\epsilon.  \] 
\end{problem}


\begin{problem}[Folland: Ch5, P7]
	\begin{solution}
		\begin{enumerate}[(a)]
			\item First, we want to show that the series $ \sum_{n=0}^\infty (I-T)^n  $ converges. First, observe that this series converges absolutely. Because
			\[ \sum_{n=0}^\infty \norm{(I-T)^n} \leq \sum_{n=0}^\infty \norm{I-T}^n  \leq \sum_{n=0}^{\infty} \gamma^n = \frac{1}{1-\gamma} < \infty. \]
			Using the fact that $ X $ is a Banach space (hence complete), it follows that $ L(X,X) $ is also complete, thus by Theorem 5.1 Folland the absolutely convergent series converges in $ L(X,X) $. Let 
			\[ L(X,X) \ni X = \sum_n (I-T)^n. \]
			Now we want to prove that $ X $ is left and right inverse of $ T $. To see this we can write
			\[ (I-T)X = \sum_{n=0}^\infty (I-T)^{n+1} = \sum_{n=1}^{\infty} (I-T)^n = \sum_{n=0}^{\infty} (I-T)^n - I = X-I. \]
			This implies
			\[ TX = X. \]
			With a similar argument we can get $ X(I-T) = X-I $, thus $ XT = I $. So we conclude that $ X $ is the right and the left inverse of $ T $, thus $ T $ is a bijection and $ X = \inv{T} $.
			\begin{remark}
				I think in above, when we proved that $ X \in L(\mathcal{X},\mathcal{X}) $, we automatically proved that $ \inv{T} $ is bounded. However, in the solution manual that I got the idea of proof, the author separately proves that $ \inv{T} $ is bounded. For the sake of completeness I will do the same here as well.
			\end{remark}
			To show that $ \inv{T} $ is bounded, consider the sequence of partial sums of $ \inv{T} $
			\[ S_n = \sum_{i=1}^n (I-T)^n. \]
			Using the continuity of $ \norm{\cdot} $ we can write
			\[ \norm{\inv{T}x} = \norm{\lim_{n} S_i x} = \lim_{n} \norm{S_n x} \leq \lim_n \sum_{i=0}^n \norm{(I-T)^nx} \leq \lim_n \sum_{i=0}^n \norm{I-T}^n \norm{x} \leq \frac{\norm{x}}{1-\gamma}. \]
			
			\item Observe that 
			\[ \norm{(S\inv{T} - I)} = \norm{(S\inv{T}-I)T\inv{T}} = \norm{S\inv{T} - T\inv{T}} \leq \norm{(S-T)}\norm{\inv{T}} < 1. \]
			So $ S\inv{T} $ has an inverse $ A \in L(\mathcal{X},\mathcal{X}) $ and we have $ A = T\inv{S} $. So $ \inv{S} = \inv{T}A $. It is also easy to see that $ \inv{S} $ is bounded. Because
			\[ \norm{\inv{S}x} \leq \norm{\inv{T}}\norm{A}\norm{x}. \]
		\end{enumerate}
		
	\end{solution}
\end{problem}