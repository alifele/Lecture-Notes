\chapter{Sequences and Series of Functions}

In this chapter we will study the sequence and series of functions. We will study different notions of convergence, namely point-wise convergence, uniform convergence, etc. Note that in this chapter, we will try to be as concrete as possible, avoiding unnecessary abstraction.

\begin{definition}[Point-wise convergence]
	Let $\seq{f}$ be a sequence of functions $f: [0,1] \to \R$. Then we say that this sequence converges \emph{point-wise} to the function $f[0,1]\to\R$, and show it as $f_n(x)\to f$ if
	\[ \forall \epsilon>0, x\in [0,1],\ \exists N>0 \st \forall n>N \wh \abs{f_n(x) - f(x)} < \epsilon.  \]
	We write this as $\lim_{n\to\infty}f_n(x) = f(x)$.
\end{definition}
\begin{remark}
	Note the order of the quantifiers in the definition above. What we are basically saying is that for any choice of $\epsilon$ and for any $x$, we can find $N$ (that depends on $\epsilon$ and $x$) such that for all $n>N$ we have $\abs{f_n(x)-f(x)}<\epsilon$. There is \textbf{NO} guarantee that by using this $N$, for any other point $x'$ we have $\abs{f_n(x') - f(x)} < \epsilon$.
\end{remark}

Thinking more carefully, we can see that this is nothing other than a collection of sequences in $\R$ that are indexed by index the set $[0,1]$. For instance, for $x=0.5$, $\set{f_n(0.5)}_n$ is a sequence in $\R$ like any other value of $x\in [0,1]$. That is why a sequence of functions can be thought of as an indexed sequence. And the point-wise notion of convergence to $f(x)$ is nothing other than indexing the converged value by $x$. Back to the previous example, for $x=0.5$, if we know that $\set{f_n(0.5)}_n$ convergence, the symbol $f(0.5)$ is the most appropriate symbol to represent the limit. Then for $x=0.6$, if $\seq{f_n(x)}$ converges, we can say it converges to $f(0.6)$. So the point-wise convergence is simply saying that all of the indexed sequences converge to a set whose values are indexed by the same index as the index of the corresponding sequence. 

Because of our discussion above, it is not surprising if we say that there is no guarantee that some good properties of $f_n$ carries over to $f$ (like continuity, differentiability, integrability, etc). That is because the point-wise notion of convergence is only expressing the convergence of a bunch of sequences (indexed with real numbers in [0,1]) in a neat way.

As we will discover through this chapter, it turns out that the question about if the properties of the function $f_n$ carries over the properties of the $f$ through the point-wise convergence, is essentially the same type of question that if in a multivariate limit, exchange of limit order is allowed. 


\begin{definition}[Uniform convergence]
	let $\set{f_n}$ be a sequence of functions $f:[0,1] \to \R$. Then we say that the sequence converges to $f: [0,1] \to \R$ if 
	\[ \forall \epsilon>0,\ \exists N>0 \st \forall n>N \wh \qquad \abs{f_n(x)-f(x)} < \epsilon \quad \forall x\in [0,1]. \]
	We show this by $f_n \to f$.
\end{definition}
\begin{remark}
	Note the order of the quantifiers and compare that closely with the definition of the point-wise convergence. In the uniform convergence, for any choice of $\epsilon$, we can find $N$ that for all $n>N$ we have $\abs{f_n(x)- f(x)} <\epsilon$ that holds true for any $x\in [0,1]$. Thus in some sense, the function $f_n$ converges to $f$ as a whole, and not in a point by point sense. This is also evident in the notation that we use to demonstrate the uniform convergence $f_n \to f$, that gives the feeling that the functions $f_n$ converges to $f$ as a whole object.
\end{remark}

\begin{theorem}[Cauchy criteria of convergence]
	Let $\set{f_n}_n$ be a sequence of functions $f_n: E \to \R$. This sequence converges to $f$ uniformally, if and only if $\forall \epsilon>0,\ \exists N>0 \st \forall n,m>N \wh \abs{f_n(x) - f_m(x)} < \epsilon \ \forall x\in E$.
\end{theorem}
\begin{proof}
	$\boxed{\implies}$ For this direction, we need to show that uniform convergence imply the sequence to be Cauchy. Let $\epsilon>0$ given. Then since $f_n \to f$, then we can find integer $N$ such that for all $n,m>N$ and $x\in E$
	\[ \abs{f_n(x)-f(x)}<\epsilon/2, \qquad \abs{f_m(x)-f(x)}<\epsilon/2 \]
	Thus we can write
	\[ \abs{f_n(x) - f_m(x)} \leq \abs{f_n(x)-f(x)} + \abs{f_m(x)+f(x)} < \epsilon \]
	
	\noindent $\boxed{\Longleftarrow}$ For this direction, we need to prove that the Cauchy criteria satisfied implies the uniform convergence. First, observe that for any fixed $x$, the sequence $\set{f_n(x)}$ is Cauchy in $\R$. Thus it converges to some real number $f(x)$. Given $\epsilon>0$ choose $N$ large enough that $\forall n,m > N$ we get
	\[ \abs{f_n(x) - f_m(x)}  < \epsilon/2 \qquad \forall x\in E.\]
	Then, for all $x\in E$ we can write
	\[ \abs{f_n(x) - f(x)} \leq \abs{f_n(x) - f_m(x)} + \abs{f_m(x) - f(x)}. \]
	The first term in the right hand side is less than $\epsilon/2$. And since $f_m(x) \to f(x)$, we can choose $m$ large enough that $\abs{f_m(x) - f(x)} < \epsilon/2$. Thus we will get
	\[ \abs{f_n(x) - f(x)} < \epsilon. \]
	This completes our proof. 
\end{proof}