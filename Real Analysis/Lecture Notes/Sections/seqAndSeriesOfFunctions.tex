\chapter{Sequences and Series of Functions}

\section{Basic Notions and Definitions}

In this chapter we will study the sequence and series of functions. We will study different notions of convergence, namely point-wise convergence, uniform convergence, etc. Note that in this chapter, we will try to be as concrete as possible, avoiding unnecessary abstraction.

\begin{definition}[Point-wise convergence]
	Let $\seq{f}$ be a sequence of functions $f: [0,1] \to \R$. Then we say that this sequence converges \emph{point-wise} to the function $f[0,1]\to\R$, and show it as $f_n(x)\to f$ if
	\[ \forall \epsilon>0, x\in [0,1],\ \exists N>0 \st \forall n>N \wh \abs{f_n(x) - f(x)} < \epsilon.  \]
	We write this as $\lim_{n\to\infty}f_n(x) = f(x)$.
\end{definition}
\begin{remark}
	Note the order of the quantifiers in the definition above. What we are basically saying is that for any choice of $\epsilon$ and for any $x$, we can find $N$ (that depends on $\epsilon$ and $x$) such that for all $n>N$ we have $\abs{f_n(x)-f(x)}<\epsilon$. There is \textbf{NO} guarantee that by using this $N$, for any other point $x'$ we have $\abs{f_n(x') - f(x)} < \epsilon$.
\end{remark}

Thinking more carefully, we can see that this is nothing other than a collection of sequences in $\R$ that are indexed by index the set $[0,1]$. For instance, for $x=0.5$, $\set{f_n(0.5)}_n$ is a sequence in $\R$ like any other value of $x\in [0,1]$. That is why a sequence of functions can be thought of as an indexed sequence. And the point-wise notion of convergence to $f(x)$ is nothing other than indexing the converged value by $x$. Back to the previous example, for $x=0.5$, if we know that $\set{f_n(0.5)}_n$ convergence, the symbol $f(0.5)$ is the most appropriate symbol to represent the limit. Then for $x=0.6$, if $\seq{f_n(x)}$ converges, we can say it converges to $f(0.6)$. So the point-wise convergence is simply saying that all of the indexed sequences converge to a set whose values are indexed by the same index as the index of the corresponding sequence. 

Because of our discussion above, it is not surprising if we say that there is no guarantee that some good properties of $f_n$ carries over to $f$ (like continuity, differentiability, integrability, etc). That is because the point-wise notion of convergence is only expressing the convergence of a bunch of sequences (indexed with real numbers in [0,1]) in a neat way.

As we will discover through this chapter, it turns out that the question about if the properties of the function $f_n$ carries over the properties of the $f$ through the point-wise convergence, is essentially the same type of question that if in a multivariate limit, exchange of limit order is allowed. 


\begin{definition}[Uniform convergence]
	let $\set{f_n}$ be a sequence of functions $f:[0,1] \to \R$. Then we say that the sequence converges to $f: [0,1] \to \R$ if 
	\[ \forall \epsilon>0,\ \exists N>0 \st \forall n>N \wh \qquad \abs{f_n(x)-f(x)} < \epsilon \quad \forall x\in [0,1]. \]
	We show this by $f_n \to f$.
\end{definition}
\begin{remark}
	Note the order of the quantifiers and compare that closely with the definition of the point-wise convergence. In the uniform convergence, for any choice of $\epsilon$, we can find $N$ that for all $n>N$ we have $\abs{f_n(x)- f(x)} <\epsilon$ that holds true for any $x\in [0,1]$. Thus in some sense, the function $f_n$ converges to $f$ as a whole, and not in a point by point sense. This is also evident in the notation that we use to demonstrate the uniform convergence $f_n \to f$, that gives the feeling that the functions $f_n$ converges to $f$ as a whole object.
\end{remark}

\begin{theorem}[Cauchy criteria of convergence]
	Let $\set{f_n}_n$ be a sequence of functions $f_n: E \to \R$. This sequence converges to $f$ uniformally, if and only if $\forall \epsilon>0,\ \exists N>0 \st \forall n,m>N \wh \abs{f_n(x) - f_m(x)} < \epsilon \ \forall x\in E$.
\end{theorem}
\begin{proof}
	$\boxed{\implies}$ For this direction, we need to show that uniform convergence imply the sequence to be Cauchy. Let $\epsilon>0$ given. Then since $f_n \to f$, then we can find integer $N$ such that for all $n,m>N$ and $x\in E$
	\[ \abs{f_n(x)-f(x)}<\epsilon/2, \qquad \abs{f_m(x)-f(x)}<\epsilon/2 \]
	Thus we can write
	\[ \abs{f_n(x) - f_m(x)} \leq \abs{f_n(x)-f(x)} + \abs{f_m(x)+f(x)} < \epsilon \]
	
	\noindent $\boxed{\Longleftarrow}$ For this direction, we need to prove that the Cauchy criteria satisfied implies the uniform convergence. First, observe that for any fixed $x$, the sequence $\set{f_n(x)}$ is Cauchy in $\R$. Thus it converges to some real number $f(x)$. Given $\epsilon>0$ choose $N$ large enough that $\forall n,m > N$ we get
	\[ \abs{f_n(x) - f_m(x)}  < \epsilon/2 \qquad \forall x\in E.\]
	Then, for all $x\in E$ we can write
	\[ \abs{f_n(x) - f(x)} \leq \abs{f_n(x) - f_m(x)} + \abs{f_m(x) - f(x)}. \]
	The first term in the right hand side is less than $\epsilon/2$. And since $f_m(x) \to f(x)$, we can choose $m$ large enough that $\abs{f_m(x) - f(x)} < \epsilon/2$. Thus we will get
	\[ \abs{f_n(x) - f(x)} < \epsilon. \]
	This completes our proof. 
\end{proof}

The following criterion often comes very handy and useful.
\begin{proposition}
	Let $\set{f_n}$ be a sequence of functions defined on $E$. Define 
	\[ M_n = \sup_{x\in E}\abs{f_n(x) - f(x)}. \]
	Then $f_n \to f$ if and only if $M_n \to 0$.
\end{proposition}.
\begin{proof}
	\noindent $\boxed{\Longleftarrow}$ For this direction, as assume $M_n \to 0$ and we prove that $f_n \to f$. For given $\epsilon>0$ we can find $N>0$ such that for all $n>N$ we have
	\[ M_n < \epsilon. \]
	By the properties of the suprimum, we know that $\forall x\in E$ we have $\abs{f_n(x) - f(x)} < M_n$. This implies 
	\[ \abs{f_n(x) - f(x)}  < \epsilon \qquad \forall x\in E.\]
	Thus we conclude $f_n \to f$.
	
	\noindent $\boxed{\Longrightarrow}$ For this direction we assume $f_n \to f$ and we prove that $M_n \to 0$. For a given $\epsilon>0$ we can find $N>0$ such that for all $n>N$ we have
	\[ \abs{f_n(x) - f(x)} < \epsilon \qquad \forall x\in E. \]
	Then it follows from the definition of suprimum that 
	\[ M_n = \sup_{x\in E} \abs{f_n(x) - f(x)}  \leq \epsilon. \]
	This implies that $M_n \to 0$ as $n\to \infty$.
 \end{proof}


\section{Uniform Convergence and Continuity}
It is very easy to show that by point-wise convergence, the continuity of the functions does not carry over to the limiting function. Consider the following example.
\begin{example}
	Let $\set{f_n}$ be a sequence of functions $f:[0,1]\to \R$ where $f_n(x) = x^n$. Then it is immediate that $f_n(x) \to 0$ for all $x \in [0,1)$ and $f_n(1) \to 1$. Thus $f_n(x) \to f(x)$ where
	\[ f(x) = \begin{cases}
		0 \quad x \in [0,1), \\
		1 \quad x \in 1.
	\end{cases} \]
	It is clear that all of $f_n$ are continuous but the limiting function $f$ is not. 
\end{example}

The example above is a very simple and good example where the continuity of a sequence of functions does not carry over to the limiting function. The following remark also shows the fact that asking about the continuity of the limiting function is in fact a question about the possibility of the exchange of the orders of the limit.

\begin{remark}
	Let $\set{f_n}$ be a set of functions defined on $E$. Assume that all of these functions are continuous on $E$, i.e. $\forall x\in E$ we have
	\[ \lim_{t\to x}f_n(t) = f_n(x), \]
	and by using the fact that $f_n(x) \to f$ we can write
	\[ \lim_{n\to \infty} \lim_{x\to t} f_n(t) = \lim_{n\to\infty}f_n(x)  = f(x). \]
	However, if $f(x)$ is continuous, then we can write
	\[ \lim_{t\to x} f(t) =f(x). \]
	but since $f_n(t) \to f(t)$, then we can re-write the expression above as
	\[ \lim_{t\to x} \lim_{n\to\infty} f_n(t)= f(x). \]
	Combining this with the result above we can get
	\[  \boxed{\lim_{n\to \infty} \lim_{x\to t} f_n(t) \stackrel{?}{=} \lim_{x\to t} \lim_{n\to \infty} f_n(t) }. \]
\end{remark}

The following theorem shows the relation between uniform convergence and continuity.

\begin{proposition}
	Let $\set{f_n}$ be a sequence of functions defined on $E$ a subset of a metric space $(M,d)$. Assume $f_n \to f$ uniformally where $f$ is also defined on $E$. Let $x$ be a limit point of $E$ and suppose that 
	\[ \lim_{t\to x}f_n(t) = A_n \qquad  (n=1,2,3,\cdots). \]
	Then $\set{A_n}$ converges and 
	\[ \lim_{n\to \infty} A_n = \lim_{t\to x} f(t), \]
	which is equivalent to 
	\[ \lim_{n\to \infty} \lim_{t\to x} f_n(t) = \lim_{t\to x} \lim_{n\to \infty} f_n(t). \]
\end{proposition}
\begin{proof}
	First, we need to prove that the sequence $\set{A_n}$ converges. We will show $\set{A_n}$ is Cauchy by using the fact that $f_n \to f$ uniformally. We can do this in two way, where the first one is much more high level (in the sense that it is close to everyday conversation) and the second way is much more verbose revealing the underlying gears. I have seen that Rudin sometimes uses the first way of reasoning in his proofs which is very fast and easy to follow. \\
	Let $\epsilon>0$ given. Then $\exists N>0$ such that $\forall n,m>N$ we have
	\[ \abs{f_n(t) - f_m(t)} < \epsilon \qquad \forall t\in E. \]
	Now taking limit $t\to x$ we will get
	\[ \abs{A_n - A_m} < \epsilon. \]
	This shows that $\set{A_n}$ is Cauchy, this converges. \\
	To reveal what is happening behind the scene of the reasoning above (when we simply say taking limit $t\to x$), we provide the following reasoning. Given $\epsilon>0$, choose $N_1$ large enough that $\forall n,m>N_1$ we have 
	\[ \abs{f_n(t) - f_m(t)} < \epsilon/3 \qquad \forall t\in E. \]
	We can always find such $N_1$ because $f_n \to f$ by hypothesis. Now choose $t$ close enough to $x$ to get
	\[ \abs{f_n(t) - A_n} < \epsilon/3,  \qquad \abs{f_m(t) - A_m} < \epsilon/3. \]
	Then we can write
	\[ \abs{A_n - A_m} = \abs{A_n - A_m \pm f_n(t) \pm f_m(t)} \leq  \abs{f_n(t) - A_n} + \abs{f_m(t) - A_m} + \abs{f_n(t) - f_m(t)}. \]
	
	TO BE CONTINUED
	
	
\end{proof}


















\newpage