\chapter{Sequences and Series of Functions}

\section{Basic Notions and Definitions}

In this chapter we will study the sequence and series of functions. We will study different notions of convergence, namely point-wise convergence, uniform convergence, etc. Note that in this chapter, we will try to be as concrete as possible, avoiding unnecessary abstraction.

\begin{definition}[Point-wise convergence]
	Let $\seq{f}$ be a sequence of functions $f: [0,1] \to \R$. Then we say that this sequence converges \emph{point-wise} to the function $f:[0,1]\to\R$, and show it as $f_n(x)\to f(x)$ if
	\[ \forall \epsilon>0, x\in [0,1],\ \exists N>0 \st \forall n>N \wh \abs{f_n(x) - f(x)} < \epsilon.  \]
	We write this as $\lim_{n\to\infty}f_n(x) = f(x)$.
\end{definition}
\begin{remark}
	Note the order of the quantifiers in the definition above. What we are basically saying is that for any choice of $\epsilon$ and for any $x$, we can find $N$ (that depends on $\epsilon$ and $x$) such that for all $n>N$ we have $\abs{f_n(x)-f(x)}<\epsilon$. There is \textbf{NO} guarantee that by using this $N$, for any other point $x'$ we have $\abs{f_n(x') - f(x)} < \epsilon$.
\end{remark}

Thinking more carefully, we can see that this is nothing other than a collection of sequences in $\R$ that are indexed by index the set $[0,1]$. For instance, for $x=0.5$, $\set{f_n(0.5)}_n$ is a sequence in $\R$ like any other value of $x\in [0,1]$. That is why a sequence of functions can be thought of as an indexed sequence. And the point-wise notion of convergence to $f(x)$ is nothing other than indexing the converged value by $x$. Back to the previous example, for $x=0.5$, if we know that $\set{f_n(0.5)}_n$ convergence, the symbol $f(0.5)$ is the most appropriate symbol to represent the limit. Then for $x=0.6$, if $\seq{f_n(x)}$ converges, we can say it converges to $f(0.6)$. So the point-wise convergence is simply saying that all of the indexed sequences converge to a set whose values are indexed by the same index as the index of the corresponding sequence. 

Because of our discussion above, it is not surprising if we say that there is no guarantee that some good properties of $f_n$ carries over to $f$ (like continuity, differentiability, integrability, etc). That is because the point-wise notion of convergence is only expressing the convergence of a bunch of sequences (indexed with real numbers in [0,1]) in a neat way.

As we will discover through this chapter, it turns out that the question about if the properties of the function $f_n$ carries over the properties of the $f$ through the point-wise convergence, is essentially the same type of question that if in a multivariate limit, exchange of limit order is allowed. 


\begin{example}[Point-wise convergence does not guarantee differentiability]
	Let $ f:\R \to \R $ where
	\[ f(x) = \frac1n \sin(2^n x). \]
	For this function we have
	\[ \lim_{n\to\infty} f_n(x) = 0, \quad \forall x \in \R. \]
	However, by calculating the derivative, we will have
	\[ f'(x) = \frac{2^n}{n}\cos(2^nx),  \]
	that does not converge point wise.
\end{example}

\begin{example}[Point-wise convergence does not guarantee continuity]
	\label{exmp:PointwiseLimCont}
	Let $\set{f_n}$ be a sequence of functions $f:[0,1]\to \R$ where $f_n(x) = x^n$. Then it is immediate that $f_n(x) \to 0$ for all $x \in [0,1)$ and $f_n(1) \to 1$. Thus $f_n(x) \to f(x)$ where
	\[ f(x) = \begin{cases}
		0 \quad x \in [0,1), \\
		1 \quad x \in 1.
	\end{cases} \]
	It is clear that all of $f_n$ are continuous but the limiting function $f$ is not. 
\end{example}

\begin{example}[Point-wise convergence does not guarantee integrablity]
	\label{exmp:PointWiseContNotPreserveInt}
	Let $ f:(0,1] \to \R $ defined as 
	\[ f_n(x) = \begin{cases}
		n \qquad 0<x<\frac1n,\\
		0 \qquad \frac1n \leq x \leq 1
	\end{cases} \]
	This function converges to $ f(x) \equiv 0 $ point wise on $ (0,1] $. However, 
	\[ \lim_{n\to\infty}\int_{0}^{1}f_n(x) = 1 \]
	whereas
	\[ \int_{0}^{1}f(x) = 0. \]
	Thus the limiting function (point-wise) do not have the same integral as the limit of the integrals of the functions in the sequence. z1
\end{example}





\begin{definition}[Uniform convergence]
	let $\set{f_n}$ be a sequence of functions $f:[0,1] \to \R$. Then we say that the sequence converges to $f: [0,1] \to \R$ if 
	\[ \forall \epsilon>0,\ \exists N>0 \st \forall n>N \wh \qquad \abs{f_n(x)-f(x)} < \epsilon \quad \forall x\in [0,1]. \]
	We show this by $f_n \to f$.
\end{definition}
\begin{remark}
	Note the order of the quantifiers and compare that closely with the definition of the point-wise convergence. In the uniform convergence, for any choice of $\epsilon$, we can find $N$ that for all $n>N$ we have $\abs{f_n(x)- f(x)} <\epsilon$ that holds true for any $x\in [0,1]$. Thus in some sense, the function $f_n$ converges to $f$ as a whole, and not in a point by point sense. This is also evident in the notation that we use to demonstrate the uniform convergence $f_n \to f$, that gives the feeling that the functions $f_n$ converges to $f$ as a whole object.
\end{remark}

\begin{theorem}[Cauchy criteria of convergence]
	\label{thm:cauchyConv}
	Let $\set{f_n}_n$ be a sequence of functions $f_n: E \to \R$. This sequence converges to $f$ uniformally, if and only if $\forall \epsilon>0,\ \exists N>0 \st \forall n,m>N \wh \abs{f_n(x) - f_m(x)} < \epsilon \ \forall x\in E$.
\end{theorem}
\begin{proof}
	$\boxed{\implies}$ For this direction, we need to show that uniform convergence imply the sequence to be Cauchy. Let $\epsilon>0$ given. Then since $f_n \to f$, then we can find integer $N$ such that for all $n,m>N$ and $x\in E$
	\[ \abs{f_n(x)-f(x)}<\epsilon/2, \qquad \abs{f_m(x)-f(x)}<\epsilon/2 \]
	Thus we can write
	\[ \abs{f_n(x) - f_m(x)} \leq \abs{f_n(x)-f(x)} + \abs{f_m(x)+f(x)} < \epsilon \]
	
	\noindent $\boxed{\Longleftarrow}$ For this direction, we need to prove that the Cauchy criteria satisfied implies the uniform convergence. First, observe that for any fixed $x$, the sequence $\set{f_n(x)}$ is Cauchy in $\R$. Thus it converges to some real number $f(x)$. Given $\epsilon>0$ choose $N$ large enough that $\forall n,m > N$ we get
	\[ \abs{f_n(x) - f_m(x)}  < \epsilon/2 \qquad \forall x\in E.\]
	Then, for all $x\in E$ we can write
	\[ \abs{f_n(x) - f(x)} \leq \abs{f_n(x) - f_m(x)} + \abs{f_m(x) - f(x)}. \]
	The first term in the right hand side is less than $\epsilon/2$. And since $f_m(x) \to f(x)$, we can choose $m$ large enough that $\abs{f_m(x) - f(x)} < \epsilon/2$. Thus we will get
	\[ \abs{f_n(x) - f(x)} < \epsilon. \]
	This completes our proof. 
\end{proof}

The following criterion often comes very handy and useful.
\begin{proposition}
	\label{prop:uniformConvSup}
	Let $\set{f_n}$ be a sequence of functions defined on $E$ that converges point-wise to $ f(x) $. Define 
	\[ M_n = \sup_{x\in E}\abs{f_n(x) - f(x)}. \]
	Then $f_n \to f$ if and only if $M_n \to 0$.
\end{proposition}.
\begin{proof}
	\noindent $\boxed{\Longleftarrow}$ For this direction, as assume $M_n \to 0$ and we prove that $f_n \to f$. For given $\epsilon>0$ we can find $N>0$ such that for all $n>N$ we have
	\[ M_n < \epsilon. \]
	By the properties of the suprimum, we know that $\forall x\in E$ we have $\abs{f_n(x) - f(x)} < M_n$. This implies 
	\[ \abs{f_n(x) - f(x)}  < \epsilon \qquad \forall x\in E.\]
	Thus we conclude $f_n \to f$.
	
	\noindent $\boxed{\Longrightarrow}$ For this direction we assume $f_n \to f$ and we prove that $M_n \to 0$. For a given $\epsilon>0$ we can find $N>0$ such that for all $n>N$ we have
	\[ \abs{f_n(x) - f(x)} < \epsilon \qquad \forall x\in E. \]
	Then it follows from the definition of suprimum that 
	\[ M_n = \sup_{x\in E} \abs{f_n(x) - f(x)}  \leq \epsilon. \]
	This implies that $M_n \to 0$ as $n\to \infty$.
 \end{proof}


\section{Uniform Convergence and Continuity}
As we say in \autoref{exmp:PointwiseLimCont} it is very easy to show that by point-wise convergence, the continuity of the functions does not carry over to the limiting function. This example is a very simple and good example where the continuity of a sequence of functions does not carry over to the limiting function. The following remark also shows the fact that asking about the continuity of the limiting function is in fact a question about the possibility of the exchange of the orders of the limit.

\begin{remark}
	Let $\set{f_n}$ be a set of functions defined on $E$. Assume that all of these functions are continuous on $E$, i.e. $\forall x\in E$ we have
	\[ \lim_{t\to x}f_n(t) = f_n(x), \]
	and by using the fact that $f_n(x) \to f$ we can write
	\[ \lim_{n\to \infty} \lim_{x\to t} f_n(t) = \lim_{n\to\infty}f_n(x)  = f(x). \]
	However, if $f(x)$ is continuous, then we can write
	\[ \lim_{t\to x} f(t) =f(x). \]
	but since $f_n(t) \to f(t)$, then we can re-write the expression above as
	\[ \lim_{t\to x} \lim_{n\to\infty} f_n(t)= f(x). \]
	Combining this with the result above we can get
	\[  \boxed{\lim_{n\to \infty} \lim_{x\to t} f_n(t) \stackrel{?}{=} \lim_{x\to t} \lim_{n\to \infty} f_n(t) }. \]
\end{remark}

The following theorem shows the relation between uniform convergence and continuity.

\begin{theorem}[Uniform convergence allows the exchange of the order of the limits]
	\label{thm:uniformContExchangeLimit}
	Let $\set{f_n}$ be a sequence of functions defined on $E$ a subset of a metric space $(M,d)$. Assume $f_n \to f$ uniformally where $f$ is also defined on $E$. Let $x$ be a limit point of $E$ and suppose that 
	\[ \lim_{t\to x}f_n(t) = A_n \qquad  (n=1,2,3,\cdots). \]
	Then $\set{A_n}$ converges and 
	\[ \lim_{n\to \infty} A_n = \lim_{t\to x} f(t), \]
	which is equivalent to 
	\[ \lim_{n\to \infty} \lim_{t\to x} f_n(t) = \lim_{t\to x} \lim_{n\to \infty} f_n(t). \]
\end{theorem}
\begin{proof}
	{\color{orange} First part: }\\
	First, we need to prove that the sequence $\set{A_n}$ converges. We will show $\set{A_n}$ is Cauchy by using the fact that $f_n \to f$ uniformally. We can do this in two way, where the first one is much more high level (in the sense that it is close to everyday conversation) and the second way is much more verbose revealing the underlying gears. I have seen that Rudin sometimes uses the first way of reasoning in his proofs which is very fast and easy to follow. \\
	Let $\epsilon>0$ given. Then $\exists N>0$ such that $\forall n,m>N$ we have
	\[ \abs{f_n(t) - f_m(t)} < \epsilon \qquad \forall t\in E. \]
	Now taking limit $t\to x$ we will get
	\[ \abs{A_n - A_m} < \epsilon. \]
	This shows that $\set{A_n}$ is Cauchy, this converges. \\
	To reveal what is happening behind the scene of the reasoning above (when we simply say taking limit $t\to x$), we provide the following reasoning. Given $\epsilon>0$, choose $N_1$ large enough that $\forall n,m>N_1$ we have 
	\[ \abs{f_n(t) - f_m(t)} < \epsilon/3 \qquad \forall t\in E. \]
	We can always find such $N_1$ because $f_n \to f$ by hypothesis. \\
	Now, let $ x $ be a limit point of $ E $. Then by choosing $ t $ close enough to $  x $ we will get
	\[ \abs{f_n(t) - A_n} < \epsilon/3,  \qquad \abs{f_m(t) - A_m} < \epsilon/3. \]
	For all $ t \in E $ we can write
	\[ \abs{A_n - A_m} = \abs{A_n - A_m \pm f_n(t) \pm f_m(t)} \leq  \abs{f_n(t) - A_n} + \abs{f_m(t) - A_m} + \abs{f_n(t) - f_m(t)}. \]
	Now by choosing $ n,m $ large enough, and $ t $ close enough to $ x $ we will get
	\[ \abs{A_n - A_m} < \epsilon. \]
	This shows that the sequence $ \set{A_n} $ is a real Cauchy sequence, thus converges. \\
	{\color{orange} Second part: }\\
	\noindent Now, we want to show that $ \lim_{t\to x}f(t) \to A $ as $ n\to \infty $, where $ x $ is a limit point of $ E $, and $ A $ is the limit of $ A_n $ as $ n\to \infty $. I.e. $ \lim_{n\to\iunfty}A_n= A $. For given $\epsilon>0$ and all $ t \in E $, we can write
	\[ \abs{f(t) - A} \leq \underbrace{\abs{A - f_n(t)}}_{\leq \abs{A_n - A}+\abs{A_n-f_n(t)}} + \abs{f(t) - f_n(t)} \leq \abs{A_n - A}+\abs{A_n-f_n(t)} + \abs{f(t) - f_n(t)}. \]
	We are in a good shape now, since each of the terms above can be controlled. To be more specific, since $ f_n \to f $ uniformally, and $ \lim_{n\to\infty} A_n = A $, then we can choose $ n $ large enough such that 
	\[ \abs{f(t) - f_n(t)} < \frac\epsilon3, \quad \abs{A_n - A} < \frac\epsilon3  , \qquad \forall t\in E. \]
	And for this choice of $ n $, since $ \lim_{t \to x}f_n(t)=A_n$, we can choose $ t $ close enough to $ x $ such that 
	\[ \abs{A_n - f_n(t)} < \frac\epsilon3. \]
	Putting all the pieces together, we shown that for given $ \epsilon>0 $, we can choose $ t $ close enough to $ x $ such that 
	\[ \abs{f(t) -  A} < \epsilon.  \]
	Thus $ f(t) \to A $ as $ t \to x $. Thus in a nutshell
	\[ \lim_{n\to\infty}\lim_{t\to x}f_n(t) = \lim_{t\to x}\lim_{n\to\infty}f_n(t). \]
\end{proof}

The theorem that we proved above is a very important theorem, as many important conclusions will follow as a kind of immediate corollary from that. The following important result is one of those.

\begin{corollary}
	\label{cor:UniformConvImpCont}
	Let $ \set{f_n} $ be a sequence of continuous functions that converges uniformally to $ f $ on $ E $. Then $ f $ is continuous.
\end{corollary}
\begin{proof}
	This is an immediate corollary of the \autoref{thm:uniformContExchangeLimit}. To be more explicit, since $ f_n \to f $ uniformally on $ E $, then we can do the following exchange of limits.
	\[ \lim_{n\to\infty}\underbrace{\lim_{t\to x}f_n(t)}_{f_n(x)} = \lim_{t\to x}\underbrace{\lim_{n\to\infty}f_n(t)}_{f(t)}. \]
	where we have used the fact that $ f_n $ are all continuous. Thus we have 
	\[ \lim_{t\to x} f(t) = f(x). \]
	This implies the continuoity of $ f $ and completes the proof.
\end{proof}

\begin{remark}
	Note that the converse of the corollary above is not true. I.e. we can have a set of continuous functions $ \set{f_n} $ that converges point-wise to a continuous function $ f  $, but the converges fails to be uniform. The case we studied in \autoref{exmp:PointWiseContNotPreserveInt} is of this type. You can easily see this fails to be a uniform convergence by applying \autoref{prop:uniformConvSup}.
\end{remark}

In the remark above we say that the converse of the corollary above is not always true. However, the following proposition provides the necessary conditions under which the converse of the corollary above is also true. 


\begin{proposition}
	Let $ K $ be a \textbf{compact} set, and 
	\begin{enumerate}[(i),noitemsep]
		\item $ \set{f_n} $ is a sequence of \textbf{continuous} functions.
		\item $ f_n(x) $ converges to $ f(x) $ \textbf{ point-wise} on $ K $, where $ f $ is also \textbf{continuous}.
		\item $ f_n(x) \geq f_{n+1}(x) $ for all $ x\in K $ and $ n=1,2,\hdots. $
	\end{enumerate}
	Then $ f_n $ is converging uniformally to $ f $ on $  K $.
\end{proposition}

\begin{proof}
	Define an auxiliary function $ g_n(x) = f_n(x)-f$. Because $ f_n(x) \geq f_{n+1}(x) $ for all $ x \in K $, and since $ f_n(x)\to f(x) $ for all $ x\in K $ then we conclude that $ g_n(x) \geq g_{n+1}(x) $. Also, observe that because $ f_n $ and $ f $ are both continuous on $ K $, then $ g_n $ is also continuous on $ K $ for all $ n \in \N $. To prove the theorem, it is equivalent to prove that $ g_n \to 0$ uniformally. 
	
	\noindent Let $\epsilon>0$ given. Define 
	\[ K_n = \set{x\in K:\ g_n(x) \geq \epsilon}. \]
	Since $ g_n $ is continuous, then $ K_n $ is closed. To show this, consider $ K_n^c $, which is the set of point $ x \in K $ such that $ g_n(x)<\epsilon $. Since $ g_n $ is continouse, we can make small enough perturbation to $  x $ and still the value of function $ g_n $ less than $ \epsilon $. Thus $ K_n^c $ is open, implying the set $ K_n $ is closed. Since $ K_n \subseteq K $ and $ K_n  $ is closed, thus $ K_n $ is also compact. On the other hand, since $ g_n(x) \geq g_{n+1}(x)  $, then we have 
	$ K_n \supset K_{n+1} $.  
	
	\noindent Since $ g_n(x) \to 0 $ point wise, then for any $ x \in K $,  we have $ x \notin K_n $ if $ n $ is large enough. Thus $ x \notin \bigcap K_n $. This implies $ \bigcap K_n = \emptyset $. From the nested intervals theorem for real intervals, then we conclude that $ K_N = \emptyset$ for all $ N $ large enough (otherwise, their infinite intersection will contain at least one element). This proves that for all $ n $ large enough, we have
	\[ 0 \leq g_n(x) < \epsilon. \]
	This implies the uniform convergence. 
\end{proof}


The following example emphasizes that fact that the compactness is really needed for the proposition above.

\begin{example}
	The function $ f_n(x)= \frac{1}{nx + 1}$ converges point wise and monotonically to $ 0 $ on (0,1), but the convergence fails to be uniform.
\end{example}

\subsection{Exploiting some of the algebraic structures}
At this stage, we can exploit some of the algebraic structures of the objects we are working with. To do this, we first start with the following definition.

\begin{definition}
	Let $ X $ be a metric space. We denote with $ C(X) $ the following set of functions.
	\[ C(X) = \set{f: X\to \C:\ f \text{ is \textbf{continuous} and \textbf{bounded}}}. \]
\end{definition}
\begin{remark}
	Note that if $ X $ is compact, then boundedness of $ f $ is for free.
\end{remark}

The following proposition reveals the underlying algebraic structure of this set. 

\begin{proposition}
	The set $ C(X) $ is a vector space for which we can define the following norm
	\[ \norm{f} = \sup_{x\in X}\abs{f(x)}. \]
\end{proposition}
\begin{proof}
	{\color{orange} Part 1: Showing $ C(X) $ is a vector space.  }
	It is very straight forward to show this, and we just need to check the properties of the vector spaces to see if they are satisfied. Here, I will point out the important properties. First, note that we have a zero element, which is the zero function. Also, note that if a function $ f $ is continouse and bounded, then the function $ -f $ defined as $ -f(x) =  -1 * f(x) $. The rest is very easy to verify.
	
	\noindent {\color{orange} Showing $ \norm{f} $ satisfies the norm properties. }
	First, we need to show that the norm defined as above is positive definite, i.e. $ \norm{f}=0 $ if and only if $ f\equiv 0 $. $ f\equiv0 $ implying $ \norm{f} = 0 $ is trivial, and $ \norm{f} = 0 $ implying $ f\equiv0 $ follows from the definition of the suprimum. Also, from definition of suprimum is follows that $ \norm{kf} = \abs{k}\norm{f} $ for $ k \in \C $. Lastly, we need to show the triangle inequality, which follows from the fact that for $ h(x) = f(x) + g(x) $ we have $ \abs{h(x)}  \leq \abs{f(x)} + \abs{g(x)}$. Then it follows from the properties of the suprimum that $ \norm{h} \leq \norm{f} + \norm{g} $.
\end{proof}

Thus we saw that $ C(X) $ is a normed linear space. The following Lemma shows that we can easily define $ C(X) $ to be also a metric space. 

\begin{lemma}
	Let $ V $ be a vector space with a norm $ \norm{\cdot} $. Then
	\[ d(x,y) = \norm{x-y}, \]
	satisfies the properties of a metric.
\end{lemma}
\begin{proof}
	To be added later.
\end{proof}

With the Lemma above in hand, we can have the following definitions. 

\begin{definition}
	Let $ C(X) $ be the set of all complex valued, bounded, and continuous functions defined on a metric space $ X $. $ C(X) $ is a metric space with a metric
	\[ d(f,g) = \sup_{x \in X}\abs{f(x)-g(x)}. \]
\end{definition}

With the definition above in hand, we can put \autoref{prop:uniformConvSup} in a different way

\begin{quote}
	A sequence $ \set{f} $ converges to $ f $ with respect to the metric of $ C(X) $ if and only if $ f_n \to f $ uniformally on $ X $.
\end{quote}
The following theorem is very central.
\begin{theorem}
	The metric space $  C(X) $ with metric
	\[ d(f,g) = \norm{f-g} \]
	where $ \norm{\cdot} $ is the sup-metric, is \textbf{complete}.
\end{theorem}
\begin{proof}
	The proof of this theorem is nothing more than gluing together the pieces of theorems and propositions that we have had before. Let $ \set{f_n} $ be a Cauchy sequence in $ C(X) $. Thus by \autoref{thm:cauchyConv}, we can conclude that converges uniformally to some $ f $. We now need to show that $  f\in C(X) $. First, since $ f_n $ is continuous for all $ n = 1,2,3,\cdots $, by \autoref{cor:UniformConvImpCont} we conclude that $ f $ is also continuous. Secondly, since $ f_n \to f $ uniformally, then for all $ n $ large enough we have
	$ \abs{f_n(x) - f(x)} < 1 $ for all $  x\in X $. Thus $ f $ is bounded, i.e. $ \sup_{x\in X}\abs{f(x)} < M $. This implies $ d(f,0) < M $ which implies $ f $ is also bounded (i.e. there exists some open ball with finite radius that contains $ f $). This shows that the limiting object $ f $ is both bounded and continuous. This $ f \in C(X) $. This implies that the metric space $ C(X) $ is complete. 
\end{proof}



\newpage
