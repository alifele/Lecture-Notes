\chapter{Sequences and Limits}


\begin{proposition}[Sequential characterization of countable sets]
	Let $X$ be a countable set. Then, $X$ is a range of some sequence $\{x_n\}$. To put it on other words, there is an \emph{onto} function $f: \N \to X$.
\end{proposition}

\begin{problem}
	Show that the following properties hold for arbitrary countable sets.
	\begin{enumerate}[(a)]
		\item All subsets of countable sets are countable.
		\item Any union of a pair of countable sets is countable.

	\end{enumerate}
\end{problem}

\begin{proof}
	~\vspace{2pt}
	\begin{enumerate}[(a)]
		\item Let $A$ be a countable set, and $C \subseteq A$ be a subset. Since $A$ is countable, then there is a numeration for its elements, i.e.
		\[ A = \{ a_1, a_2, \cdots \}. \]
		Then for $x \in C$, there exists $n \in \N$ such that $x = a_n$. Define $g: C \to \N$, and let $g(x) = n$. Since $g$ is one-to-one, then $C$ is at most countable.
		
		\item Let $A, B$ be countable sets, and $X = A \cup B$. If $A, B$ are finite, then the statement is trivially true. But if $A, B$ are infinite, then since $A$ and $B$ are countable, then there exists bijection
		\begin{align*}
			g_A: A \to \N, \\
			g_B: B \to \N.\\
		\end{align*}
		Let $x \in A \backslash B$. Then $g_A(x) = n$ for some $n \in \N$. Define $f: X \to \N$ and let $f(x) = 2n$. Also for $y \in B$, we have $g_B(y) = m$, then define $f(y) = 2m+1$. Due to the construction, the function $f$ is one-to-one, and since $\N$ is countable, then $X$ is at most countable.

	\end{enumerate}
\end{proof}


\begin{problem}
	\label{prob:unionOfCountableManyisCountable}
	Show that the following property holds for countable sets: If
	\[ S_1, S_2, S_3, \cdots \]
	is a sequence os countable sets of real numbers,then the set $S$ formed by taking all elements that belong to at least one of the sets $S_i$ is also a countable set. 
\end{problem}

\begin{proof}
	Let
	\[ X = \bigcup_{i \in \N} S_i. \]
	Let $x \in X$. Then there exists $I \subseteq \N$ such that $\forall i \in I$ we have $x \in S_i$. Let $m = \min(I)$. Note that due to the construction of $X$, the set $I$ is not empty, and since it is a subset of natural numbers, then there exists a unique minimum. So far, we have $x \in S_m$. Since $S_m$ is countable, then there exists $f_m: S_m \to \N$ one-to-one (note that since there is a possibility that $S_m$ might be finite, then requiring $f_m$ be one-to-one, given that the co-domain is countable is enough for expressing the fact $f_m$ is at most countable). Thus $f_m(x)=n$ for some $n \in \N$. Define $F: X \to \N \times \N$, and let $F(x) = (m,n)$. Do the the construction of $F$, it is one-to-one, and since $\N \times \N$ is countable, then $X$ is at most countable. 
\end{proof}


\begin{problem}
	Let $X$ be a family of sets, and $\sim$ a relation on $X$ defined as below,
	\[ A, B \in X,\ A\sim B \biImp \exists f: A \to B,\ \text{a bijection}. \]
	In words, when $A \sim B$ we can say $A$ and $B$ have the same cardinality. Show that $\sim$ is an equivalence relation. 
	
	\textbf{A side note:} There are lots of parallel notions for this in mathematics. For example in topology, a homeomorphism (bijective and continuous) define an equivalence relation on topological spaces, in differential geometry, a diffeomorphism does the job, in group theory, an isomorphism is the parallel notion, and in general, all of these notions are generalized with the notion of morphism in category theory.
\end{problem}

\begin{proof}
	~\vspace{2pt}
	\begin{enumerate}[(a)]
		\item \textbf{Reflexivity.} Let $A \in X$. Then $id_A: A \to A$, is an bijective map (identity map). Thus $A \sim A$.
		\item \textbf{Symmetry.} Let $A, B \in X$, and $A \sim B$. Thus we know $\exists f: A \to B$, a bijection. Let $g = f^{-1}: B \to  A$. Due to the construction $g$ is also bijective. Thus $B \sim A$.
		\item \textbf{Transitivity.} Let $A, B, C \in X$, $A \sim B$, and $B \sim A$. Then $\exists f: A\to B$ and $g: B \to C$ injective. Define $h: A \to B$, $h = g \circ f$. Due to the construction, $h$ is also bijective. Thus $A \sim C$.
	\end{enumerate}
\end{proof}


\begin{problem}
	We define a real number to be \emph{algebraic} is it is a solution of some polynomial equation
	\[ a_n x^n + a_{n-1}x^{n-1} + \cdots + a_1 x + a_0 = 0, \]
	where all the coefficients are integers. A real number that is not algebraic, is called a transcendental number. Prove that the set of all algebraic numbers is countable.
\end{problem}

\begin{proof}
	The set of all polynomials of degree $n-1$ is identified with $\Z^{n}$, which is countable. Thus the set all such polynomials is 
	\[  X = \bigcup_{n\in\N} \Z^{n}.\]
	As we proved in the Problem \autoref{prob:unionOfCountableManyisCountable}, $X$ is countable. On the other hand, every polynomial has a countable number of zeros. Thus The set of all zeros of all polynomial with integer coefficients is countable.
\end{proof}

\begin{problem}
	Let $\seq{s}$ be a sequence in $\R$ converging to $L \in \R$. Prove that the following definition is the equivalent to the original definition of converges.
	\[ \forall m \in \N,\ \exists N\in\N:\ \forall n> N \wh \abs{s_n - L} < 1/m.  \]
	\textbf{Side note.} This is a useful equivalent definition of convergence, because to check a sequence converges is enough to check the definition above for the integer $m$'s only.
\end{problem}
\begin{proof}
	The fact that the original definition implies the definition above is trivial. It suffice to let $\epsilon = 1/m$. However, for the converse, we assume that the definition above is true and we want to infer the original definition of convergence. Let $\epsilon>0$ is give. Then $\exists m \in \N$ such that $1/m < \epsilon$ (The Archimedes property for real numbers). On the other hand since we have assumed that the altered definition is true, them $\exists N \in \N$ such that $\forall n >N$ we have
	\[ \abs{s_n - L} < 1/m < \epsilon. \]
	This completes the proof.
\end{proof}


\begin{problem}
	If $\{s_n\}$ is a sequence of positive real numbers converging to a positive number $L$, show that $\{\sqrt{s_n}\}$ converges to $\sqrt{L}$.
\end{problem}
\begin{proof}
	Since $s_n \to L$, then $\exists N_1 > 0$ such that $\forall n > N_1$ we have $s_n < L+1$. Thus $\sqrt{s_n}<\sqrt{L+1}$, which implies $\sqrt{s_n} + \sqrt{L} < \sqrt{L+1} + \sqrt{L} = M$, for $M \in \R$. For a given $\epsilon>0$ let $\epsilon_0 = M \epsilon$. We know $\exists N_2>0$ such that $n>N_2$ we have $\abs{s_n - L} = \abs{\sqrt{s_n} - \sqrt{L}} \abs{\sqrt{s_n}+\sqrt{L}} < \epsilon_0$. Then $\abs{\sqrt{s_n} - \sqrt{L}} < \epsilon_0 /M = \epsilon $. This completes the proof.
\end{proof}


\begin{problem}
	Show that the sequence 
	\[ a_n = n^p + \alpha_1 n^{p-1} + \alpha_2 n^{p-2} + \cdots + \alpha_p  \]
	diverges to $\infty$, where $p$ is a positive integer and $\alpha_1, \alpha_2, \cdots, \alpha_p$ are real numbers (positive or negative). 
\end{problem}

\begin{proof}
	We construct the sequence $\{b_n\}$ such that $b_n \leq a_n$ always hold.
	\[ b_n = n^p - \abs{\alpha_1}n^{p-1} -\abs{\alpha_2}n^{p-2}  - \cdots - \abs{\alpha_p}. \]
	To demonstrate the proof in a more clear way, we assume $p=4$. Now we can solve the following inequalities
	\begin{align*}
		1/4 n^4 - \abs{\alpha_1}n^3 > 1/8 n^4 &\implies n > \abs{8\alpha_1},\\
		1/4 n^4 - \abs{\alpha_2}n^2 > 1/8 n^4 &\implies n > \abs{8\alpha_2}^{1/2},\\
		1/4 n^4 - \abs{\alpha_3}n^1 > 1/8 n^4 &\implies n > \abs{8\alpha_3}^{1/3},\\
		1/4 n^4 - \abs{\alpha_4}n > 1/8 n^4 &\implies n > \abs{8\alpha_4}^{1/4},\\
	\end{align*}
	Let $N = \max\{ \abs{8\alpha_1}, \abs{8\alpha_2}^{1/2},\abs{8\alpha_3}^{1/3},\abs{8\alpha_4}^{1/4}\}$. Then for $n > N$ all of the inequalities above holds. Thus we can write
	\[  n^4 + \alpha_1 n^3 + \alpha_2 n^2 + \alpha_3 n + \alpha_4 > n^4 - \abs{\alpha_1}n^3 - \abs{\alpha_2}n^2 - \abs{\alpha_3} n - \abs{\alpha_4} > 1/2 n^4. \]
	Since the very last term ($1/2n^4$) diverges, then the sequence $a_n$ diverges.
\end{proof}


\begin{problem}
	Prove that if $\seq{s}$ bounded, then $\{s_n/n\}$ converges.
\end{problem}
\begin{proof}
	We propose that the $s_n/n \to 0$. Since $s_n$ is bounded, then there is $M>0$ such that $\abs{s_n}<M$ for all $n \in \N$. For a given $\epsilon>0$, let choose a natural number $N>M/\epsilon$. Then $\forall n>N$ we have
	\[ \abs{s_n/n} < \abs{s_n}/\abs{n} < M/\abs{n} < M\epsilon/M = \epsilon. \]
	This implies that $\{s_n/n\}$ converges to zero.
\end{proof}

\begin{problem}[Order property in limits]
	Let $\seq{t}$ and $\seq{s}$ be two real sequences that converge to $T$ and $S$ respectively, and also $\forall n \in \N$ we have $s_n \leq t_n$. Prove that $T \leq S$
	
	\textbf{Side note:} By this proof you will show that the limits preserve the order property, and this fact is the main idea behind the squeeze theorem.
\end{problem} 

\begin{proof}
	Since $t_n \geq s_n$ for all $n\in\N$, then $t_n - s_n \geq 0$. Thus $t_n - s_n \pm T \pm S < 0$, and we can re-arrange the terms as $(t_n - T) + (S - s_n) + (T-S) >0$. On the other hand, since $s_n \to S$ and $t_n \to T$, then for $\epsilon>0$ there exists $N_1 > 0, N_2 > 0$ such that $n>N_1$ implies $\abs{s_n - S} < \epsilon/2$ and $n>N_2$ implies $\abs{t_n - T} < \epsilon/2$. Let $N = \max\{N_1,N_2\}$. So $\forall n>N$ we will have
	\[ \epsilon/2 + \epsilon/2 + (T-S) > (t_n - T) + (S - s_n) + (T-S) > 0. \]
	This implies $T-S > -\epsilon$ for all $\epsilon>0$. Thus we conclude $T- S \geq 0 \implies T > S$.
\end{proof}


\begin{problem}
	A careless student gives the following as a proof of the squeeze theorem. Find the flaw:
	
	"If \(\lim_{{n \to \infty}} s_n = \lim_{{n \to \infty}} t_n = L\), then take limits in the inequality \(s_n \leq x_n \leq t_n\) to get \(L \leq \lim_{{n \to \infty}} x_n \leq L\). This can only be true if \(\lim_{{n \to \infty}} x_n = L\)."
	
\end{problem}
\begin{proof}
	Formally, the student turns the inequality into two pieces, i.e. $s_n \leq x_n$ and $x_n \leq t_n$ and tries to apply the order property of limits to each of them. But the problem is that the squeeze theorem only states that $s_n$ and $t_n$ has limits and does not have any information about the existence of limit for $x_n$. Thus we can not simply apply the limit to the inequality and infer the squeeze theorem.
\end{proof}

\begin{problem}
	Let $\{s_n\}$ be a sequence of positive numbers. Show that the condition
	\[ \lim_{n\to\infty} \frac{s_{n+1}}{s_n} < 1 \]
	implies $s_n \to 0$.
\end{problem}
\begin{proof}
	Let $\lim_{n\to\infty}\frac{s_{n+1}}{s_n} = \alpha < 1 $. Also, since $\forall n\in\N$ we have $s_n > 0$, then $\alpha\geq0$. So $0 \leq \alpha < 1$. Define $d = (1-\alpha)/2$, thus $1/2 \leq d < 1$ and $d > \alpha$ (in fact, $d$ is sandwiched between $\alpha$ and 1). Since $s_{n+1}/s_n \to \alpha$, then $\exists N_1 > 0$ such that $\forall n>N_1$ we have
	\[ \frac{s_{n+1}}{s_n} < d \implies s_{n+1}<d s_n \implies s_{n+1}< d s_n < d^2 s_{n-1} < \cdots < d^n s_1. \]
	Since all $s_n$ are positive, then $s_1 = c d$, for some $c \in \R$. Thus we can write the inequality above as
	\[ s_{n+1} < d^{n+1}c \implies s_n < d^n c. \]
	Now we can proceed with the $\epsilon-N$ proof directly, or we can use squeeze theorem. We know that 
	\[ 0 \leq s_n < d^n c. \]
	Since both $0$ and $d^n$ converges to zero, then $s_n$ also converges to zero, i.e. $s_n \to 0$.
\end{proof}

\begin{problem}
	Let $\{s_n\}$ be a sequence of positive numbers. Show that the condition
	\[ \lim_{n\to\infty} \frac{s_{n+1}}{s_n} > 1 \]
	then $s_n \to \infty$.
\end{problem}

\begin{proof}
	Let $\lim_{n\to\infty} \frac{s_{n+1}}{s_n} = \alpha > 1$. Define $d = 1 + (\alpha - 1)/2$ (in fact $d$ is sandwiched between 1 and $\alpha$). Since $s_{n+1}/s_n \to \alpha$, then $\exists N>0$ such that $\forall n>N$ we have $s_{n+1}/s_n > d$. As we seen in the solution of the problem above, this implies 
	\[ s_n > d^n c, \]
	where $c = s_1/d$. Now we can proceed with the $M-N$ proof to show that $s_n$ actually diverges.
\end{proof}

\begin{problem}
	Let $\{s_n\}$ be a real sequence that is non-decreasing and bounded. Then prove that 
	\[ s_n \to \sup_{n\in\N}(s_n). \]
\end{problem}
\begin{proof}
	Let $L = \sup_{n\to \infty} s_n$. Since $s_n \in \R$ for all $n \in \N$, and is bounded, then the suprimum exists and $L \in \R$. Let $\epsilon>0$ be given, and define $\beta = L - \epsilon$. Since $L$ is the least upper bound for the sequence, then for $\exists N \in \N$ such that $s_N > \beta$, and since $s_n$ is non-decreasing, $\forall n>N$ we have $s_n > \beta \implies \abs{s_n - L} < \epsilon$. This implies that $s_n \to L$ as $n\to\infty$.
\end{proof}

\begin{problem}
	Decide on the convergence of the following sequence.
	\[ \sqrt{2}, \sqrt{2+\sqrt{2}}, \sqrt{2+\sqrt{2+\sqrt{2}}}, \cdots \]
\end{problem}

\begin{proof}
	This sequence converges, since it is increasing and bounded. The fact that this sequence is increasing is trivial, but for being bounded we start with
	\[ \boxed{\sqrt{2}< 2}  \implies 2+\sqrt{2} < 4 \implies \boxed{\sqrt{2+\sqrt{2}} < 2} \implies 2+\sqrt{2+\sqrt{2}}<4 \implies \boxed{\sqrt{2+\sqrt{2+\sqrt{2}}<2}} \implies \cdots. \]
	Thus as we can see, all of the terms of the sequence is less than 2. Thus this sequence converges.
\end{proof}