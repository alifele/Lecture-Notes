\chapter{Sets and Mappings}

\begin{remark}
	Let $\R_+$ denote the real number greater than or equal to zero\footnote{This note is from Segel, undergraduate analysis.}. The we can view the association $x \mapsto x^2$ as a map from $R$ to $\R_+$. When viewed so, the map is the surjective. Thus it is a reasonable convention not to identify this map with the map $f:\R \to \R$ defined by the same formula. To be completely accurate, we should therefore denote the set of arrival and the set of departure of the map into our notation, and for instance write
	\[ F^S_T: S \to T, \]
	instead of our $f: S \to T$ notation. In practice, this notation is too clumsy, so that we omit the indices $S, T.$ However, the reader should keep in mind the distinction between the maps 
	\[ f^\R_{\R_+}: \R \to \R_+ \quad \text{and} \quad f_\R^\R: \R \to \R  \]
	both defined by the association $x \mapsto x^2$. The first map is surjective, while the second one is not. Similarly the maps
	\[ f_{\R_+}^{R_+}: \R_+ \to \R_+ \quad \text{and} \quad f_\R^{\R_+}: \R_+ \to \R \]
	defined by the same association $x \mapsto x^2$ are injective.
\end{remark}

\begin{remark}

\end{remark}