\begin{question} 
	If $A, B$ and $C$ are sets. prove the followings:
	\begin{enumerate}[(a)]
		\item $A \cap (B \cup C) = (A \cap B) \cup (A \cap C)$. \label{section-a}
		\item $C \textbackslash (A \cup B) = (C \textbackslash A) \cap (C \backslash B)$.
		
		\item $C \backslash (A \cap B) = (C \backslash A) \cup (C \backslash B) $.
		
	\end{enumerate}
	
	\begin{ans}
		\begin{enumerate}[(a)]
			\item 
			\begin{proof}
			Let $P, Q$, and $R$ be logical statements. We can show (using the truth table) that the following biconditional implication is a tautology. \[ P \wedge (Q \vee R) \Leftrightarrow (P \wedge Q) \vee (P \wedge R). \]
			Now let $x \in A \cap (B \cup C)$. Then $x \in A \wedge (x \in B \vee x \in C)$. Using the tautology above we can write $(x \in A \wedge x \in B) \vee (x \in A \wedge x \in C)$, thus $x \in (A \cap B) \cup (A \cap C) $, which means $A \cap (B \cap C) \subset (A \cap B) \cup (A \cap C)$. Conversely, let $x \in (A \cap B) \cup (A \cap C)$. By definition $(x \in A \wedge x \in B) \vee (x \in A \wedge x \in C)$. With the similar logic as above we can infer $(A \cap B) \cup (A \cap C) \subset A \cap (B \cup C)$. Thus $A \cap (B \cup C) = (A \cap B) \cup (A \cap C)$.
			\end{proof}
			
			\item 
			\begin{proof}
			Let $x \in C \backslash (A \cup B)$. By definition $x \in C \wedge x \notin (A \cup B) \biImp x \in C \wedge x \in \overline{A \cup B} \biImp x \in C \wedge x \in \bar{A} \cap \bar{B} \biImp x \in C \wedge (x \notin A \wedge x \notin B)$. Finally, using the following tautology 
			\[ P \wedge (Q \wedge R) \Leftrightarrow (P \wedge Q) \wedge (P \wedge R), \]
			we can write $(x \in C \wedge x \notin A) \wedge (x \in C \wedge x \notin B) \biImp x \in (C \backslash A) \cap (C \backslash B) $, thus $C \textbackslash (A \cup B) subset (C \textbackslash A) \cap (C \backslash B)$. The converse can be shown is true following the similar logic as below, thus inferring $C \textbackslash (A \cup B) = (C \textbackslash A) \cap (C \backslash B)$.
			\end{proof}
			
			\item \begin{proof}
			Let $x \in C \backslash (A \cap B)$. By definition $x \in C \wedge x \notin (A \cap B) \biImp x \in C \wedge x \in \overline{A \cap B} \biImp x \in C \wedge x \in \bar{A} \cup \bar{B} \biImp x \in C \wedge (x \notin A \vee x \notin B)$. Using the tautology in section (a),
			we can write $(x \in C \wedge x \notin A) \vee (x \in C \wedge x \notin B) \biImp x \in (C \backslash A) \cup (C \backslash B) $, thus $C \textbackslash (A \cap B) \subset (C \textbackslash A) \cup (C \backslash B)$. The converse can be shown is true following the similar logic as below, thus inferring $C \textbackslash (A \cap B) = (C \textbackslash A) \cup (C \backslash B)$.
			\end{proof}
			
			
			
			
		\end{enumerate}
	\end{ans}
\end{question}